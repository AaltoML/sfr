%!TeX program = lualatex

\documentclass[final,12pt]{beamer}

% ====================
% Packages
% ====================

% Packages
\usepackage[T1]{fontenc}
\usepackage{lmodern}
\usepackage[size=a1,orientation=portrait,scale=1.4]{beamerposter}
\usepackage{amsfonts}       % blackboard math symbols
\usepackage{amsmath}
\usepackage{mathtools}
\usetheme{aalto}
\usecolortheme{aalto}
\usepackage{graphicx}
%\usepackage[square]{natbib}
%\setlength{\bibsep}{4pt plus 0.3ex}  % Squeeze line spacing in references
\usepackage{booktabs}
% Tikz
\usepackage{tikz}
\usetikzlibrary{patterns}
\usetikzlibrary{decorations,backgrounds,arrows.meta,calc}
\usetikzlibrary{shapes,arrows,positioning}
\usepackage{pgfplots}
\usepackage{subcaption}
\usetikzlibrary{}
\usepackage{chronosys}
\newcommand{\blue}[1]{\textcolor{aaltoblue}{#1}}
\newcommand{\black}[1]{\textcolor{black}{#1}}
%\usepackage{color}
\usepackage{xcolor}         % colors


% Array/table packages
\usepackage{tabularx}
\usepackage{array,multirow}
\usepackage{colortbl}
\newcommand{\PreserveBackslash}[1]{\let\temp=\\#1\let\\=\temp}
\newcolumntype{C}[1]{>{\PreserveBackslash\centering}p{#1}}
\newlength{\tblw}

% Our method
\newcommand{\our}{SFR}
	
% Custom error formatting
%\newcommand{\val}[2]{ $#1$\textcolor{gray}{\tiny ${\pm}#2$}} 



% Packages for bold math
\usepackage{bm}
\newcommand{\mathbold}[1]{\bm{#1}}
\newcommand{\mbf}[1]{\mathbf{#1}}
\renewcommand{\mid}{\,|\,}


% Variables
\newcommand{\state}{\ensuremath{\mathbf{s}}}
\newcommand{\noise}{\ensuremath{\bm\epsilon}}
\newcommand{\discount}{\ensuremath{\gamma}}
\newcommand{\inducingInput}{\ensuremath{\mathbf{Z}}}
\newcommand{\inducingVariable}{\ensuremath{\mathbf{u}}}
\newcommand{\dataset}{\ensuremath{\mathcal{D}}}
\newcommand{\dualParam}[1]{\ensuremath{\bm{\lambda}_{#1}}}
\newcommand{\meanParam}[1]{\ensuremath{\bm{\mu}_{#1}}}

% Indexes
\newcommand{\horizon}{\ensuremath{h}}
\newcommand{\Horizon}{\ensuremath{H}}
\newcommand{\numDataNew}{\ensuremath{N^{\text{new}}}}
\newcommand{\numDataOld}{\ensuremath{N^{\text{old}}}}
\newcommand{\numInducing}{\ensuremath{M}}

% Domains
\newcommand{\stateDomain}{\ensuremath{\mathcal{S}}}
\newcommand{\actionDomain}{\ensuremath{\mathcal{A}}}
\newcommand{\inputDomain}{\ensuremath{\mathbb{R}^{D}}}
\newcommand{\outputDomain}{\ensuremath{\mathbb{R}^{C}}}
\newcommand{\policyDomain}{\ensuremath{\Pi}}


% Functions
\newcommand{\rewardFn}{\ensuremath{r}}
\newcommand{\transitionFn}{\ensuremath{f}}
\newcommand{\latentFn}{\ensuremath{f}}

\newcommand{\optimisticTransition}{\ensuremath{\hat{f}}}
\newcommand{\optimisticTransitionMean}{\ensuremath{\mu_{\optimisticTransition}}}
\newcommand{\optimisticTransitionCov}{\ensuremath{\mu_{\optimisticTransition}}}
\newcommand{\optimisticTransitionSet}{\ensuremath{\mathcal{M}}}


% Parameters
% \newcommand{\weights}{\ensuremath{\bm\phi}}
\newcommand{\weights}{\ensuremath{\mathbf{w}}}
\newcommand{\valueFnParams}{\ensuremath{\psi}}
\newcommand{\policyParams}{\ensuremath{\theta}}

% Networks
\newcommand{\transitionFnWithParams}{\ensuremath{\transitionFn_{\weights}}}
\newcommand{\valueFn}{\ensuremath{\mathbf{Q}}}
\newcommand{\stateValueFn}{\ensuremath{\mathbf{V}}}
% \newcommand{\valueFn}{\ensuremath{\mathbf{Q}_{\valueFnParams}}}
\newcommand{\policy}{\ensuremath{\pi}}
\newcommand{\pPolicy}{\ensuremath{\pi_{\policyParams}}}

% Math Macros
\newcommand{\MB}{\mbf{B}}
\newcommand{\MC}{\mbf{C}}
\newcommand{\MZ}{\mbf{Z}}
\newcommand{\MV}{\mbf{V}}
\newcommand{\MX}{\mbf{X}}
\newcommand{\MA}{\mbf{A}}
\newcommand{\MK}{\mbf{K}}
\newcommand{\MI}{\mbf{I}}
\newcommand{\MH}{\mbf{H}}
\newcommand{\T}{\top}
\newcommand{\vzeros}{\mbf{0}}
\newcommand{\vtheta}[0]{\mathbold{\theta}}
\newcommand{\valpha}[0]{\mathbold{\alpha}}
\newcommand{\vkappa}[0]{\mathbold{\kappa}}
\newcommand{\vbeta}[0]{\mathbold{\beta}}
\newcommand{\MBeta}[0]{\mathbold{B}}
\newcommand{\vlambda}[0]{\mathbold{\lambda}}
\newcommand{\diag}{\text{{diag}}}

\newcommand{\vm}{\mbf{m}}
\newcommand{\vz}{\mbf{z}}
\newcommand{\vf}{\mbf{f}}
\newcommand{\vu}{\mbf{u}}
\newcommand{\vx}{\mbf{x}}
\newcommand{\vy}{\mbf{y}}
\newcommand{\vw}{\mbf{w}}
\newcommand{\va}{\mbf{a}}

\newcommand{\Jac}[2]{\mathcal{J}_{#1}(#2)}
\newcommand{\JacT}[2]{\mathcal{J}_{#1}^\top(#2)}


\newcommand{\GP}{\mathcal{GP}}
\newcommand{\KL}[2]{\mathrm{D}_\textrm{KL} \dbar*{#1}{#2}}
\newcommand{\MKzz}{\mbf{K}_{\mbf{z}\mbf{z}}}
\newcommand{\MKzzc}{\mbf{K}_{\mbf{z}\mbf{z}, c}}
\newcommand{\MKxx}{\mbf{K}_{\mbf{x}\mbf{x}}}
\newcommand{\MKzx}{\mbf{K}_{\mbf{z}\mbf{x}}}
\newcommand{\MKxz}{\mbf{K}_{\mbf{x}\mbf{z}}}
\newcommand{\vkzi}{\mbf{k}_{\mbf{z}i}}
\newcommand{\vkzic}{\mbf{k}_{\mbf{z}i,c}}
\newcommand{\vkzs}{\mbf{k}_{\mbf{z}i}}
\newcommand{\vk}{\mbf{k}}
\newcommand{\MLambda}[0]{\mathbold{\Lambda}}
\newcommand{\MSigma}[0]{\mathbold{\Sigma}}
\newcommand{\N}{\mathrm{N}}
\newcommand{\R}{\mathbb{R}}
\newcommand{\myexpect}{\mathbb{E}}

\DeclareMathOperator*{\argmax}{arg\,max}
\DeclareMathOperator*{\argmin}{arg\,min}
\newcommand{\Norm}{\mathcal{N}}

% ====================
% Lengths
% ====================

% If you have N columns, choose \sepwidth and \colwidth such that
% (N+1)*\sepwidth + N*\colwidth = \paperwidth
\newlength{\sepwidth}
\newlength{\colwidth}
\setlength{\sepwidth}{0.025\paperwidth}
\setlength{\colwidth}{0.45\paperwidth}
\newcommand{\separatorcolumn}{\begin{column}{\sepwidth}\end{column}}

% Math and miscellaneous
\renewcommand{\mid}[0]{\,|\,}


% Config for Arno's awesome TikZ plotting stuff
\newlength{\figurewidth}
\newlength{\figureheight}


% ====================
% Title
% ====================

\title{Sparse Function-space Representation \\ of Neural Networks}

\author{%
  Aidan Scannell\textsuperscript{\star}\,\inst{1}\,\inst{2} \quad
  Riccardo Mereu\textsuperscript{\star}\,\inst{1} \quad
  Paul Chang\,\inst{1} \\
  Ella Tamir\,\inst{1}\quad
  Joni Pajarinen\,\inst{1}\quad
  Arno Solin\,\inst{1}
}

%\author{Your Name\inst{1} ~~ Collaborator Name\inst{2} ~~ Arno Solin\inst{1}}

\institute[shortinst]{ \inst{1}Aalto University \qquad \inst{2}Finnish Center for Artificial Intelligence}


% ====================
% Footer (optional)
% ====================

\footercontent{
  International Conference in Machine Learning (ICML 2023) --- Hawaii, US \hfill
  \href{mailto:aidan.scannell@aalto.fi}{aidan.scannell@aalto.fi}}

% ====================
% Body
% ====================

\begin{document}
\begin{frame}[t]
\begin{columns}[t]
\begin{column}{2.05\colwidth}
\begin{alertblock}{Summary}
Deep neural networks have limitations in \textit{estimating uncertainty}, \textit{incorporating new data}, and \textit{avoiding catastrophic forgetting}. To overcome these issues, we introduce a method that converts neural networks from weight-space to a \alert{\bf low-rank function-space} representation using \alert{\bf dual parameters}. Unlike previous methods, our approach, named \alert{\bf Sparse Function Representation (SFR)}, captures the \alert{\bf full joint distribution} of the entire data set, not just a subset. This allows for a concise and reliable way of capturing uncertainty and facilitates the inclusion of new data without the need for retraining. We provide a proof-of-concept quantifying uncertainty for supervised learning tasks on UCI benchmark data sets.
%{\bf \alert{TODO: Rewrite}}
%  Deep neural networks are known to lack uncertainty estimates, struggle to incorporate new data, and suffer from catastrophic forgetting. We present a method that mitigates these issues by converting neural networks from weight-space to a low-rank function-space representation, via the so-called dual parameters. In contrast to previous work, our sparse representation captures the joint distribution over the entire data set, rather than only over a subset. This offers a compact and principled way of capturing uncertainty and enables us to incorporate new data without retraining whilst retaining predictive performance. We provide proof-of-concept demonstrations with the proposed approach for quantifying uncertainty in supervised learning on UCI benchmark tasks.
  \end{alertblock}\end{column}
\end{columns}
\begin{columns}[t]


\separatorcolumn

\begin{column}{\colwidth}
  \begin{minipage}{\textwidth}
    \vspace*{-26pt}
  	%\begin{figure}[t]
    \pgfplotsset{axis on top,scale only axis,width=\figurewidth,height=\figureheight, ylabel near ticks,ylabel style={yshift=0pt},xlabel style={yshift=-4pt},y tick label style={rotate=90},legend style={nodes={scale=.8, transform shape}},tick label style={font=\footnotesize},xtick={0,.4,.8,1.2,1.6,2.}}
  \pgfplotsset{xlabel={Input, $x$},axis line style={rounded corners=2pt}}
  %
  % Hack pgfplots addplot options
  \pgfplotsset{every axis plot/.append style={mark size=5,line width=1.2pt}}
  %
  % Set figure 
  \setlength{\figurewidth}{.4\textwidth}
  \setlength{\figureheight}{.31\textwidth}
  %
  \def\inducing{\large Sparse inducing points}
  %
  \begin{minipage}[c]{.45\textwidth}
    \raggedleft
    \pgfplotsset{ylabel={Output, $y$}}
    % This file was created with tikzplotlib v0.10.1.
\begin{tikzpicture}

\definecolor{darkgray176}{RGB}{176,176,176}
\definecolor{lightgray204}{RGB}{204,204,204}

\begin{axis}[
height=\figureheight,
legend cell align={left},
legend style={fill opacity=0.8, draw opacity=1, text opacity=1, draw=lightgray204},
tick align=outside,
tick pos=left,
width=\figurewidth,
x grid style={darkgray176},
xmin=-0.2, xmax=2.2,
xtick style={color=black},
y grid style={darkgray176},
ymin=-5.2, ymax=7,
ytick style={color=black}
]
\addplot [draw=black, fill=black, mark=+, only marks, opacity=0.5]
table{%
x  y
0.59907633700711 0.223154562086841
0.34361536741137 1.80796222965741
1.21549153760962 0.967955764303031
1.39127490332365 0.346027494892729
0.527962441193321 2.55278258834587
0.0333065151243783 5.67346791281363
1.41384715629159 0.881468575721158
0.663171303916053 -0.472049787519171
1.00205910730797 -1.96751300267146
0.420018112144075 1.21728173287471
0.631458340576862 1.08436032794776
0.729874319093148 -0.101521243817772
0.147377188445248 5.13582753705874
0.403831786198692 2.27647279843891
0.478013434900342 1.17179814470945
0.595830829634752 2.23305061724589
0.955601779278568 -1.72905380014215
0.0839045702835102 6.3929614602345
1.45472441165539 -0.0760227450398461
0.278479698750005 2.84103829741965
0.380362104273414 2.11325866978181
0.464499503069764 1.2399605663803
0.793107714094262 -1.35173014471992
0.977768289229485 -2.55927294171402
0.569210172057486 0.68494103913613
0.36980976303721 0.921314385700647
0.0214569812169767 5.68468381545469
0.424236769219515 1.63566013099165
1.37938987893611 0.868125317911011
0.137047115400012 4.58721934956631
1.08761751839489 -0.172567506567352
0.419817741770634 2.24046262513834
1.4262841184116 0.159812522509929
1.41564972631546 0.78558633647277
0.495989002819549 1.10158598995442
0.995026761218884 -1.37546551579343
1.12639463876016 -0.334794924441981
1.07766860579985 -0.792376161207397
0.413701619771477 1.4913623618045
0.942986552597874 -2.4722425346432
0.763519126454151 -0.583688218859911
0.130904366804652 4.18156576167409
0.0454384945865041 5.94112904944771
0.522204951265 1.21564815606939
0.332942022904396 2.65994126760993
1.40363369449562 1.23353940603206
0.779310595942776 -1.02035508529735
1.29407529214362 1.06264876539534
0.556033388135621 0.837778673324579
0.484633763104316 2.32967153874662
0.0920321107558388 5.09987558004654
0.56325131198675 1.98910435881925
1.45678884132294 0.154899001388011
0.408702868409535 1.25289558714472
0.458097584655798 1.52690656054213
1.46582759713057 0.56502069721856
0.507184083096661 1.52564677260393
1.36046218960706 0.507926246238466
0.921764573531882 -2.09142805291535
0.238639274847773 2.6411155956647
0.362559658595395 1.79727559711587
1.21589080021465 0.400000674361935
0.933585205353682 -1.55461336837348
0.982715141501147 -2.45533624593891
0.401490860442604 1.05987192830147
0.156059199998835 4.36370190980214
0.891378994342085 -2.36391299232751
0.388158878064094 0.788154469107815
0.667188047358641 1.10183787343592
0.625170599258625 0.6586301610886
0.957021857615636 -1.85051954010725
1.02156844318979 -1.83957136491289
1.25724919181647 0.322518415235646
0.00266557165722991 6.50557346836786
1.41816728826336 0.629351904749414
1.31945988198246 0.544524900201801
1.25294153993982 1.58846644417793
1.44809592439576 -0.343645665284646
1.12607739077688 0.0198156968068374
0.251530997239499 3.26092402478648
0.729180476405212 -0.261848785696243
0.902909475880636 -1.88887529622365
0.815297425524211 -0.263213218540796
0.399596820896549 1.8350776647408
1.19233932331138 0.594739208851601
0.212701803984358 3.69662943902046
0.48579647389144 0.781131500953395
0.754043399884807 -0.120330706281183
0.299033951960917 2.59001615301901
0.471011920030608 0.803031316713116
1.23848755238611 1.20117380925507
0.0927035953462867 5.14006372858808
0.803635270801682 -1.64573382353524
0.971532479828419 -1.26141546874724
0.866704965181163 -1.37016791118963
1.36729265920189 0.597849962429656
0.519752323523261 1.89624294678323
0.0191107582149252 5.33379579237503
0.196986770599228 4.03647401377799
0.952624784132827 -2.16148044060619
0.0015044650302265 5.2886136068202
0.902037550168227 -2.52858794998143
1.22749030238651 0.0464677514177437
1.21792105205884 -0.141830633035098
0.0102856385216306 6.14133785166822
0.863182937449097 -2.1011551265248
0.818564275101864 -1.01534185724427
0.302635244152368 1.46210778674698
0.372879666668875 1.40476340640872
0.889792662887503 -1.17727077360789
1.29641079289544 0.586535140325398
1.3272574482414 1.3843767453822
0.214325965580719 3.31884710039265
0.614137818457737 1.59184679724417
0.368321423663743 0.845629772651128
0.380933232697981 0.723731284253583
0.976159176116351 -1.93326574524374
0.857441599514585 -1.60433853663788
0.83708880960736 -2.04818343189899
1.47579759938635 -0.0805765439070799
0.442438891644387 1.73835028751105
1.38953493744193 0.417026784527823
1.14080627628873 0.550112184891199
1.48114359376883 0.298669829133318
0.736597976181201 -0.643077812882977
0.394694513831174 0.936428225891781
1.26908189826216 0.858430359609534
0.0672098890730217 5.09900003215748
0.333851467177716 3.01981294882107
0.945969964783799 -1.83629872793283
0.692843988254517 0.433249498499454
1.26598975868233 1.28216661393988
0.391972523919673 1.09693715415414
0.46917601169147 1.50263243659036
1.01098339289849 -2.31817336136178
0.777738706583271 -0.838717602177543
1.12209237980542 -0.319615415703505
1.17055379105428 0.512465979308082
0.258836084652896 2.78755665557385
0.61396534548555 1.69052927515609
1.31770140735241 1.16502121252192
0.116000397751632 4.99444756152836
0.210700698077648 3.49050340807256
1.05114996874651 -1.76278149811195
0.670454848751649 0.224613670765436
1.92406486757113 -0.108425407068489
1.94209538616793 1.7264746642509
1.99393444419407 0.403999409461401
1.95439877208446 0.827632344838344
1.98473609767015 0.0244378792655042
1.9054919129869 0.855315202370139
1.94542261961925 0.240943330223005
1.96246848086935 0.398027059259599
};
\addlegendentry{Data}
\addplot [semithick, red]
table {%
-0.2 6.2857973514481
-0.187939698492462 6.28987893254762
-0.175879396984925 6.29181588372522
-0.163819095477387 6.29143317385438
-0.151758793969849 6.28854013004549
-0.139698492462312 6.28292926239899
-0.127638190954774 6.27437510005683
-0.115577889447236 6.26263307579898
-0.103517587939698 6.24743850896844
-0.0914572864321608 6.22850575195502
-0.0793969849246231 6.20552758417457
-0.0673366834170854 6.17817495967054
-0.0552763819095477 6.14609724012402
-0.0432160804020101 6.10892307376503
-0.0311557788944724 6.06626211137595
-0.0190954773869347 6.01770778129061
-0.00703517587939698 5.96284137276835
0.00502512562814073 5.90123769647965
0.0170854271356784 5.83247259523151
0.0291457286432161 5.75613255854249
0.0412060301507538 5.67182664040638
0.0532663316582915 5.5792007786667
0.0653266331658292 5.47795445565633
0.0773869346733668 5.36785941553933
0.0894472361809046 5.24877986427611
0.101507537688442 5.12069323625788
0.11356783919598 4.9837102481237
0.125628140703518 4.83809262672679
0.137688442211055 4.68426666625203
0.149748743718593 4.52283072306446
0.161809045226131 4.35455497673142
0.173869346733668 4.18037232676687
0.185929648241206 4.00136016107162
0.197989949748744 3.81871385699869
0.210050251256281 3.63371411567772
0.222110552763819 3.44769137777358
0.234170854271357 3.26199138801039
0.246231155778894 3.0779462474231
0.258291457286432 2.89685486141262
0.27035175879397 2.71997550441124
0.282412060301508 2.54853134759842
0.294472361809045 2.38372743454599
0.306532663316583 2.2267750630501
0.318592964824121 2.07891724952624
0.330653266331658 1.94144733602477
0.342713567839196 1.81571216298775
0.354773869346734 1.70309166873519
0.366834170854271 1.60494816112118
0.378894472361809 1.52254067772778
0.390954773869347 1.45690296965451
0.403015075376884 1.40868845690316
0.415075376884422 1.37799311124833
0.42713567839196 1.36417806680222
0.439195979899497 1.36572605810096
0.451256281407035 1.38017446012216
0.463316582914573 1.40416536568561
0.475376884422111 1.43363406884371
0.487437185929648 1.46412297935564
0.499497487437186 1.49116970386928
0.511557788944724 1.51069284803595
0.523618090452261 1.51929979398149
0.535678391959799 1.51446714403481
0.547738693467337 1.49458367539806
0.559798994974875 1.45888030004499
0.571859296482412 1.40729020308166
0.58391959798995 1.3402834770427
0.595979899497487 1.2587099228498
0.608040201005025 1.16366898121746
0.620100502512563 1.05641276278727
0.632160804020101 0.938279469283438
0.644221105527638 0.81065027839526
0.656281407035176 0.674921907157222
0.668341708542714 0.53248820908157
0.680402010050251 0.384726155567839
0.692462311557789 0.232983617124391
0.704522613065327 0.0785680401076337
0.716582914572864 -0.0772637885233033
0.728643216080402 -0.233314196261137
0.74070351758794 -0.388450594830113
0.752763819095478 -0.541608554626875
0.764824120603015 -0.691791981866757
0.776884422110553 -0.838070751640898
0.788944723618091 -0.979576344982615
0.801005025125628 -1.11549603458366
0.813065326633166 -1.24506598531361
0.825125628140704 -1.36756334671807
0.837185929648241 -1.48229709669328
0.849246231155779 -1.58859712629165
0.861306532663317 -1.68580089703069
0.873366834170854 -1.773236998834
0.885427135678392 -1.85020512459456
0.89748743718593 -1.915952398347
0.909547738693467 -1.96964671375817
0.921608040201005 -2.01034885974521
0.933668341708543 -2.03698686496384
0.945728643216081 -2.0483383129386
0.957788944723618 -2.04302939259167
0.969849246231156 -2.01956288769164
0.981909547738694 -1.97639030730567
0.993969849246231 -1.91204409703886
1.00603015075377 -1.82534140335192
1.01809045226131 -1.7156576560425
1.03015075376884 -1.5832440846665
1.04221105527638 -1.42953085004441
1.05427135678392 -1.25732787946063
1.06633165829146 -1.07082794450782
1.078391959799 -0.875349345818019
1.09045226130653 -0.676830101549934
1.10251256281407 -0.481174989533318
1.11457286432161 -0.293615776282351
1.12663316582915 -0.118239365629348
1.13869346733668 0.0422281830821599
1.15075376884422 0.186383598074527
1.16281407035176 0.313924893349873
1.1748743718593 0.425342584611239
1.18693467336683 0.521607339752872
1.19899497487437 0.603906758694623
1.21105527638191 0.673452590391777
1.22311557788945 0.731357540851534
1.23517587939698 0.778569491602677
1.24723618090452 0.815847807322585
1.25929648241206 0.843767921655468
1.2713567839196 0.862743743355362
1.28341708542714 0.873060917720606
1.29547738693467 0.874916788169473
1.30753768844221 0.868464737975285
1.31959798994975 0.853861424167359
1.33165829145729 0.831315338883324
1.34371859296482 0.801134330340053
1.35577889447236 0.763768482133921
1.3678391959799 0.719843528247159
1.37989949748744 0.670179340507215
1.39195979899498 0.615788538744166
1.40402010050251 0.557852295893885
1.41608040201005 0.497673827854153
1.42814070351759 0.436614167370649
1.44020100502513 0.376018470128547
1.45226130653266 0.317143106662216
1.4643216080402 0.261093458268937
1.47638190954774 0.208779774388998
1.48844221105528 0.160894532365159
1.50050251256281 0.11791068459771
1.51256281407035 0.0800970364996313
1.52462311557789 0.0475453189557888
1.53668341708543 0.0202032889778795
1.54874371859296 -0.00209098424899307
1.5608040201005 -0.0195771400316618
1.57286432160804 -0.032544923219021
1.58492462311558 -0.0413118230085809
1.59698492462312 -0.0462059543733475
1.60904522613065 -0.0475536584522557
1.62110552763819 -0.045671049320416
1.63316582914573 -0.0408586771824737
1.64522613065327 -0.0333985352335383
1.6572864321608 -0.0235527505120653
1.66934673366834 -0.0115634290159743
1.68140703517588 0.00234675048540267
1.69346733668342 0.0179734921058688
1.70552763819095 0.0351295990490659
1.71758793969849 0.0536435941307043
1.72964824120603 0.073358338275725
1.74170854271357 0.0941297045827736
1.75376884422111 0.115825340404262
1.76582914572864 0.138323532947578
1.77788944723618 0.161512182936473
1.78994974874372 0.185287884140754
1.80201005025126 0.209555102768635
1.81407035175879 0.234225448846351
1.82613065326633 0.259217031081335
1.83819095477387 0.284453886822803
1.85025125628141 0.309865479256349
1.86231155778894 0.335386254670267
1.87437185929648 0.360955253367883
1.88643216080402 0.38651576848721
1.89849246231156 0.41201504758538
1.9105527638191 0.437404032335556
1.92261306532663 0.462637132073363
1.93467336683417 0.487672027233049
1.94673366834171 0.512469498951547
1.95879396984925 0.536993281313269
1.97085427135678 0.561209932880848
1.98291457286432 0.585088724324853
1.99497487437186 0.608601539142692
2.0070351758794 0.631722784652697
2.01909547738693 0.654429310669023
2.03115577889447 0.676700333507167
2.04321608040201 0.698517363236846
2.05527638190955 0.719864132383363
2.06733668341709 0.740726524574086
2.07939698492462 0.761092501925605
2.09145728643216 0.780952030261601
2.1035175879397 0.800297001533838
2.11557788944724 0.819121153082388
2.12763819095477 0.837419983610359
2.13969849246231 0.855190665958855
2.15175879396985 0.872431956946856
2.16381909547739 0.889144104686578
2.17587939698492 0.905328753897681
2.18793969849246 0.920988849824405
2.2 0.936128541410374
};
\addlegendentry{Neural net output}
\end{axis}

\end{tikzpicture}
%
  \end{minipage}
  \hfill  
  \begin{minipage}[c]{.03\textwidth}
    \centering
    \tikz[overlay,remember picture]\node(p0){};
  \end{minipage}  
  \hfill
  \begin{minipage}[c]{.45\textwidth}
    \raggedleft
    \pgfplotsset{yticklabels={,,},ytick={\empty}}
    % This file was created with tikzplotlib v0.10.1.
\begin{tikzpicture}

\definecolor{darkgray176}{RGB}{176,176,176}
\definecolor{lightgray204}{RGB}{204,204,204}
\definecolor{steelblue31119180}{RGB}{31,119,180}

\begin{axis}[
height=\figureheight,
legend cell align={left},
legend style={fill opacity=0.8, draw opacity=1, text opacity=1, draw=lightgray204},
tick align=outside,
tick pos=left,
width=\figurewidth,
x grid style={darkgray176},
xmin=-0.2, xmax=2.2,
xtick style={color=black},
y grid style={darkgray176},
ymin=-5.2, ymax=7,
ytick style={color=black}
]
\addplot [draw=black, fill=black, forget plot, mark=+, only marks, opacity=0.2]
table{%
x  y
0.59907633700711 0.223154562086841
0.34361536741137 1.80796222965741
1.21549153760962 0.967955764303031
1.39127490332365 0.346027494892729
0.527962441193321 2.55278258834587
0.0333065151243783 5.67346791281363
1.41384715629159 0.881468575721158
0.663171303916053 -0.472049787519171
1.00205910730797 -1.96751300267146
0.420018112144075 1.21728173287471
0.631458340576862 1.08436032794776
0.729874319093148 -0.101521243817772
0.147377188445248 5.13582753705874
0.403831786198692 2.27647279843891
0.478013434900342 1.17179814470945
0.595830829634752 2.23305061724589
0.955601779278568 -1.72905380014215
0.0839045702835102 6.3929614602345
1.45472441165539 -0.0760227450398461
0.278479698750005 2.84103829741965
0.380362104273414 2.11325866978181
0.464499503069764 1.2399605663803
0.793107714094262 -1.35173014471992
0.977768289229485 -2.55927294171402
0.569210172057486 0.68494103913613
0.36980976303721 0.921314385700647
0.0214569812169767 5.68468381545469
0.424236769219515 1.63566013099165
1.37938987893611 0.868125317911011
0.137047115400012 4.58721934956631
1.08761751839489 -0.172567506567352
0.419817741770634 2.24046262513834
1.4262841184116 0.159812522509929
1.41564972631546 0.78558633647277
0.495989002819549 1.10158598995442
0.995026761218884 -1.37546551579343
1.12639463876016 -0.334794924441981
1.07766860579985 -0.792376161207397
0.413701619771477 1.4913623618045
0.942986552597874 -2.4722425346432
0.763519126454151 -0.583688218859911
0.130904366804652 4.18156576167409
0.0454384945865041 5.94112904944771
0.522204951265 1.21564815606939
0.332942022904396 2.65994126760993
1.40363369449562 1.23353940603206
0.779310595942776 -1.02035508529735
1.29407529214362 1.06264876539534
0.556033388135621 0.837778673324579
0.484633763104316 2.32967153874662
0.0920321107558388 5.09987558004654
0.56325131198675 1.98910435881925
1.45678884132294 0.154899001388011
0.408702868409535 1.25289558714472
0.458097584655798 1.52690656054213
1.46582759713057 0.56502069721856
0.507184083096661 1.52564677260393
1.36046218960706 0.507926246238466
0.921764573531882 -2.09142805291535
0.238639274847773 2.6411155956647
0.362559658595395 1.79727559711587
1.21589080021465 0.400000674361935
0.933585205353682 -1.55461336837348
0.982715141501147 -2.45533624593891
0.401490860442604 1.05987192830147
0.156059199998835 4.36370190980214
0.891378994342085 -2.36391299232751
0.388158878064094 0.788154469107815
0.667188047358641 1.10183787343592
0.625170599258625 0.6586301610886
0.957021857615636 -1.85051954010725
1.02156844318979 -1.83957136491289
1.25724919181647 0.322518415235646
0.00266557165722991 6.50557346836786
1.41816728826336 0.629351904749414
1.31945988198246 0.544524900201801
1.25294153993982 1.58846644417793
1.44809592439576 -0.343645665284646
1.12607739077688 0.0198156968068374
0.251530997239499 3.26092402478648
0.729180476405212 -0.261848785696243
0.902909475880636 -1.88887529622365
0.815297425524211 -0.263213218540796
0.399596820896549 1.8350776647408
1.19233932331138 0.594739208851601
0.212701803984358 3.69662943902046
0.48579647389144 0.781131500953395
0.754043399884807 -0.120330706281183
0.299033951960917 2.59001615301901
0.471011920030608 0.803031316713116
1.23848755238611 1.20117380925507
0.0927035953462867 5.14006372858808
0.803635270801682 -1.64573382353524
0.971532479828419 -1.26141546874724
0.866704965181163 -1.37016791118963
1.36729265920189 0.597849962429656
0.519752323523261 1.89624294678323
0.0191107582149252 5.33379579237503
0.196986770599228 4.03647401377799
0.952624784132827 -2.16148044060619
0.0015044650302265 5.2886136068202
0.902037550168227 -2.52858794998143
1.22749030238651 0.0464677514177437
1.21792105205884 -0.141830633035098
0.0102856385216306 6.14133785166822
0.863182937449097 -2.1011551265248
0.818564275101864 -1.01534185724427
0.302635244152368 1.46210778674698
0.372879666668875 1.40476340640872
0.889792662887503 -1.17727077360789
1.29641079289544 0.586535140325398
1.3272574482414 1.3843767453822
0.214325965580719 3.31884710039265
0.614137818457737 1.59184679724417
0.368321423663743 0.845629772651128
0.380933232697981 0.723731284253583
0.976159176116351 -1.93326574524374
0.857441599514585 -1.60433853663788
0.83708880960736 -2.04818343189899
1.47579759938635 -0.0805765439070799
0.442438891644387 1.73835028751105
1.38953493744193 0.417026784527823
1.14080627628873 0.550112184891199
1.48114359376883 0.298669829133318
0.736597976181201 -0.643077812882977
0.394694513831174 0.936428225891781
1.26908189826216 0.858430359609534
0.0672098890730217 5.09900003215748
0.333851467177716 3.01981294882107
0.945969964783799 -1.83629872793283
0.692843988254517 0.433249498499454
1.26598975868233 1.28216661393988
0.391972523919673 1.09693715415414
0.46917601169147 1.50263243659036
1.01098339289849 -2.31817336136178
0.777738706583271 -0.838717602177543
1.12209237980542 -0.319615415703505
1.17055379105428 0.512465979308082
0.258836084652896 2.78755665557385
0.61396534548555 1.69052927515609
1.31770140735241 1.16502121252192
0.116000397751632 4.99444756152836
0.210700698077648 3.49050340807256
1.05114996874651 -1.76278149811195
0.670454848751649 0.224613670765436
1.92406486757113 -0.108425407068489
1.94209538616793 1.7264746642509
1.99393444419407 0.403999409461401
1.95439877208446 0.827632344838344
1.98473609767015 0.0244378792655042
1.9054919129869 0.855315202370139
1.94542261961925 0.240943330223005
1.96246848086935 0.398027059259599
};
\addplot [draw=black, fill=black, forget plot, mark=|, only marks]
table{%
x  y
0.736597976181201 -5
0.332942022904396 -5
1.21589080021465 -5
0.729874319093148 -5
0.527962441193321 -5
1.02156844318979 -5
1.47579759938635 -5
1.38953493744193 -5
0.955601779278568 -5
1.36046218960706 -5
0.729180476405212 -5
0.212701803984358 -5
0.388158878064094 -5
0.484633763104316 -5
0.380933232697981 -5
0.866704965181163 -5
1.94542261961925 -5
0.130904366804652 -5
1.9054919129869 -5
0.982715141501147 -5
0.362559658595395 -5
1.92406486757113 -5
1.98473609767015 -5
0.48579647389144 -5
0.00266557165722991 -5
1.3272574482414 -5
0.0927035953462867 -5
1.94209538616793 -5
0.495989002819549 -5
0.56325131198675 -5
};
\addplot [semithick, steelblue31119180]
table {%
-0.2 6.36070232858577
-0.187939698492462 6.35472956573446
-0.175879396984925 6.34731334659466
-0.163819095477387 6.3382973339002
-0.151758793969849 6.32750597075953
-0.139698492462312 6.31474262926839
-0.127638190954774 6.29978772331929
-0.115577889447236 6.28239683324762
-0.103517587939698 6.26229890775984
-0.0914572864321608 6.23919463075493
-0.0793969849246231 6.21275506772564
-0.0673366834170854 6.18262073884089
-0.0552763819095477 6.14840130364111
-0.0432160804020101 6.10967608498086
-0.0311557788944724 6.06599570590153
-0.0190954773869347 6.01688515932343
-0.00703517587939698 5.96184867159156
0.00502512562814073 5.9003767487355
0.0170854271356784 5.83195579694493
0.0291457286432161 5.75608067051428
0.0412060301507538 5.67227040191608
0.0532663316582915 5.5800871885529
0.0653266331658292 5.4791584294262
0.0773869346733668 5.36920121117875
0.0894472361809046 5.25004814227613
0.101507537688442 5.12167286069599
0.11356783919598 4.9842129674799
0.125628140703518 4.83798768382257
0.137688442211055 4.68350735022958
0.149748743718593 4.52147215648096
0.161809045226131 4.35275835662945
0.173869346733668 4.17839174290576
0.185929648241206 3.99951023294512
0.197989949748744 3.81731977895058
0.210050251256281 3.63304995308783
0.222110552763819 3.44791688714876
0.234170854271357 3.2631011241545
0.246231155778894 3.07974591479964
0.258291457286432 2.89897745452039
0.27035175879397 2.72194293904684
0.282412060301508 2.54985622775564
0.294472361809045 2.38403613684964
0.306532663316583 2.22592111995052
0.318592964824121 2.0770481567441
0.330653266331658 1.93899336049264
0.342713567839196 1.8132847868457
0.354773869346734 1.7013090276095
0.366834170854271 1.60423608349797
0.378894472361809 1.52297756286392
0.390954773869347 1.45817288272403
0.403015075376884 1.41017517061525
0.415075376884422 1.37899584145775
0.42713567839196 1.36417616903236
0.439195979899497 1.3645893478449
0.451256281407035 1.37822749907785
0.463316582914573 1.40207037804159
0.475376884422111 1.43213649661271
0.487437185929648 1.4637669418139
0.499497487437186 1.49210269101084
0.511557788944724 1.51263277569299
0.523618090452261 1.52166051150807
0.535678391959799 1.5165737009306
0.547738693467337 1.49588586888149
0.559798994974875 1.45909325603014
0.571859296482412 1.40643323587227
0.58391959798995 1.33862886999712
0.595979899497487 1.2566771019735
0.608040201005025 1.16170473720144
0.620100502512563 1.05489024637302
0.632160804020101 0.93743495680274
0.644221105527638 0.810562848055073
0.656281407035176 0.675530347542954
0.668341708542714 0.533632875857196
0.680402010050251 0.386201076189385
0.692462311557789 0.234585255604396
0.704522613065327 0.0801307718497179
0.716582914572864 -0.0758504607712076
0.728643216080402 -0.232106540982688
0.74070351758794 -0.387465332697582
0.752763819095478 -0.540844455742336
0.764824120603015 -0.69125092017881
0.776884422110553 -0.837771945212664
0.788944723618091 -0.979559969048386
0.801005025125628 -1.11581542763761
0.813065326633166 -1.24577057390912
0.825125628140704 -1.36867662871632
0.837185929648241 -1.48379520730234
0.849246231155779 -1.59039357303285
0.861306532663317 -1.68774210323933
0.873366834170854 -1.77511158731768
0.885427135678392 -1.85176769569789
0.89748743718593 -1.91696017190033
0.909547738693467 -1.96990500045275
0.921608040201005 -2.00975902590461
0.933668341708543 -2.03558837292993
0.945728643216081 -2.04633479373056
0.957788944723618 -2.04078805457172
0.969849246231156 -2.01757782337929
0.981909547738694 -1.97520477314883
0.993969849246231 -1.91213594055423
1.00603015075377 -1.82698974108667
1.01809045226131 -1.7188250806116
1.03015075376884 -1.58752014867175
1.04221105527638 -1.43417879120378
1.05427135678392 -1.26144888283686
1.06633165829146 -1.07360863489331
1.078391959799 -0.876309780531172
1.09045226130653 -0.675973094164984
1.10251256281407 -0.47897245200267
1.11457286432161 -0.290839862148597
1.12663316582915 -0.115713508513106
1.13869346733668 0.0438591077268476
1.15075376884422 0.186790405153377
1.16281407035176 0.3131161071239
1.1748743718593 0.423594221051813
1.18693467336683 0.519345197152491
1.19899497487437 0.601584816218358
1.21105527638191 0.671457027710173
1.22311557788945 0.729949240408179
1.23517587939698 0.777864813182883
1.24723618090452 0.815829767935315
1.25929648241206 0.844316963973451
1.2713567839196 0.863677470322242
1.28341708542714 0.874174058355338
1.29547738693467 0.876015231361452
1.30753768844221 0.869390138140973
1.31959798994975 0.85450532406461
1.33165829145729 0.831623761555799
1.34371859296482 0.801105168466419
1.35577889447236 0.763444562655624
1.3678391959799 0.71930383387754
1.37989949748744 0.669529620920853
1.39195979899498 0.615150860233467
1.40402010050251 0.557351671239538
1.41608040201005 0.497419705247187
1.42814070351759 0.436675669338059
1.44020100502513 0.376394629994868
1.45226130653266 0.317732051845667
1.4643216080402 0.261666358806096
1.47638190954774 0.208965544817018
1.48844221105528 0.160179615709723
1.50050251256281 0.115655377138293
1.51256281407035 0.0755667317508104
1.52462311557789 0.0399526613831355
1.53668341708543 0.00875601553523523
1.54874371859296 -0.0181417639555125
1.5608040201005 -0.0408926322031861
1.57286432160804 -0.0596614324750676
1.58492462311558 -0.0746124102932346
1.59698492462312 -0.0859015903797752
1.60904522613065 -0.0936736256940888
1.62110552763819 -0.0980617822863846
1.63316582914573 -0.0991899227284965
1.64522613065327 -0.0971755970033192
1.6572864321608 -0.0921335863725596
1.66934673366834 -0.0841794465301677
1.68140703517588 -0.0734327541397741
1.69346733668342 -0.0600198781738963
1.70552763819095 -0.0440761812736237
1.71758793969849 -0.0257476142938168
1.72964824120603 -0.00519170617905642
1.74170854271357 0.0174220231727022
1.75376884422111 0.0419121831149164
1.76582914572864 0.0680861115785948
1.77788944723618 0.0957408938299712
1.78994974874372 0.124664738070084
1.80201005025126 0.154638644324363
1.81407035175879 0.185438303524533
1.82613065326633 0.216836165124762
1.83819095477387 0.248603613879098
1.85025125628141 0.280513199274388
1.86231155778894 0.312340864605069
1.87437185929648 0.343868126674329
1.88643216080402 0.374884161510673
1.89849246231156 0.405187756398019
1.9105527638191 0.434589093689028
1.92261306532663 0.462911337496224
1.93467336683417 0.489992000067507
1.94673366834171 0.515684070626019
1.95879396984925 0.539856895424896
1.97085427135678 0.56239680368694
1.98291457286432 0.583207479815648
1.99497487437186 0.602210087656844
2.0070351758794 0.619343157586567
2.01909547738693 0.634562251566654
2.03115577889447 0.647839425112722
2.04321608040201 0.659162508156065
2.05527638190955 0.668534229079709
2.06733668341709 0.675971207695109
2.07939698492462 0.681502843714398
2.09145728643216 0.685170127215315
2.1035175879397 0.687024397011847
2.11557788944724 0.687126071528283
2.12763819095477 0.685543375069712
2.13969849246231 0.682351080210398
2.15175879396985 0.677629284609379
2.16381909547739 0.671462237953302
2.17587939698492 0.663937232001426
2.18793969849246 0.655143564077369
2.2 0.645171581654874
};
\addlegendentry{Mean}
\addplot [semithick, red, forget plot]
table {%
-0.2 6.2857973514481
-0.187939698492462 6.28987893254762
-0.175879396984925 6.29181588372522
-0.163819095477387 6.29143317385438
-0.151758793969849 6.28854013004549
-0.139698492462312 6.28292926239899
-0.127638190954774 6.27437510005683
-0.115577889447236 6.26263307579898
-0.103517587939698 6.24743850896844
-0.0914572864321608 6.22850575195502
-0.0793969849246231 6.20552758417457
-0.0673366834170854 6.17817495967054
-0.0552763819095477 6.14609724012402
-0.0432160804020101 6.10892307376503
-0.0311557788944724 6.06626211137595
-0.0190954773869347 6.01770778129061
-0.00703517587939698 5.96284137276835
0.00502512562814073 5.90123769647965
0.0170854271356784 5.83247259523151
0.0291457286432161 5.75613255854249
0.0412060301507538 5.67182664040638
0.0532663316582915 5.5792007786667
0.0653266331658292 5.47795445565633
0.0773869346733668 5.36785941553933
0.0894472361809046 5.24877986427611
0.101507537688442 5.12069323625788
0.11356783919598 4.9837102481237
0.125628140703518 4.83809262672679
0.137688442211055 4.68426666625203
0.149748743718593 4.52283072306446
0.161809045226131 4.35455497673142
0.173869346733668 4.18037232676687
0.185929648241206 4.00136016107162
0.197989949748744 3.81871385699869
0.210050251256281 3.63371411567772
0.222110552763819 3.44769137777358
0.234170854271357 3.26199138801039
0.246231155778894 3.0779462474231
0.258291457286432 2.89685486141262
0.27035175879397 2.71997550441124
0.282412060301508 2.54853134759842
0.294472361809045 2.38372743454599
0.306532663316583 2.2267750630501
0.318592964824121 2.07891724952624
0.330653266331658 1.94144733602477
0.342713567839196 1.81571216298775
0.354773869346734 1.70309166873519
0.366834170854271 1.60494816112118
0.378894472361809 1.52254067772778
0.390954773869347 1.45690296965451
0.403015075376884 1.40868845690316
0.415075376884422 1.37799311124833
0.42713567839196 1.36417806680222
0.439195979899497 1.36572605810096
0.451256281407035 1.38017446012216
0.463316582914573 1.40416536568561
0.475376884422111 1.43363406884371
0.487437185929648 1.46412297935564
0.499497487437186 1.49116970386928
0.511557788944724 1.51069284803595
0.523618090452261 1.51929979398149
0.535678391959799 1.51446714403481
0.547738693467337 1.49458367539806
0.559798994974875 1.45888030004499
0.571859296482412 1.40729020308166
0.58391959798995 1.3402834770427
0.595979899497487 1.2587099228498
0.608040201005025 1.16366898121746
0.620100502512563 1.05641276278727
0.632160804020101 0.938279469283438
0.644221105527638 0.81065027839526
0.656281407035176 0.674921907157222
0.668341708542714 0.53248820908157
0.680402010050251 0.384726155567839
0.692462311557789 0.232983617124391
0.704522613065327 0.0785680401076337
0.716582914572864 -0.0772637885233033
0.728643216080402 -0.233314196261137
0.74070351758794 -0.388450594830113
0.752763819095478 -0.541608554626875
0.764824120603015 -0.691791981866757
0.776884422110553 -0.838070751640898
0.788944723618091 -0.979576344982615
0.801005025125628 -1.11549603458366
0.813065326633166 -1.24506598531361
0.825125628140704 -1.36756334671807
0.837185929648241 -1.48229709669328
0.849246231155779 -1.58859712629165
0.861306532663317 -1.68580089703069
0.873366834170854 -1.773236998834
0.885427135678392 -1.85020512459456
0.89748743718593 -1.915952398347
0.909547738693467 -1.96964671375817
0.921608040201005 -2.01034885974521
0.933668341708543 -2.03698686496384
0.945728643216081 -2.0483383129386
0.957788944723618 -2.04302939259167
0.969849246231156 -2.01956288769164
0.981909547738694 -1.97639030730567
0.993969849246231 -1.91204409703886
1.00603015075377 -1.82534140335192
1.01809045226131 -1.7156576560425
1.03015075376884 -1.5832440846665
1.04221105527638 -1.42953085004441
1.05427135678392 -1.25732787946063
1.06633165829146 -1.07082794450782
1.078391959799 -0.875349345818019
1.09045226130653 -0.676830101549934
1.10251256281407 -0.481174989533318
1.11457286432161 -0.293615776282351
1.12663316582915 -0.118239365629348
1.13869346733668 0.0422281830821599
1.15075376884422 0.186383598074527
1.16281407035176 0.313924893349873
1.1748743718593 0.425342584611239
1.18693467336683 0.521607339752872
1.19899497487437 0.603906758694623
1.21105527638191 0.673452590391777
1.22311557788945 0.731357540851534
1.23517587939698 0.778569491602677
1.24723618090452 0.815847807322585
1.25929648241206 0.843767921655468
1.2713567839196 0.862743743355362
1.28341708542714 0.873060917720606
1.29547738693467 0.874916788169473
1.30753768844221 0.868464737975285
1.31959798994975 0.853861424167359
1.33165829145729 0.831315338883324
1.34371859296482 0.801134330340053
1.35577889447236 0.763768482133921
1.3678391959799 0.719843528247159
1.37989949748744 0.670179340507215
1.39195979899498 0.615788538744166
1.40402010050251 0.557852295893885
1.41608040201005 0.497673827854153
1.42814070351759 0.436614167370649
1.44020100502513 0.376018470128547
1.45226130653266 0.317143106662216
1.4643216080402 0.261093458268937
1.47638190954774 0.208779774388998
1.48844221105528 0.160894532365159
1.50050251256281 0.11791068459771
1.51256281407035 0.0800970364996313
1.52462311557789 0.0475453189557888
1.53668341708543 0.0202032889778795
1.54874371859296 -0.00209098424899307
1.5608040201005 -0.0195771400316618
1.57286432160804 -0.032544923219021
1.58492462311558 -0.0413118230085809
1.59698492462312 -0.0462059543733475
1.60904522613065 -0.0475536584522557
1.62110552763819 -0.045671049320416
1.63316582914573 -0.0408586771824737
1.64522613065327 -0.0333985352335383
1.6572864321608 -0.0235527505120653
1.66934673366834 -0.0115634290159743
1.68140703517588 0.00234675048540267
1.69346733668342 0.0179734921058688
1.70552763819095 0.0351295990490659
1.71758793969849 0.0536435941307043
1.72964824120603 0.073358338275725
1.74170854271357 0.0941297045827736
1.75376884422111 0.115825340404262
1.76582914572864 0.138323532947578
1.77788944723618 0.161512182936473
1.78994974874372 0.185287884140754
1.80201005025126 0.209555102768635
1.81407035175879 0.234225448846351
1.82613065326633 0.259217031081335
1.83819095477387 0.284453886822803
1.85025125628141 0.309865479256349
1.86231155778894 0.335386254670267
1.87437185929648 0.360955253367883
1.88643216080402 0.38651576848721
1.89849246231156 0.41201504758538
1.9105527638191 0.437404032335556
1.92261306532663 0.462637132073363
1.93467336683417 0.487672027233049
1.94673366834171 0.512469498951547
1.95879396984925 0.536993281313269
1.97085427135678 0.561209932880848
1.98291457286432 0.585088724324853
1.99497487437186 0.608601539142692
2.0070351758794 0.631722784652697
2.01909547738693 0.654429310669023
2.03115577889447 0.676700333507167
2.04321608040201 0.698517363236846
2.05527638190955 0.719864132383363
2.06733668341709 0.740726524574086
2.07939698492462 0.761092501925605
2.09145728643216 0.780952030261601
2.1035175879397 0.800297001533838
2.11557788944724 0.819121153082388
2.12763819095477 0.837419983610359
2.13969849246231 0.855190665958855
2.15175879396985 0.872431956946856
2.16381909547739 0.889144104686578
2.17587939698492 0.905328753897681
2.18793969849246 0.920988849824405
2.2 0.936128541410374
};
\path [draw=steelblue31119180, fill=steelblue31119180, opacity=0.2]
(axis cs:-0.2,13.0843990932545)
--(axis cs:-0.2,-0.362994436082984)
--(axis cs:-0.187939698492462,0.188777675787692)
--(axis cs:-0.175879396984925,0.7194155291641)
--(axis cs:-0.163819095477387,1.2276342734235)
--(axis cs:-0.151758793969849,1.71201960231573)
--(axis cs:-0.139698492462312,2.17100692931417)
--(axis cs:-0.127638190954774,2.60285535575632)
--(axis cs:-0.115577889447236,3.00561466919068)
--(axis cs:-0.103517587939698,3.37708395616428)
--(axis cs:-0.0914572864321608,3.71476279813318)
--(axis cs:-0.0793969849246231,4.01580373099019)
--(axis cs:-0.0673366834170854,4.27699439199833)
--(axis cs:-0.0552763819095477,4.49483990256926)
--(axis cs:-0.0432160804020101,4.66588458829617)
--(axis cs:-0.0311557788944724,4.78746412581474)
--(axis cs:-0.0190954773869347,4.85895610538454)
--(axis cs:-0.00703517587939698,4.88310428193668)
--(axis cs:0.00502512562814073,4.8664097441523)
--(axis cs:0.0170854271356784,4.81788700769972)
--(axis cs:0.0291457286432161,4.7468098657438)
--(axis cs:0.0412060301507538,4.66080342538419)
--(axis cs:0.0532663316582915,4.56498670637322)
--(axis cs:0.0653266331658292,4.46199078967208)
--(axis cs:0.0773869346733668,4.3524267837465)
--(axis cs:0.0894472361809046,4.23553143267436)
--(axis cs:0.101507537688442,4.1098857993086)
--(axis cs:0.11356783919598,3.97415039643458)
--(axis cs:0.125628140703518,3.82771235854031)
--(axis cs:0.137688442211055,3.671078293956)
--(axis cs:0.149748743718593,3.50586273647138)
--(axis cs:0.161809045226131,3.33435345129799)
--(axis cs:0.173869346733668,3.15881865877757)
--(axis cs:0.185929648241206,2.98084968608069)
--(axis cs:0.197989949748744,2.80104290421962)
--(axis cs:0.210050251256281,2.61920934859294)
--(axis cs:0.222110552763819,2.43506718115051)
--(axis cs:0.234170854271357,2.24907625399402)
--(axis cs:0.246231155778894,2.06289705387989)
--(axis cs:0.258291457286432,1.87912249005033)
--(axis cs:0.27035175879397,1.70041682397666)
--(axis cs:0.282412060301508,1.52864242902396)
--(axis cs:0.294472361809045,1.36460492559962)
--(axis cs:0.306532663316583,1.20863893953909)
--(axis cs:0.318592964824121,1.06160751311922)
--(axis cs:0.330653266331658,0.925450349878024)
--(axis cs:0.342713567839196,0.80268485842568)
--(axis cs:0.354773869346734,0.695178970451303)
--(axis cs:0.366834170854271,0.60325203182844)
--(axis cs:0.378894472361809,0.525959291567295)
--(axis cs:0.390954773869347,0.462381152356499)
--(axis cs:0.403015075376884,0.412768584189711)
--(axis cs:0.415075376884422,0.378486097128876)
--(axis cs:0.42713567839196,0.360830583437286)
--(axis cs:0.439195979899497,0.359736475942152)
--(axis cs:0.451256281407035,0.373203945966345)
--(axis cs:0.463316582914573,0.397517886434831)
--(axis cs:0.475376884422111,0.427870348352162)
--(axis cs:0.487437185929648,0.459047567731773)
--(axis cs:0.499497487437186,0.48603415600171)
--(axis cs:0.511557788944724,0.504486543844146)
--(axis cs:0.523618090452261,0.511068716461791)
--(axis cs:0.535678391959799,0.503641981350797)
--(axis cs:0.547738693467337,0.481241611085829)
--(axis cs:0.559798994974875,0.443770490507571)
--(axis cs:0.571859296482412,0.391493149893171)
--(axis cs:0.58391959798995,0.32459934809828)
--(axis cs:0.595979899497487,0.243111228878294)
--(axis cs:0.608040201005025,0.14718878060984)
--(axis cs:0.620100502512563,0.0376058282407554)
--(axis cs:0.632160804020101,-0.0839623454128017)
--(axis cs:0.644221105527638,-0.21507465552659)
--(axis cs:0.656281407035176,-0.35303534440161)
--(axis cs:0.668341708542714,-0.495508761111208)
--(axis cs:0.680402010050251,-0.640937449897437)
--(axis cs:0.692462311557789,-0.788597038232085)
--(axis cs:0.704522613065327,-0.938306857002468)
--(axis cs:0.716582914572864,-1.08996506541067)
--(axis cs:0.728643216080402,-1.24313912800982)
--(axis cs:0.74070351758794,-1.3968899436915)
--(axis cs:0.752763819095478,-1.54986528709949)
--(axis cs:0.764824120603015,-1.7005535909524)
--(axis cs:0.776884422110553,-1.8475342532676)
--(axis cs:0.788944723618091,-1.98961250314218)
--(axis cs:0.801005025125628,-2.12582470843833)
--(axis cs:0.813065326633166,-2.25537067852785)
--(axis cs:0.825125628140704,-2.37754111489426)
--(axis cs:0.837185929648241,-2.4916758071725)
--(axis cs:0.849246231155779,-2.59714793458498)
--(axis cs:0.861306532663317,-2.69335068206714)
--(axis cs:0.873366834170854,-2.7796694996495)
--(axis cs:0.885427135678392,-2.8554420757846)
--(axis cs:0.89748743718593,-2.91992041879412)
--(axis cs:0.909547738693467,-2.97224884082878)
--(axis cs:0.921608040201005,-3.0114655804248)
--(axis cs:0.933668341708543,-3.03653463560994)
--(axis cs:0.945728643216081,-3.04641607355877)
--(axis cs:0.957788944723618,-3.04016847572654)
--(axis cs:0.969849246231156,-3.01702802622901)
--(axis cs:0.981909547738694,-2.97634859958705)
--(axis cs:0.993969849246231,-2.91731580428514)
--(axis cs:1.00603015075377,-2.83858833729743)
--(axis cs:1.01809045226131,-2.73840294100381)
--(axis cs:1.03015075376884,-2.61574432937723)
--(axis cs:1.04221105527638,-2.47238467795966)
--(axis cs:1.05427135678392,-2.3141633175627)
--(axis cs:1.06633165829146,-2.14933052630616)
--(axis cs:1.078391959799,-1.98399586159427)
--(axis cs:1.09045226130653,-1.81855293155342)
--(axis cs:1.10251256281407,-1.64895826700318)
--(axis cs:1.11457286432161,-1.47165497624636)
--(axis cs:1.12663316582915,-1.28765552788404)
--(axis cs:1.13869346733668,-1.10311864051217)
--(axis cs:1.15075376884422,-0.926902263048286)
--(axis cs:1.16281407035176,-0.767121759234727)
--(axis cs:1.1748743718593,-0.628670149281965)
--(axis cs:1.18693467336683,-0.512655157906021)
--(axis cs:1.19899497487437,-0.417436024269111)
--(axis cs:1.21105527638191,-0.340188758569525)
--(axis cs:1.22311557788945,-0.278095071365007)
--(axis cs:1.23517587939698,-0.228892834674805)
--(axis cs:1.24723618090452,-0.190974025115765)
--(axis cs:1.25929648241206,-0.163283030927265)
--(axis cs:1.2713567839196,-0.145152141021098)
--(axis cs:1.28341708542714,-0.136115250599168)
--(axis cs:1.29547738693467,-0.1357276877719)
--(axis cs:1.30753768844221,-0.143450323358098)
--(axis cs:1.31959798994975,-0.158662526014355)
--(axis cs:1.33165829145729,-0.180812232193409)
--(axis cs:1.34371859296482,-0.209610890811642)
--(axis cs:1.35577889447236,-0.245104189481458)
--(axis cs:1.3678391959799,-0.287473674701745)
--(axis cs:1.37989949748744,-0.336582207560976)
--(axis cs:1.39195979899498,-0.391500985605476)
--(axis cs:1.40402010050251,-0.450376591406479)
--(axis cs:1.41608040201005,-0.510865342404242)
--(axis cs:1.42814070351759,-0.571042746727035)
--(axis cs:1.44020100502513,-0.630438169943246)
--(axis cs:1.45226130653266,-0.690811019403245)
--(axis cs:1.4643216080402,-0.756392696372101)
--(axis cs:1.47638190954774,-0.833405708266279)
--(axis cs:1.48844221105528,-0.928761377523517)
--(axis cs:1.50050251256281,-1.04815913866131)
--(axis cs:1.51256281407035,-1.19433841953466)
--(axis cs:1.52462311557789,-1.36634152527687)
--(axis cs:1.53668341708543,-1.55996739999482)
--(axis cs:1.54874371859296,-1.76886173103594)
--(axis cs:1.5608040201005,-1.9856200455246)
--(axis cs:1.57286432160804,-2.20262655319008)
--(axis cs:1.58492462311558,-2.41261923849622)
--(axis cs:1.59698492462312,-2.60905084629251)
--(axis cs:1.60904522613065,-2.78630568166573)
--(axis cs:1.62110552763819,-2.93981052972552)
--(axis cs:1.63316582914573,-3.06606566997947)
--(axis cs:1.64522613065327,-3.16261698606273)
--(axis cs:1.6572864321608,-3.2279877852688)
--(axis cs:1.66934673366834,-3.26158660340689)
--(axis cs:1.68140703517588,-3.26360434926382)
--(axis cs:1.69346733668342,-3.23491086416673)
--(axis cs:1.70552763819095,-3.176957754025)
--(axis cs:1.71758793969849,-3.09169152453631)
--(axis cs:1.72964824120603,-2.98147879605095)
--(axis cs:1.74170854271357,-2.84904373043813)
--(axis cs:1.75376884422111,-2.69741670232317)
--(axis cs:1.76582914572864,-2.52989255372082)
--(axis cs:1.77788944723618,-2.34999627066676)
--(axis cs:1.78994974874372,-2.16145332269919)
--(axis cs:1.80201005025126,-1.96816078725203)
--(axis cs:1.81407035175879,-1.77415315117961)
--(axis cs:1.82613065326633,-1.58355261510032)
--(axis cs:1.83819095477387,-1.40048733402221)
--(axis cs:1.85025125628141,-1.22895330539789)
--(axis cs:1.86231155778894,-1.07259175423217)
--(axis cs:1.87437185929648,-0.934366581716355)
--(axis cs:1.88643216080402,-0.81617358523308)
--(axis cs:1.89849246231156,-0.718497425061827)
--(axis cs:1.9105527638191,-0.640306359796942)
--(axis cs:1.92261306532663,-0.579347116333981)
--(axis cs:1.93467336683417,-0.532838809105391)
--(axis cs:1.94673366834171,-0.498376907135968)
--(axis cs:1.95879396984925,-0.474794625999278)
--(axis cs:1.97085427135678,-0.46279222202382)
--(axis cs:1.98291457286432,-0.46521334942141)
--(axis cs:1.99497487437186,-0.48685520609221)
--(axis cs:2.0070351758794,-0.533725669174285)
--(axis cs:2.01909547738693,-0.611847284809547)
--(axis cs:2.03115577889447,-0.726021494149759)
--(axis cs:2.04321608040201,-0.879083354665598)
--(axis cs:2.05527638190955,-1.07187075418116)
--(axis cs:2.06733668341709,-1.30368951641577)
--(axis cs:2.07939698492462,-1.57289794353505)
--(axis cs:2.09145728643216,-1.87737059448627)
--(axis cs:2.1035175879397,-2.21478533498366)
--(axis cs:2.11557788944724,-2.58277250735955)
--(axis cs:2.12763819095477,-2.97898159772531)
--(axis cs:2.13969849246231,-3.40110650153895)
--(axis cs:2.15175879396985,-3.84689322387803)
--(axis cs:2.16381909547739,-4.31414185542928)
--(axis cs:2.17587939698492,-4.80070790131854)
--(axis cs:2.18793969849246,-5.30450468250762)
--(axis cs:2.2,-5.82350703229574)
--(axis cs:2.2,7.11385019560548)
--(axis cs:2.2,7.11385019560548)
--(axis cs:2.18793969849246,6.61479181066236)
--(axis cs:2.17587939698492,6.1285823653214)
--(axis cs:2.16381909547739,5.65706633133588)
--(axis cs:2.15175879396985,5.20215179309679)
--(axis cs:2.13969849246231,4.76580866195974)
--(axis cs:2.12763819095477,4.35006834786473)
--(axis cs:2.11557788944724,3.95702465041612)
--(axis cs:2.1035175879397,3.58883412900736)
--(axis cs:2.09145728643216,3.2477108489169)
--(axis cs:2.07939698492462,2.93590363096385)
--(axis cs:2.06733668341709,2.65563193180599)
--(axis cs:2.05527638190955,2.40893921234058)
--(axis cs:2.04321608040201,2.19740837097773)
--(axis cs:2.03115577889447,2.0217003443752)
--(axis cs:2.01909547738693,1.88097178794285)
--(axis cs:2.0070351758794,1.77241198434742)
--(axis cs:1.99497487437186,1.6912753814059)
--(axis cs:1.98291457286432,1.63162830905271)
--(axis cs:1.97085427135678,1.5875858293977)
--(axis cs:1.95879396984925,1.55450841684907)
--(axis cs:1.94673366834171,1.52974504838801)
--(axis cs:1.93467336683417,1.5128228092404)
--(axis cs:1.92261306532663,1.50516979132643)
--(axis cs:1.9105527638191,1.509484547175)
--(axis cs:1.89849246231156,1.52887293785787)
--(axis cs:1.88643216080402,1.56594190825443)
--(axis cs:1.87437185929648,1.62210283506501)
--(axis cs:1.86231155778894,1.69727348344231)
--(axis cs:1.85025125628141,1.78997970394667)
--(axis cs:1.83819095477387,1.8976945617804)
--(axis cs:1.82613065326633,2.01722494534984)
--(axis cs:1.81407035175879,2.14502975822867)
--(axis cs:1.80201005025126,2.27743807590076)
--(axis cs:1.78994974874372,2.41078279883936)
--(axis cs:1.77788944723618,2.5414780583267)
--(axis cs:1.76582914572864,2.66606477687801)
--(axis cs:1.75376884422111,2.781241068553)
--(axis cs:1.74170854271357,2.88388777678353)
--(axis cs:1.72964824120603,2.97109538369283)
--(axis cs:1.71758793969849,3.04019629594868)
--(axis cs:1.70552763819095,3.08880539147775)
--(axis cs:1.69346733668342,3.11487110781893)
--(axis cs:1.68140703517588,3.11673884098427)
--(axis cs:1.66934673366834,3.09322771034655)
--(axis cs:1.6572864321608,3.04372061252368)
--(axis cs:1.64522613065327,2.96826579205609)
--(axis cs:1.63316582914573,2.86768582452247)
--(axis cs:1.62110552763819,2.74368696515275)
--(axis cs:1.60904522613065,2.59895843027755)
--(axis cs:1.59698492462312,2.43724766553296)
--(axis cs:1.58492462311558,2.26339441790975)
--(axis cs:1.57286432160804,2.08330368823994)
--(axis cs:1.5608040201005,1.90383478111823)
--(axis cs:1.54874371859296,1.73257820312492)
--(axis cs:1.53668341708543,1.57747943106529)
--(axis cs:1.52462311557789,1.44624684804314)
--(axis cs:1.51256281407035,1.34547188303628)
--(axis cs:1.50050251256281,1.27946989293789)
--(axis cs:1.48844221105528,1.24912060894296)
--(axis cs:1.47638190954774,1.25133679790032)
--(axis cs:1.4643216080402,1.27972541398429)
--(axis cs:1.45226130653266,1.32627512309458)
--(axis cs:1.44020100502513,1.38322742993298)
--(axis cs:1.42814070351759,1.44439408540315)
--(axis cs:1.41608040201005,1.50570475289862)
--(axis cs:1.40402010050251,1.56507993388555)
--(axis cs:1.39195979899498,1.62180270607241)
--(axis cs:1.37989949748744,1.67564144940268)
--(axis cs:1.3678391959799,1.72608134245683)
--(axis cs:1.35577889447236,1.7719933147927)
--(axis cs:1.34371859296482,1.81182122774448)
--(axis cs:1.33165829145729,1.84405975530501)
--(axis cs:1.31959798994975,1.86767317414357)
--(axis cs:1.30753768844221,1.88223059964004)
--(axis cs:1.29547738693467,1.8877581504948)
--(axis cs:1.28341708542714,1.88446336730985)
--(axis cs:1.2713567839196,1.87250708166558)
--(axis cs:1.25929648241206,1.85191695887417)
--(axis cs:1.24723618090452,1.8226335609864)
--(axis cs:1.23517587939698,1.78462246104057)
--(axis cs:1.22311557788945,1.73799355218136)
--(axis cs:1.21105527638191,1.68310281398987)
--(axis cs:1.19899497487437,1.62060565670583)
--(axis cs:1.18693467336683,1.551345552211)
--(axis cs:1.1748743718593,1.47585859138559)
--(axis cs:1.16281407035176,1.39335397348253)
--(axis cs:1.15075376884422,1.30048307335504)
--(axis cs:1.13869346733668,1.19083685596586)
--(axis cs:1.12663316582915,1.05622851085783)
--(axis cs:1.11457286432161,0.889975251949162)
--(axis cs:1.10251256281407,0.69101336299784)
--(axis cs:1.09045226130653,0.466606743223449)
--(axis cs:1.078391959799,0.231376300531928)
--(axis cs:1.06633165829146,0.00211325651953986)
--(axis cs:1.05427135678392,-0.208734448111016)
--(axis cs:1.04221105527638,-0.395972904447898)
--(axis cs:1.03015075376884,-0.559295967966264)
--(axis cs:1.01809045226131,-0.699247220219387)
--(axis cs:1.00603015075377,-0.815391144875914)
--(axis cs:0.993969849246231,-0.906956076823322)
--(axis cs:0.981909547738694,-0.974060946710623)
--(axis cs:0.969849246231156,-1.01812762052957)
--(axis cs:0.957788944723618,-1.0414076334169)
--(axis cs:0.945728643216081,-1.04625351390235)
--(axis cs:0.933668341708543,-1.03464211024993)
--(axis cs:0.921608040201005,-1.00805247138442)
--(axis cs:0.909547738693467,-0.967561160076724)
--(axis cs:0.89748743718593,-0.913999925006536)
--(axis cs:0.885427135678392,-0.848093315611168)
--(axis cs:0.873366834170854,-0.770553674985847)
--(axis cs:0.861306532663317,-0.682133524411515)
--(axis cs:0.849246231155779,-0.583639211480716)
--(axis cs:0.837185929648241,-0.475914607432175)
--(axis cs:0.825125628140704,-0.359812142538378)
--(axis cs:0.813065326633166,-0.236170469290393)
--(axis cs:0.801005025125628,-0.10580614683688)
--(axis cs:0.788944723618091,0.0304925650454052)
--(axis cs:0.776884422110553,0.171990362842277)
--(axis cs:0.764824120603015,0.318051750594784)
--(axis cs:0.752763819095478,0.46817637561482)
--(axis cs:0.74070351758794,0.621959278296333)
--(axis cs:0.728643216080402,0.778926046044447)
--(axis cs:0.716582914572864,0.938264143868253)
--(axis cs:0.704522613065327,1.0985684007019)
--(axis cs:0.692462311557789,1.25776754944088)
--(axis cs:0.680402010050251,1.41333960227621)
--(axis cs:0.668341708542714,1.5627745128256)
--(axis cs:0.656281407035176,1.70409603948752)
--(axis cs:0.644221105527638,1.83620035163674)
--(axis cs:0.632160804020101,1.95883225901828)
--(axis cs:0.620100502512563,2.07217466450529)
--(axis cs:0.608040201005025,2.17622069379305)
--(axis cs:0.595979899497487,2.2702429750687)
--(axis cs:0.58391959798995,2.35265839189596)
--(axis cs:0.571859296482412,2.42137332185138)
--(axis cs:0.559798994974875,2.4744160215527)
--(axis cs:0.547738693467337,2.51053012667715)
--(axis cs:0.535678391959799,2.52950542051041)
--(axis cs:0.523618090452261,2.53225230655434)
--(axis cs:0.511557788944724,2.52077900754183)
--(axis cs:0.499497487437186,2.49817122601996)
--(axis cs:0.487437185929648,2.46848631589603)
--(axis cs:0.475376884422111,2.43640264487327)
--(axis cs:0.463316582914573,2.40662286964836)
--(axis cs:0.451256281407035,2.38325105218935)
--(axis cs:0.439195979899497,2.36944221974764)
--(axis cs:0.42713567839196,2.36752175462743)
--(axis cs:0.415075376884422,2.37950558578662)
--(axis cs:0.403015075376884,2.40758175704079)
--(axis cs:0.390954773869347,2.45396461309156)
--(axis cs:0.378894472361809,2.51999583416055)
--(axis cs:0.366834170854271,2.6052201351675)
--(axis cs:0.354773869346734,2.7074390847677)
--(axis cs:0.342713567839196,2.82388471526573)
--(axis cs:0.330653266331658,2.95253637110725)
--(axis cs:0.318592964824121,3.09248880036899)
--(axis cs:0.306532663316583,3.24320330036195)
--(axis cs:0.294472361809045,3.40346734809967)
--(axis cs:0.282412060301508,3.57107002648732)
--(axis cs:0.27035175879397,3.74346905411702)
--(axis cs:0.258291457286432,3.91883241899046)
--(axis cs:0.246231155778894,4.09659477571938)
--(axis cs:0.234170854271357,4.27712599431498)
--(axis cs:0.222110552763819,4.46076659314701)
--(axis cs:0.210050251256281,4.64689055758272)
--(axis cs:0.197989949748744,4.83359665368154)
--(axis cs:0.185929648241206,5.01817077980956)
--(axis cs:0.173869346733668,5.19796482703395)
--(axis cs:0.161809045226131,5.37116326196092)
--(axis cs:0.149748743718593,5.53708157649053)
--(axis cs:0.137688442211055,5.69593640650316)
--(axis cs:0.125628140703518,5.84826300910482)
--(axis cs:0.11356783919598,5.99427553852521)
--(axis cs:0.101507537688442,6.13345992208339)
--(axis cs:0.0894472361809046,6.26456485187791)
--(axis cs:0.0773869346733668,6.38597563861099)
--(axis cs:0.0653266331658292,6.49632606918033)
--(axis cs:0.0532663316582915,6.59518767073258)
--(axis cs:0.0412060301507538,6.68373737844797)
--(axis cs:0.0291457286432161,6.76535147528475)
--(axis cs:0.0170854271356784,6.84602458619014)
--(axis cs:0.00502512562814073,6.9343437533187)
--(axis cs:-0.00703517587939698,7.04059306124644)
--(axis cs:-0.0190954773869347,7.17481421326233)
--(axis cs:-0.0311557788944724,7.34452728598831)
--(axis cs:-0.0432160804020101,7.55346758166554)
--(axis cs:-0.0552763819095477,7.80196270471297)
--(axis cs:-0.0673366834170854,8.08824708568345)
--(axis cs:-0.0793969849246231,8.40970640446108)
--(axis cs:-0.0914572864321608,8.76362646337667)
--(axis cs:-0.103517587939698,9.1475138593554)
--(axis cs:-0.115577889447236,9.55917899730456)
--(axis cs:-0.127638190954774,9.99672009088227)
--(axis cs:-0.139698492462312,10.4584783292226)
--(axis cs:-0.151758793969849,10.9429923392033)
--(axis cs:-0.163819095477387,11.4489603943769)
--(axis cs:-0.175879396984925,11.9752111640252)
--(axis cs:-0.187939698492462,12.5206814556812)
--(axis cs:-0.2,13.0843990932545)
--cycle;
\addlegendimage{area legend, draw=steelblue31119180, fill=steelblue31119180, opacity=0.2}
\addlegendentry{95\% interval}

\draw (axis cs:0,-4.3) node[
  scale=0.5,
  anchor=base west,
  text=black,
  rotate=0.0
]{\inducing};
\end{axis}

\end{tikzpicture}
%
  \end{minipage}
%  \hfill  
%  \begin{subfigure}[c]{.01\textwidth}
%    \centering
%    \tikz[overlay,remember picture]\node(p1){};
%  \end{subfigure}  
%  \hfill
%  \begin{subfigure}[c]{.28\textwidth}
%    \raggedleft
%    \pgfplotsset{yticklabels={,,},ytick={\empty}}        
%    % This file was created with tikzplotlib v0.10.1.
\begin{tikzpicture}

\definecolor{darkgray176}{RGB}{176,176,176}
\definecolor{lightgray204}{RGB}{204,204,204}
\definecolor{steelblue31119180}{RGB}{31,119,180}

\begin{axis}[
height=\figureheight,
legend cell align={left},
legend style={fill opacity=0.8, draw opacity=1, text opacity=1, draw=lightgray204},
tick align=outside,
tick pos=left,
width=\figurewidth,
x grid style={darkgray176},
xmin=-0.2, xmax=2.2,
xtick style={color=black},
y grid style={darkgray176},
ymin=-5.2, ymax=7,
ytick style={color=black}
]
\addplot [draw=black, fill=black, mark=+, only marks, opacity=0.5]
table{%
x  y
1.5 -2.33838873279464
1.55555555555556 -2.60603878590284
1.61111111111111 -1.99062318144487
1.66666666666667 -2.59534399125436
1.72222222222222 -1.01675610881956
1.77777777777778 -1.37530788722259
1.83333333333333 -0.709054708360029
1.88888888888889 -1.47906567568569
1.94444444444444 -2.10937865629948
2 -1.72924253472454
};
\addlegendentry{Data}
\addplot [draw=black, fill=black, forget plot, mark=|, only marks]
table{%
x  y
0.736597976181201 -5
0.332942022904396 -5
1.21589080021465 -5
0.729874319093148 -5
0.527962441193321 -5
1.02156844318979 -5
1.47579759938635 -5
1.38953493744193 -5
0.955601779278568 -5
1.36046218960706 -5
0.729180476405212 -5
0.212701803984358 -5
0.388158878064094 -5
0.484633763104316 -5
0.380933232697981 -5
0.866704965181163 -5
1.94542261961925 -5
0.130904366804652 -5
1.9054919129869 -5
0.982715141501147 -5
0.362559658595395 -5
1.92406486757113 -5
1.98473609767015 -5
0.48579647389144 -5
0.00266557165722991 -5
1.3272574482414 -5
0.0927035953462867 -5
1.94209538616793 -5
0.495989002819549 -5
0.56325131198675 -5
};
\addplot [semithick, steelblue31119180]
table {%
-0.2 7.2203323491566
-0.187939698492462 7.17302537698398
-0.175879396984925 7.12332387689887
-0.163819095477387 7.07113331490043
-0.151758793969849 7.01636012740997
-0.139698492462312 6.95891217910755
-0.127638190954774 6.89869916710212
-0.115577889447236 6.83563294024093
-0.103517587939698 6.76962769687773
-0.0914572864321608 6.70060002188333
-0.0793969849246231 6.62846872035078
-0.0673366834170854 6.5531544049917
-0.0552763819095477 6.47457879846279
-0.0432160804020101 6.39266371608436
-0.0311557788944724 6.30732971001411
-0.0190954773869347 6.21849437231609
-0.00703517587939698 6.12607032572215
0.00502512562814073 6.02996296753654
0.0170854271356784 5.93006808543362
0.0291457286432161 5.82626952795936
0.0412060301507538 5.71843719291106
0.0532663316582915 5.60642568897164
0.0653266331658292 5.49007412788304
0.0773869346733668 5.36920760813007
0.0894472361809046 5.24364104136716
0.101507537688442 5.11318603331694
0.11356783919598 4.97766152644543
0.125628140703518 4.83690881449077
0.137688442211055 4.69081130250645
0.149748743718593 4.53931897713814
0.161809045226131 4.38247694941929
0.173869346733668 4.22045664262365
0.185929648241206 4.05358728712307
0.197989949748744 3.88238446876047
0.210050251256281 3.70757176290828
0.222110552763819 3.53009121366829
0.234170854271357 3.35109884983081
0.246231155778894 3.17194276112861
0.258291457286432 2.99412352470432
0.27035175879397 2.81923976616166
0.282412060301508 2.64892486364589
0.294472361809045 2.48478351087494
0.306532663316583 2.32833817472422
0.318592964824121 2.18099464203531
0.330653266331658 2.04403239585604
0.342713567839196 1.91861955109528
0.354773869346734 1.80584417870536
0.366834170854271 1.70674538815316
0.378894472361809 1.62232056139909
0.390954773869347 1.55348245382482
0.403015075376884 1.50094486718089
0.415075376884422 1.46503123144864
0.42713567839196 1.44542712197097
0.439195979899497 1.44093037148911
0.451256281407035 1.44927866453635
0.463316582914573 1.46713758729309
0.475376884422111 1.49030004563194
0.487437185929648 1.51408485397569
0.499497487437186 1.53385235512472
0.511557788944724 1.54551166175605
0.523618090452261 1.54589946618042
0.535678391959799 1.53295996912637
0.547738693467337 1.50572252640025
0.559798994974875 1.46412746401061
0.571859296482412 1.40877415854175
0.58391959798995 1.34066015167455
0.595979899497487 1.26095815425813
0.608040201005025 1.17085281141914
0.620100502512563 1.07143958432024
0.632160804020101 0.963676750695352
0.644221105527638 0.848377191867576
0.656281407035176 0.726226787721228
0.668341708542714 0.597818505559056
0.680402010050251 0.463694040168495
0.692462311557789 0.32438731237536
0.704522613065327 0.180465975603184
0.716582914572864 0.0325683379717039
0.728643216080402 -0.118566075983825
0.74070351758794 -0.27207355814606
0.752763819095478 -0.426951682683291
0.764824120603015 -0.582053710519418
0.776884422110553 -0.736092139658136
0.788944723618091 -0.887651340082885
0.801005025125628 -1.03520899303093
0.813065326633166 -1.1771658860622
0.825125628140704 -1.31188347957674
0.837185929648241 -1.43772849688976
0.849246231155779 -1.55312349395787
0.861306532663317 -1.65660181032971
0.873366834170854 -1.74686434183995
0.885427135678392 -1.82283412068215
0.89748743718593 -1.88370275128427
0.909547738693467 -1.92896057620176
0.921608040201005 -1.95840059338188
0.933668341708543 -1.9720856083632
0.945728643216081 -1.97027024953081
0.957788944723618 -1.95327581127084
0.969849246231156 -1.92132745235005
0.981909547738694 -1.8743796236403
0.993969849246231 -1.81197370519398
1.00603015075377 -1.73318540040166
1.01809045226131 -1.63671969240958
1.03015075376884 -1.52118987922872
1.04221105527638 -1.38557171648803
1.05427135678392 -1.22976200679136
1.06633165829146 -1.05511295139341
1.078391959799 -0.864785379531581
1.09045226130653 -0.663785802303595
1.10251256281407 -0.458625646558682
1.11457286432161 -0.256644023890771
1.12663316582915 -0.0651306075715523
1.13869346733668 0.109562492233798
1.15075376884422 0.26273775481509
1.16281407035176 0.39177341025362
1.1748743718593 0.496205752484929
1.18693467336683 0.577506414960809
1.19899497487437 0.638639800494499
1.21105527638191 0.683505347581828
1.22311557788945 0.716362112325698
1.23517587939698 0.741310149673496
1.24723618090452 0.761875337373091
1.25929648241206 0.780719252971717
1.2713567839196 0.799477075980337
1.28341708542714 0.818714562935007
1.29547738693467 0.837988450872102
1.30753768844221 0.85599115647747
1.31959798994975 0.870758681620799
1.33165829145729 0.879919281691881
1.34371859296482 0.880959557408355
1.35577889447236 0.871484594683419
1.3678391959799 0.849450221296664
1.37989949748744 0.813348823507755
1.39195979899498 0.762335501725598
1.40402010050251 0.696288148565084
1.41608040201005 0.615802374776736
1.42814070351759 0.52212897434228
1.44020100502513 0.417066799834615
1.45226130653266 0.302826858761748
1.4643216080402 0.181883974434114
1.47638190954774 0.056830780876655
1.48844221105528 -0.0697542528600935
1.50050251256281 -0.195416879042048
1.51256281407035 -0.317911168529548
1.52462311557789 -0.435256963613047
1.53668341708543 -0.545772555321343
1.54874371859296 -0.648086111394771
1.5608040201005 -0.741130205907252
1.57286432160804 -0.824123954471479
1.58492462311558 -0.896546940052393
1.59698492462312 -0.958108512145041
1.60904522613065 -1.00871531810153
1.62110552763819 -1.04843919337307
1.63316582914573 -1.07748687324256
1.64522613065327 -1.09617242971612
1.6572864321608 -1.10489289705935
1.66934673366834 -1.10410722403945
1.68140703517588 -1.09431846470962
1.69346733668342 -1.0760589766039
1.70552763819095 -1.0498783151003
1.71758793969849 -1.01633347958378
1.72964824120603 -0.975981165488694
1.74170854271357 -0.92937169530682
1.75376884422111 -0.877044333400025
1.76582914572864 -0.819523725765559
1.77788944723618 -0.757317243751139
1.78994974874372 -0.690913047621605
1.80201005025126 -0.620778719351701
1.81407035175879 -0.547360342667454
1.82613065326633 -0.471081933922646
1.83819095477387 -0.392345148644131
1.85025125628141 -0.311529204682149
1.86231155778894 -0.228990977495387
1.87437185929648 -0.145065233223523
1.88643216080402 -0.0600649731306022
1.89849246231156 0.0257181300387008
1.9105527638191 0.112013219730557
1.92261306532663 0.198569679453356
1.93467336683417 0.285156502731781
1.94673366834171 0.371561617807493
1.95879396984925 0.457591174555051
1.97085427135678 0.543068800251282
1.98291457286432 0.627834830960961
1.99497487437186 0.711745524080019
2.0070351758794 0.79467225816381
2.01909547738693 0.876500725648696
2.03115577889447 0.957130123214262
2.04321608040201 1.03647234560315
2.05527638190955 1.11445118689469
2.06733668341709 1.19100155377083
2.07939698492462 1.26606869507882
2.09145728643216 1.33960745017399
2.1035175879397 1.41158152048245
2.11557788944724 1.48196276567023
2.12763819095477 1.55073052708516
2.13969849246231 1.61787098073654
2.15175879396985 1.68337651984731
2.16381909547739 1.74724516937091
2.17587939698492 1.80948003175644
2.18793969849246 1.87008876522996
2.2 1.92908309386792
};
\addlegendentry{Mean}
\addplot [semithick, red, forget plot]
table {%
-0.2 7.24867568901548
-0.187939698492462 7.19590358873445
-0.175879396984925 7.14124601796166
-0.163819095477387 7.08462499020852
-0.151758793969849 7.02595859954269
-0.139698492462312 6.96516067327429
-0.127638190954774 6.90214037827348
-0.115577889447236 6.83680177884236
-0.103517587939698 6.76904334632426
-0.0914572864321608 6.69875742404539
-0.0793969849246231 6.6258296560714
-0.0673366834170854 6.55013839498866
-0.0552763819095477 6.47155411287053
-0.0432160804020101 6.3899388511344
-0.0311557788944724 6.30514575944335
-0.0190954773869347 6.21701879132138
-0.00703517587939698 6.1253926446503
0.00502512562814073 6.03009305822542
0.0170854271356784 5.93093760004066
0.0291457286432161 5.82773710718453
0.0412060301507538 5.72029795848551
0.0532663316582915 5.60842537565601
0.0653266331658292 5.49192795195375
0.0773869346733668 5.37062359383999
0.0894472361809046 5.24434702506691
0.101507537688442 5.11295893906314
0.11356783919598 4.97635679140713
0.125628140703518 4.83448710017482
0.137688442211055 4.68735897377169
0.149748743718593 4.53505842547647
0.161809045226131 4.37776287944607
0.173869346733668 4.21575514687847
0.185929648241206 4.04943607669059
0.197989949748744 3.87933508141797
0.210050251256281 3.70611781539401
0.222110552763819 3.53059043436181
0.234170854271357 3.35370007471994
0.246231155778894 3.17653142557618
0.258291457286432 3.00029948855522
0.27035175879397 2.82633878485409
0.282412060301508 2.65608932732422
0.294472361809045 2.49107957040221
0.306532663316583 2.33290621697894
0.318592964824121 2.18321013560277
0.330653266331658 2.04364669521146
0.342713567839196 1.91584762280493
0.354773869346734 1.80137027018393
0.366834170854271 1.70162942793025
0.378894472361809 1.6178073156897
0.390954773869347 1.5507400544463
0.403015075376884 1.50078459371033
0.415075376884422 1.46767878961759
0.42713567839196 1.45041767399629
0.439195979899497 1.44717748587621
0.451256281407035 1.45532080089253
0.463316582914573 1.47150663529401
0.475376884422111 1.49190791886951
0.487437185929648 1.51251047016641
0.499497487437186 1.52944299548483
0.511557788944724 1.5392774101215
0.523618090452261 1.53924777682774
0.535678391959799 1.52736026257848
0.547738693467337 1.50239525510763
0.559798994974875 1.46382543490621
0.571859296482412 1.41168434284617
0.58391959798995 1.34641911982289
0.595979899497487 1.26875285462352
0.608040201005025 1.179571349011
0.620100502512563 1.07983969995335
0.632160804020101 0.970547543231725
0.644221105527638 0.852678210953086
0.656281407035176 0.727195760434736
0.668341708542714 0.595043984014542
0.680402010050251 0.457152395531045
0.692462311557789 0.314445330181133
0.704522613065327 0.167851421581666
0.716582914572864 0.0183117109703086
0.728643216080402 -0.133214539776085
0.74070351758794 -0.285746616976165
0.752763819095478 -0.438281510401979
0.764824120603015 -0.589794927057618
0.776884422110553 -0.739243441280149
0.788944723618091 -0.88556699334338
0.801005025125628 -1.02769100071174
0.813065326633166 -1.164527501375
0.825125628140704 -1.29497498991212
0.837185929648241 -1.41791691504562
0.849246231155779 -1.53221916368967
0.861306532663317 -1.63672724359393
0.873366834170854 -1.73026427833148
0.885427135678392 -1.81163132624899
0.89748743718593 -1.87961190320013
0.909547738693467 -1.93298288639373
0.921608040201005 -1.97053413925676
0.933668341708543 -1.99109913148179
0.945728643216081 -1.9935984126579
0.957788944723618 -1.97709689523318
0.969849246231156 -1.94087439350784
0.981909547738694 -1.88450670339379
0.993969849246231 -1.8079517945402
1.00603015075377 -1.71163274860337
1.01809045226131 -1.59650650656458
1.03015075376884 -1.46410609405966
1.04221105527638 -1.31654462810105
1.05427135678392 -1.15647265997781
1.06633165829146 -0.986986255637398
1.078391959799 -0.811490796210921
1.09045226130653 -0.6335331668618
1.10251256281407 -0.456620826601057
1.11457286432161 -0.284048594561349
1.12663316582915 -0.118752197702156
1.13869346733668 0.0367977265354205
1.15075376884422 0.180654840550517
1.16281407035176 0.311410650912182
1.1748743718593 0.428169432469408
1.18693467336683 0.530495082607193
1.19899497487437 0.618341452391358
1.21105527638191 0.691976696436971
1.22311557788945 0.751909953973284
1.23517587939698 0.798825979363388
1.24723618090452 0.833530753675198
1.25929648241206 0.856908965660246
1.2713567839196 0.869892682098542
1.28341708542714 0.873439527277184
1.29547738693467 0.868518180625305
1.30753768844221 0.856098877421822
1.31959798994975 0.837146759858236
1.33165829145729 0.812616286565955
1.34371859296482 0.783445391558032
1.35577889447236 0.750548619534196
1.3678391959799 0.714808989881389
1.37989949748744 0.677068798996254
1.39195979899498 0.63811991331617
1.40402010050251 0.598694304398998
1.41608040201005 0.55945562512577
1.42814070351759 0.520992537914159
1.44020100502513 0.48381431583639
1.45226130653266 0.448348990867952
1.4643216080402 0.414944066821041
1.47638190954774 0.383869587436202
1.48844221105528 0.355323179440805
1.50050251256281 0.329436588226164
1.51256281407035 0.30628318894439
1.52462311557789 0.285885977443845
1.53668341708543 0.268225607538989
1.54874371859296 0.253248126542515
1.5608040201005 0.240872154552414
1.57286432160804 0.230995343001239
1.58492462311558 0.223500026791292
1.59698492462312 0.218258048015834
1.60904522613065 0.215134776789071
1.62110552763819 0.213992387075605
1.63316582914573 0.214692464784923
1.64522613065327 0.217098034428747
1.6572864321608 0.221075091994987
1.66934673366834 0.226493727769496
1.68140703517588 0.233228915626415
1.69346733668342 0.241161036356613
1.70552763819095 0.250176193042851
1.71758793969849 0.260166367093624
1.72964824120603 0.271029454803686
1.74170854271357 0.282669216489721
1.75376884422111 0.29499516347084
1.76582914572864 0.307922402436821
1.77788944723618 0.321371452016464
1.78994974874372 0.335268042530452
1.80201005025126 0.349542906876126
1.81407035175879 0.364131568129409
1.82613065326633 0.378974127649832
1.83819095477387 0.394015056135516
1.85025125628141 0.409202989104526
1.86231155778894 0.4244905275988
1.87437185929648 0.439834044449901
1.88643216080402 0.455193496157268
1.89849246231156 0.470532240264361
1.9105527638191 0.485816858040484
1.92261306532663 0.501016982257437
1.93467336683417 0.516105129869536
1.94673366834171 0.531056539446087
1.95879396984925 0.545849013255474
1.97085427135678 0.560462763951404
1.98291457286432 0.574880265858725
1.99497487437186 0.589086110895446
2.0070351758794 0.603066869197251
2.01909547738693 0.616810954530383
2.03115577889447 0.630308494588596
2.04321608040201 0.643551206270745
2.05527638190955 0.656532276028658
2.06733668341709 0.669246245361732
2.07939698492462 0.681688901516442
2.09145728643216 0.69385717342717
2.1035175879397 0.705749032910674
2.11557788944724 0.71736340110133
2.12763819095477 0.728700060088809
2.13969849246231 0.739759569695229
2.15175879396985 0.750543189305271
2.16381909547739 0.761052804641239
2.17587939698492 0.771290859355482
2.18793969849246 0.781260291295607
2.2 0.79096447328328
};
\path [draw=steelblue31119180, fill=steelblue31119180, opacity=0.2]
(axis cs:-0.2,12.0372516966282)
--(axis cs:-0.2,2.40341300168499)
--(axis cs:-0.187939698492462,2.70072622447415)
--(axis cs:-0.175879396984925,2.98805414807005)
--(axis cs:-0.163819095477387,3.26421329069681)
--(axis cs:-0.151758793969849,3.52791911334948)
--(axis cs:-0.139698492462312,3.77776691558578)
--(axis cs:-0.127638190954774,4.01220848389338)
--(axis cs:-0.115577889447236,4.22952467246238)
--(axis cs:-0.103517587939698,4.42779568171396)
--(axis cs:-0.0914572864321608,4.60487448631751)
--(axis cs:-0.0793969849246231,4.75837659791213)
--(axis cs:-0.0673366834170854,4.88571371836987)
--(axis cs:-0.0552763819095477,4.98422078193783)
--(axis cs:-0.0432160804020101,5.05144683873124)
--(axis cs:-0.0311557788944724,5.08566781623375)
--(axis cs:-0.0190954773869347,5.08657605474796)
--(axis cs:-0.00703517587939698,5.05589328211584)
--(axis cs:0.00502512562814073,4.99750021545681)
--(axis cs:0.0170854271356784,4.91684793683211)
--(axis cs:0.0291457286432161,4.81987576763946)
--(axis cs:0.0412060301507538,4.71194764826379)
--(axis cs:0.0532663316582915,4.59717533019192)
--(axis cs:0.0653266331658292,4.47818149405189)
--(axis cs:0.0773869346733668,4.35617785938055)
--(axis cs:0.0894472361809046,4.23122384714828)
--(axis cs:0.101507537688442,4.10258305752529)
--(axis cs:0.11356783919598,3.96913106790429)
--(axis cs:0.125628140703518,3.82976979300184)
--(axis cs:0.137688442211055,3.6837843070856)
--(axis cs:0.149748743718593,3.53106493313875)
--(axis cs:0.161809045226131,3.37213602530238)
--(axis cs:0.173869346733668,3.20798807075906)
--(axis cs:0.185929648241206,3.03977897099989)
--(axis cs:0.197989949748744,2.86852061107078)
--(axis cs:0.210050251256281,2.69487798647023)
--(axis cs:0.222110552763819,2.51917706330281)
--(axis cs:0.234170854271357,2.34164658898451)
--(axis cs:0.246231155778894,2.16281604163854)
--(axis cs:0.258291457286432,1.98389031792286)
--(axis cs:0.27035175879397,1.80688940607242)
--(axis cs:0.282412060301508,1.63443379872251)
--(axis cs:0.294472361809045,1.46924296309955)
--(axis cs:0.306532663316583,1.31357776047065)
--(axis cs:0.318592964824121,1.16889280604772)
--(axis cs:0.330653266331658,1.03585551295608)
--(axis cs:0.342713567839196,0.914695538309928)
--(axis cs:0.354773869346734,0.805667985105772)
--(axis cs:0.366834170854271,0.709353763472104)
--(axis cs:0.378894472361809,0.626642715709888)
--(axis cs:0.390954773869347,0.558478527543229)
--(axis cs:0.403015075376884,0.505596648698394)
--(axis cs:0.415075376884422,0.468409503913204)
--(axis cs:0.42713567839196,0.446952075354797)
--(axis cs:0.439195979899497,0.440654949494222)
--(axis cs:0.451256281407035,0.447871903884292)
--(axis cs:0.463316582914573,0.465442851363149)
--(axis cs:0.475376884422111,0.488736861310755)
--(axis cs:0.487437185929648,0.512354699982146)
--(axis cs:0.499497487437186,0.531180202048726)
--(axis cs:0.511557788944724,0.541215402921844)
--(axis cs:0.523618090452261,0.539846978229296)
--(axis cs:0.535678391959799,0.525625641976536)
--(axis cs:0.547738693467337,0.497883786656016)
--(axis cs:0.559798994974875,0.45644952703268)
--(axis cs:0.571859296482412,0.401522591990458)
--(axis cs:0.58391959798995,0.333648585633413)
--(axis cs:0.595979899497487,0.253703421388906)
--(axis cs:0.608040201005025,0.16283234531801)
--(axis cs:0.620100502512563,0.0623316387434372)
--(axis cs:0.632160804020101,-0.0465048884407876)
--(axis cs:0.644221105527638,-0.162534115231301)
--(axis cs:0.656281407035176,-0.284864481395732)
--(axis cs:0.668341708542714,-0.4128808285729)
--(axis cs:0.680402010050251,-0.546187549649295)
--(axis cs:0.692462311557789,-0.684489562323619)
--(axis cs:0.704522613065327,-0.827449806853467)
--(axis cs:0.716582914572864,-0.974568043798782)
--(axis cs:0.728643216080402,-1.12511537190816)
--(axis cs:0.74070351758794,-1.27813550860949)
--(axis cs:0.752763819095478,-1.43249755829902)
--(axis cs:0.764824120603015,-1.58696755221476)
--(axis cs:0.776884422110553,-1.74026436889538)
--(axis cs:0.788944723618091,-1.8910784904852)
--(axis cs:0.801005025125628,-2.03805213231983)
--(axis cs:0.813065326633166,-2.179737828926)
--(axis cs:0.825125628140704,-2.31456302938136)
--(axis cs:0.837185929648241,-2.44082805133172)
--(axis cs:0.849246231155779,-2.55675524192143)
--(axis cs:0.861306532663317,-2.66059238416166)
--(axis cs:0.873366834170854,-2.75075749765348)
--(axis cs:0.885427135678392,-2.82599783110279)
--(axis cs:0.89748743718593,-2.88552425436813)
--(axis cs:0.909547738693467,-2.92907586463367)
--(axis cs:0.921608040201005,-2.95687461761727)
--(axis cs:0.933668341708543,-2.96945483247787)
--(axis cs:0.945728643216081,-2.96740019531806)
--(axis cs:0.957788944723618,-2.95107566414397)
--(axis cs:0.969849246231156,-2.92046633481646)
--(axis cs:0.981909547738694,-2.87519352075115)
--(axis cs:0.993969849246231,-2.81467347189904)
--(axis cs:1.00603015075377,-2.73827808196145)
--(axis cs:1.01809045226131,-2.64534274489032)
--(axis cs:1.03015075376884,-2.53500544083693)
--(axis cs:1.04221105527638,-2.40610033039254)
--(axis cs:1.05427135678392,-2.25746191946229)
--(axis cs:1.06633165829146,-2.08880410734113)
--(axis cs:1.078391959799,-1.90188019206425)
--(axis cs:1.09045226130653,-1.70127670421615)
--(axis cs:1.10251256281407,-1.49426629224193)
--(axis cs:1.11457286432161,-1.28961731401398)
--(axis cs:1.12663316582915,-1.09583104708343)
--(axis cs:1.13869346733668,-0.919585985726894)
--(axis cs:1.15075376884422,-0.76499098370297)
--(axis cs:1.16281407035176,-0.633718021888032)
--(axis cs:1.1748743718593,-0.525625068356583)
--(axis cs:1.18693467336683,-0.439390273370006)
--(axis cs:1.19899497487437,-0.372892027718809)
--(axis cs:1.21105527638191,-0.323341971737567)
--(axis cs:1.22311557788945,-0.287345291657879)
--(axis cs:1.23517587939698,-0.261079956202939)
--(axis cs:1.24723618090452,-0.240681245193972)
--(axis cs:1.25929648241206,-0.222768291752852)
--(axis cs:1.2713567839196,-0.20495106501719)
--(axis cs:1.28341708542714,-0.186157560332503)
--(axis cs:1.29547738693467,-0.166696609174159)
--(axis cs:1.30753768844221,-0.148064077512495)
--(axis cs:1.31959798994975,-0.132568831142598)
--(axis cs:1.33165829145729,-0.122887042182418)
--(axis cs:1.34371859296482,-0.12165068356812)
--(axis cs:1.35577889447236,-0.13114508216456)
--(axis cs:1.3678391959799,-0.153143723526279)
--(axis cs:1.37989949748744,-0.188863926798309)
--(axis cs:1.39195979899498,-0.238999774803283)
--(axis cs:1.40402010050251,-0.303783412202329)
--(axis cs:1.41608040201005,-0.383036690636754)
--(axis cs:1.42814070351759,-0.476192935904022)
--(axis cs:1.44020100502513,-0.582286894527358)
--(axis cs:1.45226130653266,-0.699926736148567)
--(axis cs:1.4643216080402,-0.827273076245549)
--(axis cs:1.47638190954774,-0.962052876987785)
--(axis cs:1.48844221105528,-1.10162794921868)
--(axis cs:1.50050251256281,-1.24312029933656)
--(axis cs:1.51256281407035,-1.38357725994988)
--(axis cs:1.52462311557789,-1.52014727344804)
--(axis cs:1.53668341708543,-1.65023681099649)
--(axis cs:1.54874371859296,-1.77162784265652)
--(axis cs:1.5608040201005,-1.88254749276236)
--(axis cs:1.57286432160804,-1.98169181341997)
--(axis cs:1.58492462311558,-2.06821175322597)
--(axis cs:1.59698492462312,-2.14167165169848)
--(axis cs:1.60904522613065,-2.20199029590421)
--(axis cs:1.62110552763819,-2.24937306527974)
--(axis cs:1.63316582914573,-2.28424187157043)
--(axis cs:1.64522613065327,-2.30716792053327)
--(axis cs:1.6572864321608,-2.31881092709543)
--(axis cs:1.66934673366834,-2.31986728043618)
--(axis cs:1.68140703517588,-2.31102869658335)
--(axis cs:1.69346733668342,-2.29295204029479)
--(axis cs:1.70552763819095,-2.26624020963493)
--(axis cs:1.71758793969849,-2.2314332724724)
--(axis cs:1.72964824120603,-2.18900847276936)
--(axis cs:1.74170854271357,-2.13938733965581)
--(axis cs:1.75376884422111,-2.08294797114832)
--(axis cs:1.76582914572864,-2.02004062014182)
--(axis cs:1.77788944723618,-1.95100494762426)
--(axis cs:1.78994974874372,-1.87618765730576)
--(axis cs:1.80201005025126,-1.79595961728485)
--(axis cs:1.81407035175879,-1.71073194030719)
--(axis cs:1.82613065326633,-1.6209707846866)
--(axis cs:1.83819095477387,-1.52721081862633)
--(axis cs:1.85025125628141,-1.43006733167076)
--(axis cs:1.86231155778894,-1.33024685828639)
--(axis cs:1.87437185929648,-1.22855587411559)
--(axis cs:1.88643216080402,-1.12590661719021)
--(axis cs:1.89849246231156,-1.02331838726175)
--(axis cs:1.9105527638191,-0.921911871685224)
--(axis cs:1.92261306532663,-0.822893371246743)
--(axis cs:1.93467336683417,-0.727525689352663)
--(axis cs:1.94673366834171,-0.637083512414379)
--(axis cs:1.95879396984925,-0.552793881597556)
--(axis cs:1.97085427135678,-0.475766778059749)
--(axis cs:1.98291457286432,-0.406925716823653)
--(axis cs:1.99497487437186,-0.346951261386128)
--(axis cs:2.0070351758794,-0.296249247281935)
--(axis cs:2.01909547738693,-0.254949784296247)
--(axis cs:2.03115577889447,-0.222935024825775)
--(axis cs:2.04321608040201,-0.199886909641106)
--(axis cs:2.05527638190955,-0.185343291176736)
--(axis cs:2.06733668341709,-0.178752149731648)
--(axis cs:2.07939698492462,-0.179517302237995)
--(axis cs:2.09145728643216,-0.187032953170438)
--(axis cs:2.1035175879397,-0.200707344410082)
--(axis cs:2.11557788944724,-0.219977316392519)
--(axis cs:2.12763819095477,-0.24431606432471)
--(axis cs:2.13969849246231,-0.273236215473308)
--(axis cs:2.15175879396985,-0.306289930234395)
--(axis cs:2.16381909547739,-0.343067264769994)
--(axis cs:2.17587939698492,-0.383193634411652)
--(axis cs:2.18793969849246,-0.426326911678581)
--(axis cs:2.2,-0.472154479203049)
--(axis cs:2.2,4.33032066693889)
--(axis cs:2.2,4.33032066693889)
--(axis cs:2.18793969849246,4.1665044421385)
--(axis cs:2.17587939698492,4.00215369792453)
--(axis cs:2.16381909547739,3.83755760351182)
--(axis cs:2.15175879396985,3.67304296992902)
--(axis cs:2.13969849246231,3.50897817694638)
--(axis cs:2.12763819095477,3.34577711849502)
--(axis cs:2.11557788944724,3.18390284773298)
--(axis cs:2.1035175879397,3.02387038537498)
--(axis cs:2.09145728643216,2.86624785351841)
--(axis cs:2.07939698492462,2.71165469239563)
--(axis cs:2.06733668341709,2.5607552572733)
--(axis cs:2.05527638190955,2.41424566496613)
--(axis cs:2.04321608040201,2.27283160084741)
--(axis cs:2.03115577889447,2.1371952712543)
--(axis cs:2.01909547738693,2.00795123559364)
--(axis cs:2.0070351758794,1.88559376360956)
--(axis cs:1.99497487437186,1.77044230954617)
--(axis cs:1.98291457286432,1.66259537874557)
--(axis cs:1.97085427135678,1.56190437856231)
--(axis cs:1.95879396984925,1.46797623070766)
--(axis cs:1.94673366834171,1.38020674802936)
--(axis cs:1.93467336683417,1.29783869481622)
--(axis cs:1.92261306532663,1.22003273015346)
--(axis cs:1.9105527638191,1.14593831114634)
--(axis cs:1.89849246231156,1.07475464733915)
--(axis cs:1.88643216080402,1.00577667092901)
--(axis cs:1.87437185929648,0.938425407668545)
--(axis cs:1.86231155778894,0.872264903295621)
--(axis cs:1.85025125628141,0.807008922306459)
--(axis cs:1.83819095477387,0.742520521338065)
--(axis cs:1.82613065326633,0.678806916841305)
--(axis cs:1.81407035175879,0.616011254972281)
--(axis cs:1.80201005025126,0.554402178581445)
--(axis cs:1.78994974874372,0.494361562062552)
--(axis cs:1.77788944723618,0.436370460121981)
--(axis cs:1.76582914572864,0.380993168610703)
--(axis cs:1.75376884422111,0.328859304348266)
--(axis cs:1.74170854271357,0.280643949042174)
--(axis cs:1.72964824120603,0.237046141791974)
--(axis cs:1.71758793969849,0.198766313304847)
--(axis cs:1.70552763819095,0.166483579434319)
--(axis cs:1.69346733668342,0.140834087086989)
--(axis cs:1.68140703517588,0.122391767164106)
--(axis cs:1.66934673366834,0.111652832357286)
--(axis cs:1.6572864321608,0.109025132976718)
--(axis cs:1.64522613065327,0.114823061101037)
--(axis cs:1.63316582914573,0.129268125085296)
--(axis cs:1.62110552763819,0.152494678533598)
--(axis cs:1.60904522613065,0.184559659701159)
--(axis cs:1.59698492462312,0.225454627408403)
--(axis cs:1.58492462311558,0.27511787312118)
--(axis cs:1.57286432160804,0.33344390447701)
--(axis cs:1.5608040201005,0.400287080947857)
--(axis cs:1.54874371859296,0.475455619866979)
--(axis cs:1.53668341708543,0.558691700353802)
--(axis cs:1.52462311557789,0.649633346221942)
--(axis cs:1.51256281407035,0.747754922890785)
--(axis cs:1.50050251256281,0.852286541252469)
--(axis cs:1.48844221105528,0.962119443498491)
--(axis cs:1.47638190954774,1.0757144387411)
--(axis cs:1.4643216080402,1.19104102511378)
--(axis cs:1.45226130653266,1.30558045367206)
--(axis cs:1.44020100502513,1.41642049419659)
--(axis cs:1.42814070351759,1.52045088458858)
--(axis cs:1.41608040201005,1.61464144019023)
--(axis cs:1.40402010050251,1.6963597093325)
--(axis cs:1.39195979899498,1.76367077825448)
--(axis cs:1.37989949748744,1.81556157381382)
--(axis cs:1.3678391959799,1.85204416611961)
--(axis cs:1.35577889447236,1.8741142715314)
--(axis cs:1.34371859296482,1.88356979838483)
--(axis cs:1.33165829145729,1.88272560556618)
--(axis cs:1.31959798994975,1.8740861943842)
--(axis cs:1.30753768844221,1.86004639046743)
--(axis cs:1.29547738693467,1.84267351091836)
--(axis cs:1.28341708542714,1.82358668620252)
--(axis cs:1.2713567839196,1.80390521697786)
--(axis cs:1.25929648241206,1.78420679769629)
--(axis cs:1.24723618090452,1.76443191994015)
--(axis cs:1.23517587939698,1.74370025554993)
--(axis cs:1.22311557788945,1.72006951630927)
--(axis cs:1.21105527638191,1.69035266690122)
--(axis cs:1.19899497487437,1.65017162870781)
--(axis cs:1.18693467336683,1.59440310329162)
--(axis cs:1.1748743718593,1.51803657332644)
--(axis cs:1.16281407035176,1.41726484239527)
--(axis cs:1.15075376884422,1.29046649333315)
--(axis cs:1.13869346733668,1.13871097019449)
--(axis cs:1.12663316582915,0.96556983194033)
--(axis cs:1.11457286432161,0.776329266232442)
--(axis cs:1.10251256281407,0.577014999124568)
--(axis cs:1.09045226130653,0.373705099608955)
--(axis cs:1.078391959799,0.172309433001085)
--(axis cs:1.06633165829146,-0.0214217954456968)
--(axis cs:1.05427135678392,-0.202062094120438)
--(axis cs:1.04221105527638,-0.365043102583513)
--(axis cs:1.03015075376884,-0.5073743176205)
--(axis cs:1.01809045226131,-0.628096639928841)
--(axis cs:1.00603015075377,-0.728092718841873)
--(axis cs:0.993969849246231,-0.80927393848892)
--(axis cs:0.981909547738694,-0.873565726529452)
--(axis cs:0.969849246231156,-0.922188569883635)
--(axis cs:0.957788944723618,-0.955475958397709)
--(axis cs:0.945728643216081,-0.973140303743558)
--(axis cs:0.933668341708543,-0.974716384248528)
--(axis cs:0.921608040201005,-0.959926569146492)
--(axis cs:0.909547738693467,-0.928845287769847)
--(axis cs:0.89748743718593,-0.881881248200404)
--(axis cs:0.885427135678392,-0.819670410261509)
--(axis cs:0.873366834170854,-0.742971186026422)
--(axis cs:0.861306532663317,-0.652611236497764)
--(axis cs:0.849246231155779,-0.549491745994308)
--(axis cs:0.837185929648241,-0.434628942447797)
--(axis cs:0.825125628140704,-0.309203929772123)
--(axis cs:0.813065326633166,-0.1745939431984)
--(axis cs:0.801005025125628,-0.0323658537420339)
--(axis cs:0.788944723618091,0.115775810319428)
--(axis cs:0.776884422110553,0.268080089579104)
--(axis cs:0.764824120603015,0.422860131175924)
--(axis cs:0.752763819095478,0.578594192932437)
--(axis cs:0.74070351758794,0.733988392317371)
--(axis cs:0.728643216080402,0.887983219940508)
--(axis cs:0.716582914572864,1.03970471974219)
--(axis cs:0.704522613065327,1.18838175805984)
--(axis cs:0.692462311557789,1.33326418707434)
--(axis cs:0.680402010050251,1.47357562998629)
--(axis cs:0.668341708542714,1.60851783969101)
--(axis cs:0.656281407035176,1.73731805683819)
--(axis cs:0.644221105527638,1.85928849896645)
--(axis cs:0.632160804020101,1.97385838983149)
--(axis cs:0.620100502512563,2.08054752989704)
--(axis cs:0.608040201005025,2.17887327752027)
--(axis cs:0.595979899497487,2.26821288712735)
--(axis cs:0.58391959798995,2.34767171771569)
--(axis cs:0.571859296482412,2.41602572509303)
--(axis cs:0.559798994974875,2.47180540098854)
--(axis cs:0.547738693467337,2.51356126614449)
--(axis cs:0.535678391959799,2.5402942962762)
--(axis cs:0.523618090452261,2.55195195413155)
--(axis cs:0.511557788944724,2.54980792059025)
--(axis cs:0.499497487437186,2.53652450820071)
--(axis cs:0.487437185929648,2.51581500796923)
--(axis cs:0.475376884422111,2.49186322995313)
--(axis cs:0.463316582914573,2.46883232322302)
--(axis cs:0.451256281407035,2.4506854251884)
--(axis cs:0.439195979899497,2.441205793484)
--(axis cs:0.42713567839196,2.44390216858714)
--(axis cs:0.415075376884422,2.46165295898408)
--(axis cs:0.403015075376884,2.49629308566338)
--(axis cs:0.390954773869347,2.54848638010641)
--(axis cs:0.378894472361809,2.61799840708829)
--(axis cs:0.366834170854271,2.70413701283421)
--(axis cs:0.354773869346734,2.80602037230494)
--(axis cs:0.342713567839196,2.92254356388064)
--(axis cs:0.330653266331658,3.052209278756)
--(axis cs:0.318592964824121,3.1930964780229)
--(axis cs:0.306532663316583,3.34309858897778)
--(axis cs:0.294472361809045,3.50032405865034)
--(axis cs:0.282412060301508,3.66341592856928)
--(axis cs:0.27035175879397,3.83159012625091)
--(axis cs:0.258291457286432,4.00435673148579)
--(axis cs:0.246231155778894,4.18106948061869)
--(axis cs:0.234170854271357,4.36055111067711)
--(axis cs:0.222110552763819,4.54100536403378)
--(axis cs:0.210050251256281,4.72026553934633)
--(axis cs:0.197989949748744,4.89624832645017)
--(axis cs:0.185929648241206,5.06739560324624)
--(axis cs:0.173869346733668,5.23292521448824)
--(axis cs:0.161809045226131,5.3928178735362)
--(axis cs:0.149748743718593,5.54757302113754)
--(axis cs:0.137688442211055,5.6978382979273)
--(axis cs:0.125628140703518,5.8440478359797)
--(axis cs:0.11356783919598,5.98619198498657)
--(axis cs:0.101507537688442,6.12378900910859)
--(axis cs:0.0894472361809046,6.25605823558604)
--(axis cs:0.0773869346733668,6.38223735687959)
--(axis cs:0.0653266331658292,6.50196676171419)
--(axis cs:0.0532663316582915,6.61567604775135)
--(axis cs:0.0412060301507538,6.72492673755833)
--(axis cs:0.0291457286432161,6.83266328827925)
--(axis cs:0.0170854271356784,6.94328823403513)
--(axis cs:0.00502512562814073,7.06242571961627)
--(axis cs:-0.00703517587939698,7.19624736932846)
--(axis cs:-0.0190954773869347,7.35041268988422)
--(axis cs:-0.0311557788944724,7.52899160379446)
--(axis cs:-0.0432160804020101,7.73388059343748)
--(axis cs:-0.0552763819095477,7.96493681498775)
--(axis cs:-0.0673366834170854,8.22059509161352)
--(axis cs:-0.0793969849246231,8.49856084278943)
--(axis cs:-0.0914572864321608,8.79632555744914)
--(axis cs:-0.103517587939698,9.1114597120415)
--(axis cs:-0.115577889447236,9.44174120801947)
--(axis cs:-0.127638190954774,9.78518985031087)
--(axis cs:-0.139698492462312,10.1400574426293)
--(axis cs:-0.151758793969849,10.5048011414705)
--(axis cs:-0.163819095477387,10.878053339104)
--(axis cs:-0.175879396984925,11.2585936057277)
--(axis cs:-0.187939698492462,11.6453245294938)
--(axis cs:-0.2,12.0372516966282)
--cycle;
\addlegendimage{area legend, draw=steelblue31119180, fill=steelblue31119180, opacity=0.2}
\addlegendentry{95\% interval}

\draw (axis cs:0,-4.3) node[
  scale=0.5,
  anchor=base west,
  text=black,
  rotate=0.0
]{\inducing};
\end{axis}

\end{tikzpicture}
%
%  \end{subfigure}
  % 
  \begin{tikzpicture}[remember picture,overlay]
      % Arrow style
    \tikzstyle{myarrow} = [draw=black!80, single arrow, minimum height=3em, minimum width=3pt, single arrow head extend=6pt, fill=black!80, anchor=center, rotate=0, inner sep=10pt, rounded corners=1pt]
    % Arrows
    \node[myarrow] (p0-arr) at ($(p0) + (1.2em,1.5em)$) {};
    %\node[myarrow] (p1-arr) at ($(p1) + (1em,1.5em)$) {};
    % Arrow labels
    \node[font=\scriptsize\sc,color=white,scale=.8] at (p0-arr) {\our};
    %\node[font=\scriptsize\sc,color=white] at (p1-arr) {new data};   
  \end{tikzpicture}
  %\end{figure}
    \alert{\bf Regression w/ 2-layers MLP.} 
    Prediction from a trained neural network's predictions {\it (left)}, our approach using inducing points to summarize the training data {\it (middle)}. SFR captures the predictive mean and uncertainty, and can incorporate new data without retraining the model {\it (right)}.
   \end{minipage}\\[2cm]

  \begin{block}{Background: NN's Function-space Representation}
  \alert{\bf Inputs} in a {\it supervised setting} for NNs $f_\mathbf{w}: \inputDomain \to \outputDomain$:\
  \begin{itemize}
  \item $\dataset = \{(\vx_{i} , \vy_{i})\}_{i=1}^{N}$, a data set w/ input $\vx_i \in \inputDomain$ and output $\vy_i \in \outputDomain$;
  \item $\weights \in \R^{P}$, the initial weights of the neural network.
  \end{itemize}
\alert{\bf Goal: }
  minimize the empirical (regularized) risk loss function:
\begin{align} 
  \weights^{*} =  \arg \min_{\weights} \mathcal{L}(\dataset,\weights) \nonumber
     = \arg \min_{\weights} \textstyle\sum_{i=1}^{N} \ell(f_\weights(\mathbf{x}_{i}), y_i) + \delta \mathcal{R}(\weights).
 \end{align}
\alert{\bf Output:} $\weights^*$, the Maximum A-Posteriori (MAP) weights of the NN.\\[0.5cm]

\alert{\bf How to capture distribution over NN model functions?}	\\
Use their first two moments, obtaining a Gaussian process with a \alert{mean function $\mu(\cdot)$} and a \alert{covariance function $\kappa(\cdot,\cdot)$} (or kernel). \\
{\it For GPs}, linear approximations in weight space lead to function-space equivalent approximations:
\begin{equation*}
	f_\weights(\vx) \approx 
\phi^\top\!(\vx) \, \vw \quad\implies\quad \mu(\vx) = 0 \quad \text{and} \quad \kappa(\vx, \vx') = \frac{1}{\delta} \, \phi^\T\!(\vx) \, \phi(\vx')
\end{equation*}\\
{\it For NNs}, we can use the \alert{Laplace-GGN approximation} to get a linear model of the neural network at the MAP as:
\begin{equation*}
	f_{\weights^*}(\vx) \approx \Jac{\weights_*}{\vx} \, \weights \implies   \mu(\vx) =  0 \quad \text{and} \quad
  \kappa(\vx, \vx')
  = \frac{1}{\delta} \, \Jac{\weights^*}{\vx} \, \JacT{\weights^*}{\vx'}, 
\end{equation*}
where $\Jac{\weights}{\vx} \coloneqq \left[ \nabla_\weights f_\weights(\vx)\right]^\top \in \R^{C \times P}$ is the Jacobian at $\weights^*$.
%\begin{itemize}
%	\item Maximum a-posteriori 
%\end{itemize}
\end{block}
\end{column}

\separatorcolumn

\begin{column}{\colwidth}
\begin{minipage}{\textwidth}
  % Colours
  \definecolor{C0}{HTML}{DF6679}
  \definecolor{C1}{HTML}{69A9CE}
\begin{figure}[t]
  \centering
  % Set figure size
  \setlength{\figurewidth}{.31\textwidth}
  \setlength{\figureheight}{\figurewidth}
  %
  \begin{tikzpicture}[outer sep=0,inner sep=0]
% This command wasn't working 
%    \newcommand{\addfig}[2]{
%    \begin{scope}
%      \clip[rounded corners=3pt] ($(#1)+(-.5\figurewidth,-.5\figureheight)$) rectangle ++(\figurewidth,\figureheight);
%      \node (#2) at (#1) {\includegraphics[width=1.05\figurewidth]{./fig/#2}};
%    \end{scope}
%    %\draw[rounded corners=3pt,line width=1.2pt,black!60] ($(#1)+(-.5\figurewidth,-.5\figureheight)$) rectangle ++(\figurewidth,\figureheight);
%    }
%
%    % The neural network
%    \addfig{0,0}{banana-nn}
    \begin{scope}
      \clip[rounded corners=3pt] ($(0,0)+(-.5\figurewidth,-.5\figureheight)$) rectangle ++(\figurewidth,\figureheight);
      \node (banana-nn) at (0,0) {\includegraphics[width=1.05\figurewidth]{./fig/banana-nn}};
    \end{scope}
    
    % The nn2svgp
    %\addfig{1.1\figurewidth,0}{banana-nn2svgp}
    \begin{scope}
      \clip[rounded corners=3pt] ($(1.1\figurewidth,0)+(-.5\figurewidth,-.5\figureheight)$) rectangle ++(\figurewidth,\figureheight);
      \node (banana-nn2svgp) at (1.1\figurewidth,0) {\includegraphics[width=1.05\figurewidth]{./fig/banana-nn2svgp}};
    \end{scope}
    

    % The update
    %\addfig{2.2\figurewidth,0}{banana-hmc}
    \begin{scope}
      \clip[rounded corners=3pt] ($(2.2\figurewidth,0)+(-.5\figurewidth,-.5\figureheight)$) rectangle ++(\figurewidth,\figureheight);
      \node (banana-hmc) at (2.2\figurewidth,0) {\includegraphics[width=1.05\figurewidth]{./fig/banana-hmc}};
    \end{scope}
	% The arrow
    \tikzstyle{myarrow} = [draw=black!80, single arrow, minimum height=3em, minimum width=3pt, single arrow head extend=6pt, fill=black!80, anchor=center, rotate=0, inner sep=10pt, rounded corners=1pt]
    \tikzstyle{myblock} = [draw=black!80, minimum height=8mm, minimum width=7mm, fill=black!80, anchor=center, rotate=0, inner sep=5pt, rounded corners=1pt]
    \node[myarrow] (first-arr) at ($(banana-nn)!0.5!(banana-nn2svgp)$) {};
    \node[myblock] (second-arr) at ($(banana-nn2svgp)!0.5!(banana-hmc)$) {};

    % Arrow labels
    \node[font=\scriptsize\sc,color=white,scale=.8] at (first-arr) {\our};
    \node[font=\scriptsize,color=white] at (second-arr) {\normalsize$\bm\approx$};
         
    % Labels
    \node[anchor=north, font=\footnotesize] at ($(banana-nn) + (0,-.55\figureheight)$) {Neural network prediction};
    \node[anchor=north, font=\footnotesize] at ($(banana-nn2svgp) + (0,-.55\figureheight)$) {Sparse function-space representation};
    \node[anchor=north, font=\footnotesize] at ($(banana-hmc) + (0,-.55\figureheight)$) {HMC result as baseline};      
    
  \end{tikzpicture}

\end{figure}
  \newcommand{\mycircle}{\protect\tikz[baseline=-.6ex]\protect\node[circle,inner sep=5pt,draw=black,fill=C0,opacity=.5]{};}
  \newcommand{\mysquare}{\protect\tikz[baseline=-.6ex]\protect\node[inner sep=8pt,draw=black,fill=C1,opacity=.5]{};}
  \newcommand{\myinducing}{\protect\tikz[baseline=-.7ex]\protect\node[circle,inner sep=5pt,draw=black,fill=black]{};}
\alert{\bf Uncertainty quantification for classification (\,\mysquare~vs.~\mycircle).} We convert the trained neural network {\it (left)} to a sparse GP model with a set of inducing points~\myinducing\ {\it(middle)}. Results show a similar behaviour as running full Hamiltonian Monte Carlo (HMC) on the original NN model weights {\it (right)}. Marginal uncertainty depicted by colour intensity.
\end{minipage}\\[1.5cm]

 \begin{block}{SFR: Sparse Function Representation}
\alert{TODO: Leave or remove these???}
  \begin{align*}
  \alpha_i &\coloneqq \myexpect_{p(\vw \mid \vy)}[\nabla_{f}\log p(y_i \mid f) |_{f=f_i}]
  \quad \text{and} \\
  \beta_i &\coloneqq - \myexpect_{p(\vw \mid \vy)}[\nabla^2_{f f}\log p(y_i \mid f_i) |_{f=f_i}]
\end{align*}

%\heading{Dual parameterization}
\alert{\bf Dual parameterization.} The first two moments of the resultant posterior process using the Laplace can be approximated via the dual parameters $\hat{\valpha}, \hat{\vbeta} \in \R^{N}$: 
\begin{equation*}
  \hat{\alpha}_i \coloneqq \nabla_{f}\log p(y_i \mid f) |_{f=f_i}
  \quad \text{and} \quad
  \hat{\beta}_i \coloneqq - \nabla^2_{ff}\log p(y_i \mid f) |_{f=f_i}.
\end{equation*}
But it scales w/ $\mathcal{O}(N^3)$ $\rightarrow$ Sparsify the resulting GP model to get $\mathcal{O}(M^3)$
%\heading{Sparse Neural Network GP}
\alert{\bf Sparsifying the NN's GP.}
\begin{equation*}
\valpha_{\vu}  =  \sum_{i=1}^N  \vkzi \, \hat{\alpha}_{i} \in \R^M
\; \text{and} \;
  \MBeta_{\vu} =  \sum_{i=1}^N \vkzi \,\hat{\beta}_{i} \, \vkzi^{\T} \in \R^{M \times M}
\end{equation*}
Using this sparse definition of the dual variables, our sparse GP posterior takes the following form:
\begin{align*}   
\myexpect_{q_{\vu}(\vf)}[f_i] &= \vkzs^{\T} \MKzz^{-1} \valpha_{\vu}
   \quad \text{and} \\ 
\textrm{Var}_{q_{\vu}(\vf)}[f_i] &= k_{ii} - \vkzs^\top [\MKzz^{-1} - (\MKzz + \MBeta_{\vu})^{-1} ]\vkzs
\end{align*}



\end{block}
\end{column}

\separatorcolumn
\end{columns}

\begin{columns}
	\begin{column}{2.05\colwidth}
		  \begin{block}{Results and Discussion}

%\begin{minipage}{\textwidth}
%\begin{table}
%  \centering\scriptsize
%  %\caption{Comparisons and ablations on UCI data with negative log predictive density (NLPD\textcolor{gray}{\footnotesize$\pm$std}, lower better). Our sparse \our ($M=256$) is on par with full models (left) and outperforms the GP subset approach of \citet{immer2021improving} (right). Results for methods marked with * as reported in the original benchmark~\citep{immer2021improving}.} %See \cref{app:uci} for additional tables with comparisons.}
%	
%	% Control table spacing
%	\renewcommand{\arraystretch}{1.}
%	\setlength{\tabcolsep}{1.2pt}
%	\setlength{\tblw}{0.083\textwidth}  
%
%    % THE TABLE NUMBER ARE GENERATED BY A SCRIPT	
%	\begin{tabular}{l C{0.5\tblw} C{0.5\tblw} C{0.5\tblw} C{0.5\tblw} C{0.5\tblw} C{0.5\tblw} C{0.5\tblw} C{0.5\tblw} C{0.5\tblw} C{0.5\tblw}}
\toprule
& NN MAP & MFVI & BNN & GLM & GLM diag & GLM refine & GLM refine d & SVGP (quarter) & SVGP (half) & GP  \\
\midrule
\sc australian & \val{5}{2} & \val{5}{2} & \val{5}{2} & \val{5}{2} & \val{5}{2} & \val{5}{2} & \val{5}{2} & \val{5}{2} & \val{5}{2} & \val{5}{2} \\
\sc cancer & \val{5}{2} & \val{5}{2} & \val{5}{2} & \val{5}{2} & \val{5}{2} & \val{5}{2} & \val{5}{2} & \val{5}{2} & \val{5}{2} & \val{5}{2} \\
\sc ionosphere & \val{5}{2} & \val{5}{2} & \val{5}{2} & \val{5}{2} & \val{5}{2} & \val{5}{2} & \val{5}{2} & \val{5}{2} & \val{5}{2} & \val{5}{2} \\
\sc glass & \val{5}{2} & \val{5}{2} & \val{5}{2} & \val{5}{2} & \val{5}{2} & \val{5}{2} & \val{5}{2} & \val{5}{2} & \val{5}{2} & \val{5}{2} \\
\sc vehicle & \val{5}{2} & \val{5}{2} & \val{5}{2} & \val{5}{2} & \val{5}{2} & \val{5}{2} & \val{5}{2} & \val{5}{2} & \val{5}{2} & \val{5}{2} \\
\sc waveform & \val{5}{3} & \val{5}{3} & \val{5}{3} & \val{5}{3} & \val{5}{3} & \val{5}{3} & \val{5}{3} & \val{5}{3} & \val{5}{3} & \val{5}{3} \\
\sc digits & \val{5}{4} & \val{5}{4} & \val{5}{4} & \val{5}{4} & \val{5}{4} & \val{5}{4} & \val{5}{4} & \val{5}{4} & \val{5}{4} & \val{5}{4} \\
\sc satellite & \val{5}{5} & \val{5}{5} & \val{5}{5} & \val{5}{5} & \val{5}{5} & \val{5}{5} & \val{5}{5} & \val{5}{5} & \val{5}{5} & \val{5}{5} \\
\bottomrule
\end{tabular}

%\end{table}
\begin{table}[t!]
\resizebox{0.9\textwidth}{!}{
\begin{tabular}{llllllll}
\toprule
Model & BNN & GLM & GP Subset (GP) & GP Subset (NN) & NN MAP & SFR (GP) & SFR (NN) \\
Data set &  &  &  &  &  &  &  \\
\midrule
Glass & 0.9562 $\pm$ 0.2649 & 0.8966 $\pm$ 0.1424 & 1.5108 $\pm$ 0.0367 & 1.1420 $\pm$ 0.0738 & 1.8350 $\pm$ 0.6147 & 1.3709 $\pm$ 0.0375 & 1.0922 $\pm$ 0.1363 \\
Waveform & 0.2693 $\pm$ 0.0310 & 0.2676 $\pm$ 0.0227 & 0.2702 $\pm$ 0.0192 & 0.2670 $\pm$ 0.0162 & 0.2670 $\pm$ 0.0162 & 0.2657 $\pm$ 0.0163 & 0.2670 $\pm$ 0.0162 \\
\bottomrule
\end{tabular}
}
\end{table}
\alert{\bf UCI Results.} For now a nice placeholder table that should contain the comparisons and ablations on UCI data with negative log predictive density. %(NLPD\textcolor{gray}{\footnotesize$\pm$std}, lower better). Our sparse our ($M=256$) is on par with full models (left) and outperforms the GP subset approach of \citet{immer2021improving} (right). Results for methods marked with * as reported in the original benchmark~\citep{immer2021improving}. 
%\end{minipage}
  \begin{itemize}
     \item Item here
     \item Another item
  \end{itemize}

  \end{block}


%  \begin{block}{Discussion}
%
%  \begin{itemize}
%     \item Item here
%     \item Another item
%  \end{itemize}
%
%  \end{block}

  \vspace*{1em}

  \nocite{*} % <-- This lists all references that are in the bib file

  \begin{block}{References}
    \vspace*{-.25em}
    \footnotesize{\bibliographystyle{ieeetr}\bibliography{bibliography}}
  \end{block}

	\end{column}
\end{columns}
\end{frame}

\end{document}
