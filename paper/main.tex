\documentclass{article}

%\title{Investigatin Uncertainty Quantification in Model-based Reinforcement Learning}
\title{Model-based Reinforcement Learning with Fast Posterior Updates}
\author{%
  Aidan ~Scannell \\
  Aalto University\\
  %Finnish Center for Artificial Intelligence \\
  \texttt{aidan.scannell@aalto.fi}
  \And
  Paul ~Chang \\
  Aalto University\\
  \texttt{paul.chang@aalto.fi}
  \And
  Ella ~Tamir \\
  Aalto University\\
  \texttt{ella.tamir@aalto.fi}
  \And
  Arno ~Solin \\
  Aalto University\\
  \texttt{arno.solin@aalto.fi}
  \And
  Joni ~Pajarinen \\
  Aalto University\\
  \texttt{joni.pajarinen@aalto.fi}
  % examples of more authors
  % \And
  % Coauthor \\
  % Affiliation \\
  % Address \\
  % \texttt{email} \\
}


% NeurIPS packages
\usepackage[preprint,nonatbib]{neurips_2022}
\usepackage[utf8]{inputenc} % allow utf-8 input
\usepackage[T1]{fontenc}    % use 8-bit T1 fonts
%\usepackage{hyperref}       % hyperlinks
\usepackage{url}            % simple URL typesetting
\usepackage{booktabs}       % professional-quality tables
\usepackage{amsfonts}       % blackboard math symbols
\usepackage{nicefrac}       % compact symbols for 1/2, etc.
\usepackage{microtype}      % microtypography
\usepackage{xcolor}         % colors

% Bibliography
\usepackage[maxcitenames=1, maxbibnames=4, doi=false, isbn=false, eprint=true, backend=bibtex, hyperref=true, url=false, style=authoryear-comp]{biblatex}
\addbibresource{zotero-library.bib}
% \addbibresource{paper/zotero-library.bib}

% Our packages
\usepackage{todonotes}
\usepackage[colorlinks=true,linkcolor=blue,allcolors=blue]{hyperref}
\usepackage{amsmath}
\usepackage{bm}
\usepackage{algpseudocode}
\usepackage{algorithm}

\usepackage{tikz,pgfplots}
\usepackage{subcaption}
\usetikzlibrary{}

\newcommand{\defeq}{\vcentcolon=}

% Definitions/assumptions etc
\usepackage{mathtools}
\newtheorem{definition}{Definition}[section]
\newtheorem{assumption}{Assumption}[section]
\newtheorem{theorem}{Theorem}[section]
\newtheorem{lemma}{Lemma}[section]
% \newtheorem*{remark}{Remark}

% Short commands for commonly used stuff
\DeclareMathOperator{\R}{\mathbb{R}}
\DeclareMathOperator{\E}{\mathbb{E}}
\DeclareMathOperator{\V}{\mathbb{V}}


% Short section names etc
% This must be imported last!
%\usepackage{cleveref}
\usepackage[capitalise,nameinlink]{cleveref}
\crefname{section}{Sec.}{Secs.}
\crefname{algorithm}{Alg.}{Algs.}
\crefname{appendix}{App.}{Apps.}
\crefname{definition}{Def.}{Defs.}
\crefname{table}{Tab.}{Tabs}

% Config for Arno's awesome TikZ plotting stuff
\newlength{\figurewidth}
\newlength{\figureheight}


% Variables
\newcommand{\state}{\ensuremath{\mathbf{s}}}
\newcommand{\action}{\ensuremath{\mathbf{a}}}
\newcommand{\noise}{\ensuremath{\bm\epsilon}}
\newcommand{\discount}{\ensuremath{\gamma}}
\newcommand{\inducingVariable}{\ensuremath{\mathbf{u}}}
\newcommand{\dataset}{\ensuremath{\mathcal{D}}}
\newcommand{\dualParam}[1]{\ensuremath{\bm{\lambda}_{#1}}}
\newcommand{\meanParam}[1]{\ensuremath{\bm{\mu}_{#1}}}

% Indexes
\newcommand{\horizon}{\ensuremath{h}}
\newcommand{\Horizon}{\ensuremath{H}}
\newcommand{\numDataNew}{\ensuremath{N^{\text{new}}}}
\newcommand{\numDataOld}{\ensuremath{N^{\text{old}}}}
\newcommand{\numInducing}{\ensuremath{M}}

% Domains
\newcommand{\stateDomain}{\ensuremath{\mathcal{S}}}
\newcommand{\actionDomain}{\ensuremath{\mathcal{A}}}
\newcommand{\policyDomain}{\ensuremath{\Pi}}

% Functions
\newcommand{\rewardFn}{\ensuremath{r}}
\newcommand{\transitionFn}{\ensuremath{f}}
\newcommand{\latentFn}{\ensuremath{f}}

\newcommand{\optimisticTransition}{\ensuremath{\hat{f}}}
\newcommand{\optimisticTransitionMean}{\ensuremath{\mu_{\optimisticTransition}}}
\newcommand{\optimisticTransitionCov}{\ensuremath{\mu_{\optimisticTransition}}}
\newcommand{\optimisticTransitionSet}{\ensuremath{\mathcal{M}}}


% Parameters
\newcommand{\transitionParams}{\ensuremath{\phi}}
\newcommand{\valueFnParams}{\ensuremath{\psi}}
\newcommand{\policyParams}{\ensuremath{\theta}}

% Networks
\newcommand{\transition}{\ensuremath{\mathbf{d}_{\transitionParams}}}
\newcommand{\valueFn}{\ensuremath{\mathbf{Q}}}
\newcommand{\stateValueFn}{\ensuremath{\mathbf{V}}}
% \newcommand{\valueFn}{\ensuremath{\mathbf{Q}_{\valueFnParams}}}
\newcommand{\policy}{\ensuremath{\pi}}
\newcommand{\pPolicy}{\ensuremath{\pi_{\policyParams}}}

\begin{document}

\maketitle

\begin{abstract}
  % Reinforcement learning (RL) agents typically do not utilise the information previously gained during an episode.
  % as they perform learning offline, i.e. after an episode.
  % Reinforcement learning (RL) agents typically perform learning after an episode, ignoring the potential information gain whilst collecting data.
  Reinforcement learning (RL) agents typically perform learning offline and do not utilise the information gained during an episode.
  We present a Bayesian model-based RL algorithm which incorporates this information by updating a dynamic model's posterior during an episode.
  Our method leverages the fast posterior updates available in the dual parameterisation of sparse variational Gaussian processes.
  Importantly, our algorithm scales to 1) environment's with high dimensional state-spaces as it is built upon the connection between Bayesian neural networks (BNNs) and
  Gaussian processes (GPs) and 2) long episodes, as the complexity of our updates is independent of episode length.
  We demonstrate that our fast posterior updates improve sample efficiency in environments with action penalties, a notoriously difficult
  challenge for model-based RL algorithms.
\end{abstract}

\section{Introduction} \label{sec:intro}

Why model-based RL:
\begin{itemize}
  \item Better sample efficiency than model-free RL
\end{itemize}

Why Bayesian model-based RL:
\begin{itemize}
  \item Can be used to combat model bias
  \item Better exploration in environment's that are hard to explore (sparse reward/control penalties)
  \item Bayesian framwork can naturally handle stochastic environments
\end{itemize}

Issues with uncertainty in model-based RL:
\begin{itemize}
  \item Over exploration when using optimism in the face of uncertainty
  \begin{itemize}
    \item Agent drives exploration into region of high epistemic uncertainty because it thinks it could have high return.
    \item In reality, the dynamic model is innacurate and the agent spends the majority of an episode exploiting this innacuracy.
    \item What if the agent could update it's model during an episode and realise it's pointless to explore in this region
    \item We show that this can reduce model exploration and in turn, improve sample efficiency.
  \end{itemize}
\end{itemize}

\section{Problem Statement and Background} \label{sec:problem-statement}
We consider environments with states \(\state \in \stateDomain \), actions \(\action \in \actionDomain\) and transition dynamics \(\transitionFn: \stateDomain \times \actionDomain \rightarrow \stateDomain \) subject to
iid noise \(\noise_{t}\), given by,
\begin{align}
\state_{t+1} = \transitionFn(\state_{t}, \action_{t}) + \noise_{t}.
\end{align}
\begin{assumption}
  (system properties) The environment's dynamics are $L_{\transitionFn}\text{-Lipschitz}$ continuous and the transition noise $\noise_{t}$ is $\sigma\text{-sub-Gaussian}$ for all $t \geq 0$.
\end{assumption}

\subsection{Model-based Reinforcement Learning (RL)}
The goal of reinforcement learning is to find a policy \(\pi \in \Pi\) that maximises the sum of discounted
rewards in expecation under the transition noise (aleatoric uncertainty),
\begin{align} \label{eq-model-free-objective}
\policy^{*} = \arg \max_{\policy \in \policyDomain} J(\transitionFn, \policy) = \arg \max_{\policy \in \policyDomain} \mathbb{E}_{\noise_{0:\infty}} \left[ \sum_{t=0}^{\infty} \discount^{t} \rewardFn(\state_{t},\action_{t}) \right],
\end{align}
where $\gamma \in [0, 1]$

\textbf{Model-based}
In Bayesian model-based RL, we obtain the posterior over the dynamics \(p(f\mid\mathcal{D})\) after performing (approximate) Bayesian
inference given a state transition data set \(\mathcal{D} = \{\{(s_{t},a_{t}), s_{t+1}\}^{T_{i}}_{t=1}\}_{i=0}^{N}\).
\cref{alg-mbrl} shows the typical model-based RL loop.
Importantly, the dynamics are usually only updated after an episode.

\begin{algorithm}[!t]
\caption{Model-based RL}\label{alg-mbrl}
\begin{algorithmic}[1]
  \Require Start state $\state_{0}$, initial data set $\dataset_{0}$, dynamics posterior $p(\transitionFn \mid \dataset_{0})$, policy $\policy_{0}$
\For{$i  \in \{1, 2, \ldots, \text{num episodes} \}$}
    \State Reset the system to $\state_{0}$ and reset trajectory buffers $\bm\tau_{t} = \emptyset \ \forall t$
    \For{$t  \in \{1, 2, \ldots, \text{num steps} \}$}
      % \State Collect  $\tau_{0:t} = \tau_{0:t-1} \cup (\state_{j}, \action_{j}, \state_{j+1}, r_{j+1})$
      \State Use \cref{eq-greedy}/\cref{eq-posterior-sampling}/\cref{eq-hucrl} to collect data $\bm\tau_{t} = \bm\tau_{t-1} \cup (\state_{t}, \policy_{i}(\state_{t}), \transitionFn(\state_{t}, \policy_{i}(\state_{t})), r_{t+1})$
      % \State Execute policy $\policy_{i}(\state_{t})$ in environment and update trajectory $\tau_{i+1} = \{\state_{j}, \action_{j}, \state_{j+1}, r_{j+1}) \}_{j=0}^{t}$
    \EndFor
    \State Update data set $\dataset_{0:i} = \dataset_{0:i-1} \cup \tau$
    \State Train dynamics $p(\transitionFn \mid \dataset_{0:i}) \leftarrow \text{update\_dynamics}(\dataset_{0:i}, p(\transitionFn \mid \dataset_{0:i-1}))$
    % \State Train dynamics $p(\transitionFn \mid \dataset_{0:i+1})$ using $\dataset_{0:i+1}$
    % \State Improve policy $\pi_{i+1}$ using $p(\transitionFn \mid \dataset_{0:i+1})$ and/or $\dataset_{0:i+1}$
    \State Improve policy $\pi_{i+1} \leftarrow \text{update\_policy}(p\left(\transitionFn \mid \dataset_{0:i}), \dataset_{0:i} \right)$
    %\State Improve policy $\pi_{i+1}$ using $p(\transitionFn \mid \dataset_{0:i+1})$ and/or $\dataset_{0:i+1}$
\EndFor
\end{algorithmic}
\end{algorithm}

% \begin{minipage}{0.499\textwidth}
% \begin{algorithm}[H]
% \caption{Model-based RL}\label{alg-mbrl}
% \begin{algorithmic}[1]
%   \Require Initial data set $\dataset_{0}$, dynamics posterior $p(\transitionFn \mid \dataset_{0})$, policy $\policy_{0}$
% \For{$i  \in \{0, 1, \ldots, \text{num episodes} \}$}
%     \For{$t  \in \{0, 1, \ldots, \text{num steps} \}$}
%       \State Execute policy $\policy_{i}(\state_{t})$ in environment
%       \State $\tau_{i+1} = \{\state_{j}, \action_{j}, \state_{j+1}, r_{j+1}) \}_{j=0}^{t}$
%     \EndFor
%     \State Update data set $\mathcal{D}_{0:i+1} = \mathcal{D}_{0:i} \cup \tau_{i+1}$
%     \State Train dynamics $p(\transitionFn \mid \dataset_{0:i+1})$
%     \State Improve policy $\pi_{i+1}$
%     %\State Improve policy $\pi_{i+1}$ using $p(\transitionFn \mid \dataset_{0:i+1})$ and/or $\dataset_{0:i+1}$
% \EndFor
% \end{algorithmic}
% \end{algorithm}
% \end{minipage}
% \hfill
% \begin{minipage}{0.499\textwidth}
% \begin{algorithm}[H]
% \caption{Model-based RL with fast updates}\label{alg-mbrl-fast-updates}
% \begin{algorithmic}[1]
%   \Require Initial data set $\dataset_{0}$, dynamics posterior $p(\transitionFn \mid \dataset_{0})$, policy $\policy_{0}$
%     % ${p(\state_{\timeInd+1} \mid \singleInput, \dataset_{0})}$}
% \For{$i  \in \{0, 1, \ldots, \text{num episodes} \}$}
%     \For{$t  \in \{0, 1, \ldots, \text{num steps} \}$}
%       \State Execute policy $\policy_{i}(\state_{t})$ in environment
%       % \State Append transition $\state_{t}, \action_{t}, \state_{t+1}, r_{t+1})$ to trajectory $\tau_{i}$
%       \State $\tau_{i+1} = \{\state_{j}, \action_{j}, \state_{j+1}, r_{j+1}) \}_{j=0}^{t}$
%       \State {\color{blue}Update dynamics $p(\transitionFn \mid \dataset_{0:i} \cup \tau_{i+1})$}
%     \EndFor
%     \State Update data set $\mathcal{D}_{0:i+1} = \mathcal{D}_{0:i} \cup \tau_{i+1}$
%     \State Train dynamics $p(\transitionFn \mid \dataset_{0:i+1})$
%     \State Improve policy $\pi_{i+1}$
%     %\State Improve policy $\pi_{i+1}$ using $p(\transitionFn \mid \dataset_{0:i+1})$ and/or $\dataset_{0:i+1}$
% \EndFor
% \end{algorithmic}
% \end{algorithm}
% \end{minipage}



\subsection{Exploration Strategies}
\textbf{Greedy exploitation}
Given the posterior dynamics \(p(\transitionFn \mid \mathcal{D})\),
a common approach is to simply take the expecation over both the aleatoric and epistemic uncertainty,
\begin{align} \label{eq-greedy}
\policy_{i+1}^{\text{greedy}} = \arg \max_{\policy \in \policyDomain} \mathbb{E}_{\transitionFn \sim p(\transitionFn \mid \dataset_{0:i})} \left[ J(\transitionFn, \policy) \right],
\end{align}
This approach has been widely adopted, for example, in PILCO, PETS, GP-MPC
\cite{deisenrothPILCO2011,chuaDeepReinforcementLearning2018,kamtheDataEfficient2018}.
This approach helps to alleviate model bias as the posterior ``knows what the model does not know''.
This is because the predictive posterior \(p(f(s_{t},a_{t}) \mid (s_{t},a_{t}),  \mathcal{D} )\) will be (or should be) uncertain when making
predictions far away from the training data.
The expectation considers all possible dynamics models which prevents the policy optimisation from
exploiting innacuracies in the model.
This approach has no guarantees for exploration in the general case.
However, under specific dynamics and reward structures (e.g. PILCO) this objective can achieve sublinear regret.
\todo{need to double check sublinear regret statement. And give a reference}


\textbf{Posterior sampling}
\cite{osbandWhyPosteriorSampling2017,osbandMoreEfficientReinforcement2013}
\begin{align} \label{eq-posterior-sampling}
\policy_{i+1}^{\text{PS}} = \arg \max_{\policy \in \policyDomain} \left[ J(\transitionFn, \policy) \right] \quad \text{s.t. } \transitionFn \sim p(\transitionFn \mid \dataset_{0:i})
\end{align}

\textbf{Hallucinated upper confidence RL}
A more theoretically grounded exploration strategy is UCRL \autocite{jakschNearoptimal2010}, which optimises joinly over
policies and models inside the set
\(\mathcal{M} = \{ f \mid | f(s,a) - \mu_{i}(s, a) | \leq \beta_{i} \Sigma_{i}(s, a) \quad \forall s, a \in \mathcal{S} \times \mathcal{A} \}\), representing all statistically plausible
models under the posterior \(p(f(s,a) \mid \mathcal{D}_{0:i} \cup (s,a)) = \mathcal{N}(f(s,a) \mid \mu_{i}(s,a), \Sigma_{i}(s,a))\) at episode \(i\).
This strategy is given by,
\begin{align}
\policy_{i+1}^{\text{UCRL}} = \arg \max_{\policy \in \policyDomain} \max_{\transitionFn \in \mathcal{M}} J(\transitionFn, \policy).
\end{align}
This strategy optimises an optimistic policy over the set of plausible dynamics models.
Although this joint optimisation is intractable in general,
\cite{curiEfficient2020} proposed a practical alternative which is detailed in \cref{sec-hucrl}.

\textbf{MPC vs policy learning}
It is worth noting that the strategies in \cref{eq-greedy,eq-posterior-sampling,eq-hucrl} can be used with both model predictive control (MPC)
techniques, such as the cross entoropy method (CEM), and model-free RL techniques, such as soft actor-critic (SAC).


In this work we are interested in how we can use \(p(f \mid \mathcal{D})\) to alleviate some of the issues in model-based RL,
for example, model bias and the exploration-exploitation trade-off.



\todo{show how to get new $\dualParam{1}$ and $\dualParam{2}$ in Train dynamics line of \cref{alg-mbrl-fast-updates}}
\begin{algorithm}[!t]
\caption{Model-based RL with fast updates}\label{alg-mbrl-fast-updates}
\begin{algorithmic}[1]
  \Require Start state $\state_{0}$, initial data set $\dataset_{0}$, dynamics posterior $p(\transitionFn \mid \dataset_{0})$ (inc. dual parameters $\dualParam{1}, \dualParam{2}$), policy $\policy_{0}$
    % ${p(\state_{\timeInd+1} \mid \singleInput, \dataset_{0})}$}
\For{$i  \in \{1, 2, \ldots, \text{num episodes} \}$}
    \State Reset the system to $\state_{0}$ and reset trajectory buffers $\bm\tau_{t} = \emptyset \ \forall t$
    \For{$t  \in \{1, 2, \ldots, \text{num steps} \}$}
      % \State Execute policy $\policy_{i}(\state_{t})$ (\cref{eq-fast-update-mpc}) in environment
      \State Use \cref{eq-fast-update-mpc} to collect data $\bm\tau_{t} = \bm\tau_{t-1} \cup (\state_{t}, \policy_{i}(\state_{t}), \transitionFn(\state_{t}, \policy^{\text{fast}}_{i}(\state_{t})), r_{t+1})$
      % \State Append transition $\state_{t}, \action_{t}, \state_{t+1}, r_{t+1})$ to trajectory $\tau_{i}$
      % \State $\tau_{i+1} = \{\state_{j}, \action_{j}, \state_{j+1}, r_{j+1}) \}_{j=0}^{t}$
      %\State {\color{blue}Update dynamics $p(\transitionFn \mid \dataset_{0:i} \cup \tau_{i+1})$}
      \State {\color{blue}Update dynamics posterior using \cref{eq-dual-update-svgp}, i.e. fast update}
      % \begin{align}
      % \dualParam{1}^{t+1} &\leftarrow \dualParam{1}^{t} +  \nabla_{\meanParam{1}} \mathbb{E}_{q_{\inducingVariable}(\latentFn(\state_{t}, \action_{t}))} \left[ \log p(\state_{t+1} \mid \latentFn(\state_{t}, \action_{t}) ) \right] \\
      % \dualParam{2}^{t+1} &\leftarrow \dualParam{2}^{t} +  \nabla_{\meanParam{2}} \mathbb{E}_{q_{\inducingVariable}(\latentFn(\state_{t}, \action_{t}))}  \left[ \log p(\state_{t+1} \mid \latentFn(\state_{t}, \action_{t}) ) \right]
      % \end{align}}
    \EndFor
    \State Update data set $\dataset_{0:i} = \dataset_{0:i-1} \cup \tau$
    \State Train dynamics $p(\transitionFn \mid \dataset_{0:i}) \leftarrow \text{update\_dynamics}(\dataset_{0:i}, p(\transitionFn \mid \dataset_{0:i-1}))$
    \State Improve policy $\pi^{\text{fast}}_{i+1} \leftarrow \text{update\_policy}(p\left(\transitionFn \mid \dataset_{0:i}), \dataset_{0:i} \right)$
\EndFor
\end{algorithmic}
\end{algorithm}


\section{Model-based Reinforcement Learning with Fast Updates}
The strategies in \cref{eq-greedy,eq-posterior-sampling,eq-hucrl} do not update the dynamic model during an episode.
A better approach would be to update the posterior at every time step during an episode, for example,
\begin{subequations}
\begin{align} \label{eq-fast-update-mpc}
  \policy_{i+1}^{\text{greedy}}(\state) &= \arg \max_{\action_{0}} \max_{\action_{1:\Horizon}}
\E_{p(\transitionFn \mid \dataset_{0:i})} \left[J^{\Horizon}(\policy, \transitionFn) \right] + \stateValueFn(\state_{\Horizon+1}) \\
  \policy_{i+1}^{\text{PS}}(\state) &= \arg \max_{\action_{0}} \max_{\action_{1:\Horizon}}
J^{\Horizon}(\policy, \transitionFn) + \stateValueFn(\state_{\Horizon+1}) \quad \text{s.t. } \transitionFn \sim p(\transitionFn \mid \dataset_{0:i} \cup \bm\tau_{t}) \\
  \policy_{i+1}^{\text{UCRL}}(\state) &= \arg \max_{\action_{0}} \max_{\action_{1:\Horizon}}
J^{\Horizon}(\policy, \transitionFn) + \stateValueFn(\state_{\Horizon+1}) \\
  \stateValueFn(\state) &= \mathbb{E} \left[ \sum_{t=0}^{\infty}     \discount^{t} \rewardFn(\state_{t},\action_{t}) \mid \state_{0}=\state \right] \label{eq-value-fn}
\end{align}
\end{subequations}
\begin{subequations}
\begin{align} \label{eq-fast-update-mpc-old}
  \policy^{\text{fast}}(\state) = \arg &\max_{\action_{0}} \max_{\action_{1}, \ldots, \action_{\Horizon}}
  \mathbb{E}_{\state_{\horizon} \sim p(\state_{\horizon+1} \mid \transitionFn(\state_{\horizon}, \action_{\horizon}))} \left[ \sum_{\horizon=0}^{\Horizon}     \discount^{\horizon} \rewardFn(\state_{\horizon},\action_{\horizon}) \mid \state_{0}=\state \right] + \discount^{\Horizon+1} \stateValueFn(\state_{\Horizon+1}) \\
  \stateValueFn(\state) &= \mathbb{E} \left[ \sum_{t=0}^{\infty}     \discount^{t} \rewardFn(\state_{t},\action_{t}) \mid \state_{0}=\state \right] \label{eq-fast-update-mpc}
\end{align}
\end{subequations}

\begin{align} \label{}
  \policy(\state) = \arg &\max_{\action_{0}} \max_{\action_{1}, \ldots, \action_{\Horizon}} \max_{\optimisticTransition \in \optimisticTransitionSet}
  \sum_{\horizon=0}^{\Horizon}  \mathbb{E}_{\noise_{\horizon}} \left[  \discount^{\horizon} \rewardFn(\state_{\horizon},\action_{\horizon}) \right] + \discount^{\Horizon+1} \stateValueFn(\state_{\Horizon+1}) \\
  \text{s.t. } \state_{\horizon+1} &= \optimisticTransition(\state_{\horizon}, \action_{\horizon}) + \noise_{\horizon} \\
  \optimisticTransition(\state_{\horizon}, \action_{\horizon}) &=
\optimisticTransitionMean(\state_{\horizon}, \action_{\horizon}) \pm \beta_{i}
\optimisticTransitionCov(\state_{\horizon}, \action_{\horizon})
\end{align}

\subsection{Dual GPs for Fast Updates}
Sparse GP predictive posterior is given by,
\begin{align} \label{eq-svgp-predictive-posterior}
  q_{\inducingVariable}(\transitionFn(\state_{t}, \action_{t})) = \mathcal{N}
  \left( \transitionFn(\state_{t}, \action_{t}) \mid \mathbf{A} \mathbf{m}^{*}, \mathbf{A}\mathbf{K}_{\mathbf{z}\mathbf{z}}^{-1} \mathbf{A}^{T} + \mathbf{A} \mathbf{V}^{*} \mathbf{A}^{T} \right)
\end{align}
where $\mathbf{A} = \mathbf{K}_{\mathbf{X}\mathbf{Z}} \mathbf{K}^{-1}_{\mathbf{z}\mathbf{z}}$.
\cite{adamDualParameterizationSparse2021} showed that optimal variational parameters are given by,
\begin{align} \label{eq-dual-params}
\mathbf{m}^{*} = \mathbf{V}^{*}\dualParam{1}^{*} \quad \mathbf{V}^{*} = [\mathbf{K}_{\mathbf{z}\mathbf{z}}^{-1} + \dualParam{2}^{*}]
\end{align}
Importantly, \cite{changFantasizingDualGPs2022} show that in the dual space,
conditioning on new observations reduces to suming the dual variables from the previous time
step with an update, given by,
\todo{pick best way to show new data (using superscript new or just time indexing?)}
\begin{align} \label{eq-dual-update-svgp}
\dualParam{1}^{t+1} &\leftarrow \dualParam{1}^{t} +  \nabla_{\meanParam{1}} \mathbb{E}_{q_{\inducingVariable}(\latentFn(\state_{t}, \action_{t}))} \left[ \log p(\state_{t+1} \mid \latentFn(\state_{t}, \action_{t}) ) \right] \\
\dualParam{2}^{t+1} &\leftarrow \dualParam{2}^{t} +  \nabla_{\meanParam{2}} \mathbb{E}_{q_{\inducingVariable}(\latentFn(\state_{t}, \action_{t}))}  \left[ \log p(\state_{t+1} \mid \latentFn(\state_{t}, \action_{t}) ) \right]
\end{align}
Importantly, for a single new observation $((\state_{t}, \action_{t}), \state_{t+1})$ this update has
complexity $\mathcal{O}(\numInducing^{2})$.
This is a significant improvement to naive GP conditioning, which has complexity $\mathcal{O}((\numDataNew + \numDataOld)^{3})$
and sparse GP conditioning, which has complexity $\mathcal{O}((\numDataNew + \numDataOld)\numInducing^{2})$.
\todo{double check these complexities are right and cite them/show equaitons}
It is worth highlighting that the complexity of Paul et als update does not increase during an episode,

% \begin{align} \label{eq-dual-update-svgp}
%  \dualParam{1}^{\text{new}} &\leftarrow \dualParam{1}^{\text{old}} +
%   \nabla_{\meanParam{1}} \mathbb{E}_{q_{\inducingVariable}(\latentFn(\state_t^{\text{new}}, \action_t^{\text{new}}))}
%  \left[ \log p(\state_{t+1}^{\text{new}} \mid \latentFn(\state_t^{\text{new}}, \action_t^{\text{new}}) ) \right] \\
%  \dualParam{2}^{\text{new}} &\leftarrow \dualParam{2}^{\text{old}} +
%   \nabla_{\meanParam{2}} \mathbb{E}_{q_{\inducingVariable}(\latentFn(\state_t^{\text{new}}, \action_t^{\text{new}}))}
%  \left[ \log p(\state_t^{\horizon+1} \mid \latentFn(\state_t^{\text{new}}, \action_t^{\text{new}}) ) \right] \\
%  \dualParam{1}^{\horizon+1} &\leftarrow \dualParam{1}^{\horizon} +
%   \nabla_{\meanParam{1}} \mathbb{E}_{q_{\inducingVariable}(\latentFn(\state_{\horizon}, \action_{\horizon}))}
%  \left[ \log p(\state_{\horizon+1} \mid \latentFn(\state_{\horizon}, \action_{\horizon}) ) \right] \\
%  \dualParam{2}^{\horizon+1} &\leftarrow \dualParam{2}^{\horizon} +
%   \nabla_{\meanParam{2}} \mathbb{E}_{q_{\inducingVariable}(\latentFn(\state_{\horizon}, \action_{\horizon}))}
%  \left[ \log p(\state_{\horizon+1} \mid \latentFn(\state_{\horizon}, \action_{\horizon}) ) \right]
% \end{align}


\subsection{BNN to SVGP}
\begin{itemize}
  \item Write up Laplace BNN to SVGP
  \item Extend fast updates to SVGP
\end{itemize}

\begin{assumption} \label{assumption-ntk-linearisation}
  Something about NTK being a linearisation around $\theta^{MAP}$ but each update moves away from $\theta^{MAP}$
\end{assumption}

\subsection{Efficiently Sampling Functions for Posterior Sampling}
\cite{wilsonEfficiently2020}


\section{Experiments}
Environments
\begin{itemize}
  \item Compare updated posterior with non updated posterior
  \item Need to show update is fast enough for practical use?
  \begin{itemize}
    \item How frequently can we do the posterior update? Every time step or every $K$ steps?
    \item Update depends on number of inducing points?
    \item What experiments can we show for this? Wall clock time vs planning horizon with different numbers of inducing points?
  \end{itemize}
  \item How does number of inducing points affect performance?
  \begin{itemize}
    \item More inducing points means better approximation but larger computational complexity.
  \end{itemize}
  \item How good is the assumption that we can linearise BNN to get NTK?
  \begin{itemize}
    \item Does model's performance get worse during an episode because linearisation becomes more wrong?
    \item SVGP does't have this issue so compare to SVGP in cartpole?
  \end{itemize}
  \item What length planning horizon to use?
  \begin{itemize}
    \item Too small (e.g. $\Horizon=0$) would mean no exploration?
    \item Can $\Horizon$ be too large?
  \end{itemize}
  \begin{itemize}
  \item Compare for different strategies of making decisions under dynamic model's uncertainty:
    \begin{itemize}
      \item \textbf{Greedy exploitation} \(\pi_{\text{greedy}}\)
      \item \textbf{Hallucinated-UCRL} \(\pi_{\text{HUCRL}}\)
      \item \textbf{Thompson sampling} \(\pi_{\text{TS}}\)
    \end{itemize}
  \end{itemize}
  \item Compare impact of stationary vs non-stationary priors
  \begin{itemize}
    \item First do this with GP
    \item Then try to do with Laplace BNN
  \end{itemize}
  \item Compare impact of function vs weight space
\end{itemize}

\section{Conclusion} \label{sec:conclusion}


\section*{Broader Impact}

\section*{Acknowledgements}
Aidan Scannell is funded by the Finnish Center for Artificial Intelligence.

% \section*{References}
\small
\printbibliography
\normalsize
% TODO make bibliography small a better way

References follow the acknowledgments. Use unnumbered first-level heading for
the references. Any choice of citation style is acceptable as long as you are
consistent. It is permissible to reduce the font size to \verb+small+ (9 point)
when listing the references.
Note that the Reference section does not count towards the page limit.
\medskip



\appendix

\section{Appendix}

Optionally include extra information (complete proofs, additional experiments and plots) in the appendix.
This section will often be part of the supplemental material.

\section{Hallucinated Upper Confidence Reinforcement Learning (H-UCRL)} \label{sec-hucrl}
\cite{curiEfficient2020} introduced a tractable approximation which retains some of the theoretical guarantees whilst
being applicable with deep model-based RL.
They introduce a function \(\eta: \mathcal{S} \times \mathcal{A} \rightarrow [-1, 1]^{p}\) which acts as a hallucinated control input.
The strategy is given by,
\begin{align} \label{eq-hucrl}
\pi_i^{\text{UCRL}} = \arg \max_{\policy \in \policyDomain} \max_{\eta(\cdot) \in [-1,1]} J(\transitionFn, \policy) \quad \text{s.t.} \quad \transitionFn = \mu_{i}(\state_{t}, \action_{t}) + \beta_{i} \Sigma_{i}(\state_{t}, \action_{t}) \eta(\state_{t},\action_{t}).
\end{align}
Intuitively, \(\eta(\state,\action) \in [-1,1]\) enables the optimisation to select any dynamics model
\(\transitionFn\) within \(\pm \beta \Sigma_{i}(\state_{t}, \action_{t})\) of the posterior mean \(\mu_{i}(\state_{t}, \action_{t})\).



\section{Template stuff}
\subsection{Generate TikZ Figures from Python}
We can generate figures in \texttt{.tex} format directly from Python:
\begin{verbatim}
tikzplotlib.save("fig.tex", axis_width="\\figurewidth", axis_height="\\figureheight")
\end{verbatim}
\cref{fig:example} shows that we get nicely formatted lables/titles/etc when we include them in our paper.
\begin{figure}[h]
    \centering\footnotesize

    % Set your figure size here
    \setlength{\figurewidth}{.33\textwidth}
    \setlength{\figureheight}{.75\figurewidth}

    % Customize your plot here
    % (scale only axis applies the size to the axis box and not entire figure)
    \pgfplotsset{grid style={dotted},title={Foo},scale only axis}

    % Use the subcaption package (= subfigure) for sub-plots, that is
    % plot the separate plots separately in Python
    \begin{subfigure}{.4\textwidth}
        \centering
        % This file was created with tikzplotlib v0.10.1.
\begin{tikzpicture}

\definecolor{darkgray176}{RGB}{176,176,176}
\definecolor{steelblue31119180}{RGB}{31,119,180}

\begin{axis}[
height=\figureheight,
tick align=outside,
tick pos=left,
width=\figurewidth,
x grid style={darkgray176},
xmin=-0.05, xmax=1.05,
xtick style={color=black},
y grid style={darkgray176},
ymin=-0.0420735492403948, ymax=0.883544534048291,
ytick style={color=black}
]
\addplot [semithick, steelblue31119180]
table {%
0 0
0.111111111111111 0.110882628509953
0.222222222222222 0.220397743456122
0.333333333333333 0.327194696796152
0.444444444444444 0.429956363528356
0.555555555555556 0.527415385771866
0.666666666666667 0.618369803069737
0.777777777777778 0.701697876146735
0.888888888888889 0.77637192130066
1 0.841470984807897
};
\end{axis}

\end{tikzpicture}

    \end{subfigure}
    \hfill
    \begin{subfigure}{.4\textwidth}
        \centering
        % This file was created with tikzplotlib v0.10.1.
\begin{tikzpicture}

\definecolor{darkgray176}{RGB}{176,176,176}
\definecolor{steelblue31119180}{RGB}{31,119,180}

\begin{axis}[
height=\figureheight,
tick align=outside,
tick pos=left,
width=\figurewidth,
x grid style={darkgray176},
xmin=-0.05, xmax=1.05,
xtick style={color=black},
y grid style={darkgray176},
ymin=-0.0420735492403948, ymax=0.883544534048291,
ytick style={color=black}
]
\addplot [semithick, steelblue31119180]
table {%
0 0
0.111111111111111 0.110882628509953
0.222222222222222 0.220397743456122
0.333333333333333 0.327194696796152
0.444444444444444 0.429956363528356
0.555555555555556 0.527415385771866
0.666666666666667 0.618369803069737
0.777777777777778 0.701697876146735
0.888888888888889 0.77637192130066
1 0.841470984807897
};
\end{axis}

\end{tikzpicture}

    \end{subfigure}
    \caption{Foo}
    \label{fig:example}
\end{figure}

\subsection{Generate Tables from Python}
We can also generate tables straight from python using \href{https://github.com/astanin/python-tabulate}{tabulate}:
\begin{verbatim}
table = [["Sun",696000,1989100000],["Earth",6371,5973.6],
        ["Moon",1737,73.5],["Mars",3390,641.85]]
headers = ["Planet","R (km)", "mass (x 10^29 kg)"]
table = tabulate(table, headers=headers, tablefmt="latex")
with open("table.tex", 'w') as file:
    file.write(table)
\end{verbatim}

\begin{table}[h]
    \centering
    \begin{tabular}{lrr}
\hline
 Planet   &   R (km) &   mass (x 10\^{}29 kg) \\
\hline
 Sun      &   696000 &          1.9891e+09 \\
 Earth    &     6371 &       5973.6        \\
 Moon     &     1737 &         73.5        \\
 Mars     &     3390 &        641.85       \\
\hline
\end{tabular}
\end{table}

\subsection{Biblatex}
Rember when using biblatex to use 'parencite' for \parencite{kamtheDataEfficient2018} and when using natbib to use 'citep'.


\end{document}
