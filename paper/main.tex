\documentclass{article}

\title{INSERT AWESOME TITLE}
\author{%
  Aidan ~Scannell \\
  Aalto University\\
  Finnish Center for Artificial Intelligence \\
  \texttt{aidan.scannell@aalto.fi} \\
  % examples of more authors
  % \And
  % Coauthor \\
  % Affiliation \\
  % Address \\
  % \texttt{email} \\
}


% NeurIPS packages
\usepackage[preprint,nonatbib]{neurips_2022}
\usepackage[utf8]{inputenc} % allow utf-8 input
\usepackage[T1]{fontenc}    % use 8-bit T1 fonts
%\usepackage{hyperref}       % hyperlinks
\usepackage{url}            % simple URL typesetting
\usepackage{booktabs}       % professional-quality tables
\usepackage{amsfonts}       % blackboard math symbols
\usepackage{nicefrac}       % compact symbols for 1/2, etc.
\usepackage{microtype}      % microtypography
\usepackage{xcolor}         % colors

% Bibliography
\usepackage[maxcitenames=1, maxbibnames=4, doi=false, isbn=false, eprint=true, backend=bibtex, hyperref=true, url=false, style=authoryear-comp]{biblatex}
\addbibresource{zotero-library.bib}

% Our packages
\usepackage{todonotes}
\usepackage[colorlinks=true,linkcolor=blue,allcolors=blue]{hyperref}

\usepackage{tikz,pgfplots}
\usepackage{subcaption}
\usetikzlibrary{}

% This must be imported last!
%\usepackage{cleveref}
\usepackage[capitalise,nameinlink]{cleveref}

% Config for Arno's awesome TikZ plotting stuff
\newlength{\figurewidth}
\newlength{\figureheight}

\begin{document}

\maketitle

\section{Introduction} \label{sec:intro}

\subsection{Generate TikZ Figures from Python}
We can generate figures in \texttt{.tex} format directly from Python:
\begin{verbatim}
tikzplotlib.save("fig.tex", axis_width="\\figurewidth", axis_height="\\figureheight")
\end{verbatim}
\cref{fig:example} shows that we get nicely formatted lables/titles/etc when we include them in our paper.
\begin{figure}[h]
    \centering\footnotesize

    % Set your figure size here
    \setlength{\figurewidth}{.33\textwidth}
    \setlength{\figureheight}{.75\figurewidth}

    % Customize your plot here
    % (scale only axis applies the size to the axis box and not entire figure)
    \pgfplotsset{grid style={dotted},title={Foo},scale only axis}

    % Use the subcaption package (= subfigure) for sub-plots, that is
    % plot the separate plots separately in Python
    \begin{subfigure}{.4\textwidth}
        \centering
        % This file was created with tikzplotlib v0.10.1.
\begin{tikzpicture}

\definecolor{darkgray176}{RGB}{176,176,176}
\definecolor{steelblue31119180}{RGB}{31,119,180}

\begin{axis}[
height=\figureheight,
tick align=outside,
tick pos=left,
width=\figurewidth,
x grid style={darkgray176},
xmin=-0.05, xmax=1.05,
xtick style={color=black},
y grid style={darkgray176},
ymin=-0.0420735492403948, ymax=0.883544534048291,
ytick style={color=black}
]
\addplot [semithick, steelblue31119180]
table {%
0 0
0.111111111111111 0.110882628509953
0.222222222222222 0.220397743456122
0.333333333333333 0.327194696796152
0.444444444444444 0.429956363528356
0.555555555555556 0.527415385771866
0.666666666666667 0.618369803069737
0.777777777777778 0.701697876146735
0.888888888888889 0.77637192130066
1 0.841470984807897
};
\end{axis}

\end{tikzpicture}

    \end{subfigure}
    \hfill
    \begin{subfigure}{.4\textwidth}
        \centering
        % This file was created with tikzplotlib v0.10.1.
\begin{tikzpicture}

\definecolor{darkgray176}{RGB}{176,176,176}
\definecolor{steelblue31119180}{RGB}{31,119,180}

\begin{axis}[
height=\figureheight,
tick align=outside,
tick pos=left,
width=\figurewidth,
x grid style={darkgray176},
xmin=-0.05, xmax=1.05,
xtick style={color=black},
y grid style={darkgray176},
ymin=-0.0420735492403948, ymax=0.883544534048291,
ytick style={color=black}
]
\addplot [semithick, steelblue31119180]
table {%
0 0
0.111111111111111 0.110882628509953
0.222222222222222 0.220397743456122
0.333333333333333 0.327194696796152
0.444444444444444 0.429956363528356
0.555555555555556 0.527415385771866
0.666666666666667 0.618369803069737
0.777777777777778 0.701697876146735
0.888888888888889 0.77637192130066
1 0.841470984807897
};
\end{axis}

\end{tikzpicture}

    \end{subfigure}
    \caption{Foo}
    \label{fig:example}
\end{figure}

\subsection{Generate Tables from Python}
We can also generate tables straight from python using \href{https://github.com/astanin/python-tabulate}{tabulate}:
\begin{verbatim}
table = [["Sun",696000,1989100000],["Earth",6371,5973.6],
        ["Moon",1737,73.5],["Mars",3390,641.85]]
headers = ["Planet","R (km)", "mass (x 10^29 kg)"]
table = tabulate(table, headers=headers, tablefmt="latex")
with open("table.tex", 'w') as file:
    file.write(table)
\end{verbatim}

\begin{table}[h]
    \centering
    \begin{tabular}{lrr}
\hline
 Planet   &   R (km) &   mass (x 10\^{}29 kg) \\
\hline
 Sun      &   696000 &          1.9891e+09 \\
 Earth    &     6371 &       5973.6        \\
 Moon     &     1737 &         73.5        \\
 Mars     &     3390 &        641.85       \\
\hline
\end{tabular}
\end{table}

\subsection{Biblatex}
Rember when using biblatex to use 'parencite' for \parencite{kamtheDataEfficient2018} and when using natbib to use 'citep'.


\section*{Acknowledgements}
Based on Arno Solin's python/TikZ setup.

% \section*{References}
\small
\printbibliography
\normalsize
% TODO make bibliography small a better way

References follow the acknowledgments. Use unnumbered first-level heading for
the references. Any choice of citation style is acceptable as long as you are
consistent. It is permissible to reduce the font size to \verb+small+ (9 point)
when listing the references.
Note that the Reference section does not count towards the page limit.
\medskip



\appendix

\section{Appendix}

Optionally include extra information (complete proofs, additional experiments and plots) in the appendix.
This section will often be part of the supplemental material.


\end{document}
