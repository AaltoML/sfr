\documentclass{article}

%\title{Investigatin Uncertainty Quantification in Model-based Reinforcement Learning}
% \title{Model-based Reinforcement Learning with Fast Posterior Updates}
%\title{Sequential Decision-Making under Uncertainty with Big Data}
\title{Neural Network to Sparse Gaussian Process: For Online Exploration}

\author{%
  Aidan ~Scannell \\
  Aalto University\\
  %Finnish Center for Artificial Intelligence \\
  \texttt{aidan.scannell@aalto.fi}
  \And
  Paul ~Chang \\
  Aalto University\\
  \texttt{paul.chang@aalto.fi}
  \And
  Ella ~Tamir \\
  Aalto University\\
  \texttt{ella.tamir@aalto.fi}
  \And
  Arno ~Solin \\
  Aalto University\\
  \texttt{arno.solin@aalto.fi}
  \And
  Joni ~Pajarinen \\
  Aalto University\\
  \texttt{joni.pajarinen@aalto.fi}
  % examples of more authors
  % \And
  % Coauthor \\
  % Affiliation \\
  % Address \\
  % \texttt{email} \\
}

% Pass options to natbib
\PassOptionsToPackage{numbers, compress}{natbib}

% NeurIPS packages
\usepackage[]{neurips_2023}
\usepackage[utf8]{inputenc} % allow utf-8 input
\usepackage[T1]{fontenc}    % use 8-bit T1 fonts
%\usepackage{hyperref}       % hyperlinks
\usepackage{url}            % simple URL typesetting
\usepackage{booktabs}       % professional-quality tables
\usepackage{amsfonts}       % blackboard math symbols
\usepackage{nicefrac}       % compact symbols for 1/2, etc.
\usepackage{microtype}      % microtypography
\usepackage{xcolor}         % colors

% Bibliography
%\usepackage[maxcitenames=1, maxbibnames=4, doi=false, isbn=false, eprint=true, backend=bibtex, hyperref=true, url=false, style=authoryear-comp]{biblatex}
%\addbibresource{zotero-library.bib}
% \addbibresource{paper/zotero-library.bib}

% Let's use good old bibtex instead


% Our packages
\usepackage{todonotes}
\usepackage[colorlinks=true,linkcolor=blue,allcolors=blue]{hyperref}
\usepackage{amsmath}
\usepackage{bm}
\usepackage{algpseudocode}
\usepackage{algorithm}
\usepackage{derivative}

\usepackage{tikz,pgfplots}
\usepackage{subcaption}
\usetikzlibrary{}

\newcommand{\defeq}{\vcentcolon=}

% Definitions/assumptions etc
\usepackage{mathtools}
\newtheorem{definition}{Definition}[section]
\newtheorem{assumption}{Assumption}[section]
\newtheorem{theorem}{Theorem}[section]
\newtheorem{lemma}{Lemma}[section]
% \newtheorem*{remark}{Remark}

% Short commands for commonly used stuff
\DeclareMathOperator{\R}{\mathbb{R}}
\DeclareMathOperator{\E}{\mathbb{E}}
\DeclareMathOperator{\V}{\mathbb{V}}


% Short section names etc
% This must be imported last!
%\usepackage{cleveref}
\usepackage[capitalise,nameinlink]{cleveref}
\crefname{section}{Sec.}{Secs.}
\crefname{algorithm}{Alg.}{Algs.}
\crefname{appendix}{App.}{Apps.}
\crefname{definition}{Def.}{Defs.}
\crefname{table}{Tab.}{Tabs}

% Config for Arno's awesome TikZ plotting stuff
\newlength{\figurewidth}
\newlength{\figureheight}


% Variables
\newcommand{\state}{\ensuremath{\mathbf{s}}}
\newcommand{\action}{\ensuremath{\mathbf{a}}}
\newcommand{\noise}{\ensuremath{\bm\epsilon}}
\newcommand{\discount}{\ensuremath{\gamma}}
\newcommand{\inducingInput}{\ensuremath{\mathbf{Z}}}
\newcommand{\inducingVariable}{\ensuremath{\mathbf{u}}}
\newcommand{\dataset}{\ensuremath{\mathcal{D}}}
\newcommand{\dualParam}[1]{\ensuremath{\bm{\lambda}_{#1}}}
\newcommand{\meanParam}[1]{\ensuremath{\bm{\mu}_{#1}}}

% Indexes
\newcommand{\horizon}{\ensuremath{h}}
\newcommand{\Horizon}{\ensuremath{H}}
\newcommand{\numDataNew}{\ensuremath{N^{\text{new}}}}
\newcommand{\numDataOld}{\ensuremath{N^{\text{old}}}}
\newcommand{\numInducing}{\ensuremath{M}}

% Domains
\newcommand{\stateDomain}{\ensuremath{\mathcal{S}}}
\newcommand{\actionDomain}{\ensuremath{\mathcal{A}}}
\newcommand{\inputDomain}{\ensuremath{\mathbb{R}^{D}}}
\newcommand{\outputDomain}{\ensuremath{\mathbb{R}^{C}}}
\newcommand{\policyDomain}{\ensuremath{\Pi}}

% Functions
\newcommand{\rewardFn}{\ensuremath{r}}
\newcommand{\transitionFn}{\ensuremath{f}}
\newcommand{\latentFn}{\ensuremath{f}}

\newcommand{\optimisticTransition}{\ensuremath{\hat{f}}}
\newcommand{\optimisticTransitionMean}{\ensuremath{\mu_{\optimisticTransition}}}
\newcommand{\optimisticTransitionCov}{\ensuremath{\mu_{\optimisticTransition}}}
\newcommand{\optimisticTransitionSet}{\ensuremath{\mathcal{M}}}


% Parameters
% \newcommand{\transitionParams}{\ensuremath{\bm\phi}}
\newcommand{\transitionParams}{\ensuremath{\mathbf{w}}}
\newcommand{\valueFnParams}{\ensuremath{\psi}}
\newcommand{\policyParams}{\ensuremath{\theta}}

% Networks
\newcommand{\transitionFnWithParams}{\ensuremath{\transitionFn_{\transitionParams}}}
\newcommand{\valueFn}{\ensuremath{\mathbf{Q}}}
\newcommand{\stateValueFn}{\ensuremath{\mathbf{V}}}
% \newcommand{\valueFn}{\ensuremath{\mathbf{Q}_{\valueFnParams}}}
\newcommand{\policy}{\ensuremath{\pi}}
\newcommand{\pPolicy}{\ensuremath{\pi_{\policyParams}}}

\begin{document}

\maketitle

\begin{abstract}

Building systems that adapt to new information and express uncertainty is essential to deploy methods in real-world applications e.g. healthcare, autonomous vehicles. While gradient-based training is symbiotic to modern deep learning, it is rigid regarding adding further information, usually involving retraining from scratch. In a functional view, we can view data as the parameters of our model instead of the weights, which we call the dual parameterization of a model. Using the formulation we can assimilate data without gradient-based retraining. Our proposed method allows for deployment in any sequential decision-making process. We show its effectiveness in model-based reinforcement learning using updated information for more efficient exploration.

\end{abstract}


\section{Introduction} \label{sec:intro}

%Explain uncertainty in sequential decision making but need for neural networks.
Sequential decision-making requires uncertainty estimates for effective exploration \citep{srinivas2009gaussian}. For this reason, Gaussian processes (GPs) are a standard surrogate model for Bayesian optimization \citep{garnett_bayesoptbook_2022} and in model-based reinforcement learning \citep{deisenroth2011pilco}. Yet many problems lie in high dimensional input space; for example, images are where GPs cannot learn representations. However, uncertainty is still essential to ensure effective exploration for sequential algorithms \todo{cite here}. Successful approaches have attempted to blend neural networks with uncertainty estimates around predictions, allowing for sophisticated exploration strategies.  [ensembles of NN and citations]

% Intoduce the laplace GNN 

%Need for adaptive learning methods + failures with current methods
Whether using a Gaussian process or a neural network it is often the case that the surrogate model obtains new information which it wants to incorporate in to its 

%Need uncertainty and adaptive methods

%Dual formulation in GPs space solves this

%Talk about planning and exploration in RL.

%List the contributions.
The contributions of the paper our is as follows:
\begin{itemize}
\item We show how to take a trained neural network and convert it to a dual sparse GP. We are able to do this without retraining a variational objective for the Sparse GP. Our sparse GP uses the variational formulation, and thus gives better uncertainty estimates than other no variational sparse approaches used previously.
\item The sparse GP now gives us a compact representation of our parameters in the function space. We can therefore take advantage of the dual parameters formulation for fast conditioning of new data in to our posterior avoiding retraining of the neural network. Crucially this allows for fast adaptation of models that our used in sequential decision making. We show how this is effective in the planning stage of model-based reinforcement exploration.
\end{itemize}



\begin{figure}[t!]
  \setlength{\figurewidth}{.48\textwidth}
  \setlength{\figureheight}{.5\figurewidth}
  \begin{subfigure}{.48\textwidth}
    \centering
    \tikz\node[fill=black!10,minimum width=\figurewidth,minimum height=\figureheight,rounded corners=5pt]{};
  \end{subfigure}
  \hfill
  \begin{subfigure}{.48\textwidth}
    \centering
    \tikz\node[fill=black!10,minimum width=\figurewidth,minimum height=\figureheight,rounded corners=5pt]{};
  \end{subfigure}
  \caption{Teaser figure goes here.}
  \label{fig:teaser}  
\end{figure}



\section{Background} \label{sec:background}
Sequential decision making requires uncertainty estimates for effective exploration. 


In this section, we recap how can obtain a GP posterior from a trained deep neural network (DNN).
% We start by obtaining a posterior over the weights of a trained neural network via the Laplace approximation.
% We then show how we can obtain a function-space posterior, (i.e. a GP posterior)
% by linearising the neural network around the weights Maximum a Posteriori (MAP) estimate.
% We then show that by linearising the neural network we can obtain a function-space posterior, i.e. a GP posterior.
% We paraterise our single-step dynamic model $\transitionFnWithParams : \inputDomain \rightarrow \outputDomain$
% as an $L\text{-layer}$ NN with weights $\transitionParams$.
\textbf{Gaussian processes}
A Gaussian process (GP) is a distribution over real-valued functions $f(\cdot): \mathcal{X} \rightarrow \R$ defined over $\mathcal{X}$.
Formally, a GP is defined as,
\begin{align}
\label{eq-gp-prior}
  f(\cdot) \sim \mathcal{N}\left( \mu(\cdot), k(\cdot,\cdot') \right) \quad
  \mu(\cdot): \mathcal{X} \rightarrow \R \quad
  k(\cdot,\cdot'): \mathcal{X} \times \mathcal{X} \rightarrow \R
\end{align}
where $\mu(\cdot)$ denotes a mean function and $k(\cdot,\cdot')$ a positive definite covariance function, also known as a kernel.
Given a data set $\dataset = \{\mathbf{x}_{n} \in \inputDomain, \mathbf{y}_{n} \in \outputDomain\}_{n=0}^{N}$,
we can obtain the GP posterior via multivariate normal conditioning,
\begin{align}
\label{eq-gp-predictive-posterior}
  p(f(\mathbf{x}_{*}) \mid \mathbf{y}) &= \mathcal{N}
  % \left( f(\mathbf{x}_{*}) \mid \mathbf{A} \mathbf{m}^{*}, \mathbf{A}\mathbf{K}_{\mathbf{x}\mathbf{x}}^{-1} \mathbf{A}^{T} \right) \\
  \left( f(\mathbf{x}_{*}) \mid \mu(\mathbf{x}_{*}) + \mathbf{k}_{*\mathbf{x}} \mathbf{K}^{-1}_{\mathbf{x}\mathbf{x}} (\mathbf{y} - \bm\mu_{\mathbf{X}})),
  k_{**} - \mathbf{k}_{*\mathbf{x}} \left(\mathbf{K}_{\mathbf{x}\mathbf{x}} \right)^{-1} \mathbf{k}_{*\mathbf{x}}^{T} \right)
\end{align}
% Our GP posterior is then given by,
\todo{add mean func to GP posterior if we're using one}
\begin{align}
\label{eq-gp-predictive-posterior}
  p(f(\mathbf{x}_{*}) \mid \mathbf{y}) &= \mathcal{N}
  % \left( f(\mathbf{x}_{*}) \mid \mathbf{A} \mathbf{m}^{*}, \mathbf{A}\mathbf{K}_{\mathbf{x}\mathbf{x}}^{-1} \mathbf{A}^{T} \right) \\
  \left( f(\mathbf{x}_{*}) \mid \mu(\mathbf{x}_{*}) + \mathbf{k}_{*\mathbf{x}} \mathbf{K}^{-1}_{\mathbf{x}\mathbf{x}} \mu_{\mathbf{x}},
  k_{**} - \mathbf{k}_{*\mathbf{x}} \left(\mathbf{K}_{\mathbf{x}\mathbf{x}} + \bm\Lambda(\mathbf{y} ; \mathbf{f})^{-1} \right)^{-1} \mathbf{k}_{*\mathbf{x}}^{T} \right)
\end{align}
where $\bm\Lambda(\mathbf{y} ; \mathbf{f}) = \nabla^{2}_{\mathbf{f} \mathbf{f}} \log p(\mathbf{y} \mid \mathbf{f})$ can be viewed as per-input noise. \todo{I'm unsure about this per input noise bit. Check with paul.}
\todo{what about regression/classification likelhoods}
This GP predictive posterior is computationally expensive as it requires inverting an $N\times N$ matrix.

\textbf{Sparse Gaussian processes}
% The GP predictive posterior in \cref{eq-gp-predictive-posterior} is computationally expensive as it requires inverting an $N\times N$ matrix.
A practical and computationally appealing alternative is to augment the joint probability space with pseudo inputs $\mathbf{Z} \in \R^{M \times D}$
and corresponding outputs
$\mathbf{u} = f(\mathbf{Z}; \transitionParams) \in \R^{M \times C}$, known as inducing variables, where $M \ll N$.
\todo{cite sparse gp papers}
Instead of collapsing the inducing variables like \cite{titsiasVariational2009}, they can be treated variationally
$p(\mathbf{u} \mid \mathbf{y}) \approx q(\mathbf{u}) = \mathcal{N}\left( \mathbf{u} \mid \mathbf{m}, \mathbf{V} \right)$, where $\mathbf{m}$ and $\mathbf{V}$ are variational parameters
which need to be optimised \citep{hensmanGaussian2013}.
The sparse variational GP (SVGP) predictive posterior is then given by,
\begin{align}
\label{eq-dual-svgp-predictive-posterior}
  p(f(\mathbf{x}_{*}) \mid \mathbf{y})
&\approx \mathcal{N} \left( f(\mathbf{x}_{*}) \mid \mathbf{k}_{*\mathbf{z}} \mathbf{K}^{-1}_{\mathbf{z}\mathbf{z}} \mathbf{m}^{*},
  k_{**} - \mathbf{k}_{*\mathbf{z}} \mathbf{K}^{-1}_{\mathbf{z}\mathbf{z}} \mathbf{k}_{*\mathbf{z}}^{T}
  % \mathbf{k}_{*\inducingInput} \mathbf{K}^{-1}_{\mathbf{z}\mathbf{z}} \mathbf{k}_{*\inducingInput}^{T}
  + \mathbf{k}_{*\mathbf{z}} \mathbf{K}^{-1}_{\mathbf{z}\mathbf{z}}  \mathbf{V}^{*}  \mathbf{K}^{-1}_{\mathbf{z}\mathbf{z}} \mathbf{k}_{*\mathbf{z}}^{T}
  \right)
\coloneqq q_{\inducingVariable}(f(\mathbf{x}_{*}))
\end{align}


\textbf{Deep learning}
Given the data set $\dataset = \{\mathbf{x}_{n} \in \inputDomain, \mathbf{y}_{n} \in \outputDomain\}_{n=0}^{N}$,
the goal of (supervised) deep learning, is to train the weights $\transitionParams \in \R^{P}$ of an $L\text{-layer}$ NN
$f : \inputDomain \rightarrow \outputDomain$ to minimize the (regularized) empirical risk,
\begin{align} \label{eq-empirical-risk}
  \transitionParams^{*} = \arg \min_{\transitionParams \in \R^{D}} \mathcal{L}(\dataset;\transitionParams) = \arg \min_{\transitionParams \in \R^{D}} \left(
  \sum_{n=0}^{N-1} l(f(\mathbf{x}_{n} ; \transitionParams), \mathbf{y}_{n}) + R(\transitionParams)  \right).
\end{align}
In practice, it is common to use the mean squared error (MSE) loss
$l(f(\mathbf{x}_{n} ; \transitionParams),\mathbf{y}_{n}) = \|f(\mathbf{x}_{n} ; \transitionParams) - \mathbf{y}_{n} \|^{2}_{2}$
and to use the weight decay regularizer $R(\transitionParams)=\frac{1}{2}\gamma^{-2}\|\transitionParams\|^{2}_{2}$.
However, the practioner is free to choose any loss and regularizer. \todo{I think?}
% \textbf{Bayesian neural networks}
% \textbf{Weight space to function space}
From a Bayesian perspective, we can interpret the terms in \cref{eq-empirical-risk} as the log-prior and the i.i.d. log likelihood, i.e.,
\begin{align} \label{eq-log-prior}
  R(\transitionParams) &= \underbrace{\log p(\transitionParams)}_{\text{log prior}} = \log \mathcal{N}(\transitionParams \mid 0, \gamma^{2} \mathbf{I}) \\
  l(f(\mathbf{x}_{n} ; \transitionParams), \mathbf{y}_{n}) &= \underbrace{\log p(\mathbf{y}_{n} \mid f(\mathbf{x}_{n}; \transitionParams))}_{\text{log likelihood}}
  = \log \mathcal{N} \left( \mathbf{y}_{n} \mid f(\mathbf{x}_{n}; \transitionParams), \mathbf{I} \right) \label{eq-log-likelihood}
\end{align}

\textbf{NN to GP}
Given this interpretation, we can represent the weight space prior from \cref{eq-log-prior}, in
function space, by linearising the BNN around $\transitionParams^{*}$ and interpretting it as a GP.
The GP prior is then given by,
\begin{align} \label{eq-laplace-approx-function-space}
  % \transitionFn_{\transitionParams_{i}}(\cdot) \sim \mathcal{N} \left( \mu_{\transitionParams_{i}^{*}}(\cdot), K_{\transitionParams_{i}^{*}}(\cdot, \cdot') \right) \qquad
  f(\cdot ;\transitionParams) \sim \mathcal{N} \left( \mu(\cdot), k(\cdot, \cdot') \right) \quad
  % \mu_{\transitionParams_{i}^{*}}(\cdot)
  \mu(\cdot)
  = 0 \quad
  % = f(\cdot ;\transitionParams^{*}) \quad
  % K_{\transitionParams_{i}^{*}}(\cdot, \cdot')
  k(\cdot, \cdot')
  = \frac{1}{\gamma^{2}} \mathbf{J}(\cdot) \mathbf{J}(\cdot')^{T},
\end{align}
where the kernel is the neural tangent kernel (NTK) \citep{immerImprovingPredictionsBayesian2021},
\todo{what mean function shoudl this use???}
\todo{NTK times some factor??}
with Jacobian given by $\mathbf{J}(\cdot) = \odv{\transitionFn(\cdot; \transitionParams)}{\transitionParams}_{\transitionParams=\transitionParams^{*}}$.
% The kernel in \cref{eq-laplace-approx-function-space} is the neural tangent kernel (NTK)
% $k(\mathbf{x}, \mathbf{x}') = \sigma_{0}^{2} J_{\transitionParams^{*}}(\mathbf{x}) J_{\transitionParams^{*}}(\mathbf{x}')^{T}$
% \citep{immerImprovingPredictionsBayesian2021}
% to formulate the GP posterior $p(f(\mathbf{X} ;\transitionParams_{0:i}) \mid \dataset_{0:i})$.
Given this GP prior, we can use \cref{eq-gp-predictive-posterior} to obtain our GP posterior by conditioning on the data.
% Given our GP prior, we can use the properties of multivariate normals to obtain our GP posterior by conditioning on the data.
% where $\mathbf{A} = \mathbf{k}_{*\mathbf{x}} \mathbf{K}^{-1}_{\mathbf{x}\mathbf{x}}$
In contrast to conventional GP kernels, the NTK does not have any hyperparameters which need to be learned.
Intuitively, we can view the optimisation in \cref{eq-empirical-risk} as learning our kernel.
As a result, our NTK may be highly non stationary as it is dependent on the NN architecture, for example, the activation functions.
It is worth noting that the NTK linearises the network around the optimised parameters $\transitionParams^{*}$,
so the function space prior (and thus posterior) is only a locally linear approximation.

We now have a method to obtain a GP posterior from a trained DNN.
% In contrast to other approaches, our formulation has not restricted us to using a weight-space posterior obtained from the Laplace approximation.
In contrast to other approaches, our formulation has not restricted us to using a weight-space posterior from a BNN.
However, in practice, the GP posterior is not very useful as it requires inverting an $N\times N$ matrix which has complexity $\mathcal{O}(N^{3})$.
The SVGP in \cref{eq-dual-svgp-predictive-posterior} offers a solution to this problem but it is not immediately clear how to set $\mathbf{m}$ and $\mathbf{V}$.
In the next section, we show how we can directly obtain an SVGP posterior (i.e. set $\mathbf{m}$ and $\mathbf{V}$) directly from both
1) a trained NN and 2) a BNN weight-space posterior.
% $\mathbf{m}$ and $\mathbf{V}$
% However, it is not obvious how we can use a SVGP to set $\mathbf{m}$ and $\mathbf{V}$



% In contrast, to previous work, this approach did not require us Laplace approximation. we will not restrict our
% It is worth noting that the NTK linearises the network around the optimised parameters $\transitionParams^{*}$,
% so the function space prior (and thus posterior) is only a locally linear approximation.
% In contrast to conventional GP kernels, the NTK does not have any hyperparameters which need to be learned.
% It is also worth noting that the NTK may be highly non stationary and is dependent on the NN architecture, for example, the activation functions.


\section{Methods}
In the next section we show how we can obtain a SVGP directly from 1) a trained NN and 2) a BNN posterior.

We follow \cite{csatoSparseOnlineGaussian2002} and express our GP posterior in the dual parameter space.
However, we consider the variational formulation introduced by \cite{adamDualParameterizationSparse2021}, where the optimal variational parameters are given by,
\begin{align} \label{eq-dual-params}
\mathbf{m}^{*} = \mathbf{V}^{*}\bm\alpha^{*} \quad \mathbf{V}^{*} = [\mathbf{K}_{\mathbf{z}\mathbf{z}}^{-1} + \bm\beta^{*}]
\end{align}
Importantly, we can calculate the natural parameters $(\bm\alpha, \bm\beta)$ in closed form,
\todo{paul needs to update \cref{eq-dual-svgp-params}}
\begin{align} \label{eq-dual-svgp-params}
\bm\alpha_{n} &=  \nabla_{\meanParam{1}} \mathbb{E}_{q_{\inducingVariable}(f(\mathbf{x}_{n}))} \left[ \log p(\mathbf{y}_{n} \mid f(\mathbf{x}_{n}) ) \right] \\
\bm\beta_{n} &=  \nabla_{\meanParam{1}} \mathbb{E}_{q_{\inducingVariable}(f(\mathbf{x}_{n}))} \left[ \log p(\mathbf{y}_{n} \mid f(\mathbf{x}_{n}) ) \right].
\end{align}
This simplifies SVGP inference \citep{hensmanGaussian2013} as it no longer requires the variational parameters $(\mathbf{m}, \mathbf{V})$ to be optimised.

% Importantly, \cite{changFantasizingDualGPs2022} show that in the dual space,
% conditioning on new observations reduces to suming the dual variables from the previous time
% step with an update, given by,
% \todo{pick best way to show new data (using superscript new or just time indexing?)}
% \begin{align} \label{eq-dual-update-svgp}
% \bm\alpha_{n} &=  \nabla_{\meanParam{1}} \mathbb{E}_{q_{\inducingVariable}(f(\mathbf{x}_{n}))} \left[ \log p(\mathbf{y}_{n} \mid f(\mathbf{x}_{n}) ) \right] \\
% \bm\beta_{n} &=  \nabla_{\meanParam{1}} \mathbb{E}_{q_{\inducingVariable}(f(\mathbf{x}_{n}))} \left[ \log p(\mathbf{y}_{n} \mid f(\mathbf{x}_{n}) ) \right]
% \end{align}

% We follow \cite{csatoSparseOnlineGaussian2002} and express our GP posterior in the dual parameter space,
% % to write GP posterior $p(f(\mathbf{X} ;\transitionParams_{0:i}) \mid \dataset_{0:i})$ as,
% \begin{align} \label{eq-gp-post}
%   \mathbb{E}_{p(f(\mathbf{x}_{*}) \mid \mathbf{y}_{n})} \left[ f(\mathbf{x}_{*}) \right] &= \mu(\mathbf{x}_{*}) + \mathbf{k}_{\mathbf{x}*}^{T} \bm\alpha_{n} \\
%   \mathbb{V}_{p(f(\mathbf{x}_{*}) \mid \mathbf{y}_{n})} \left[ f(\mathbf{x}_{*}) \right] &= k_{**} - \mathbf{k}_{\mathbf{x}*}^{T} \left( \mathbf{K}_{\mathbf{x} \mathbf{x}} + \text{diag}(\bm\beta_{n})^{-1} \right)^{-1} \mathbf{k}_{\mathbf{x}*}
% \end{align}
% % \begin{align} \label{eq-laplace-approx-function-space}
% %   \mathbb{E}_{p(f_{*} \mid \mathbf{y}_{n})} \left[ f_{*} \right] &= \mathbf{k}_{\mathbf{x}*}^{T} \bm\alpha_{n} \\
% %   \mathbb{V}_{p(f_{*} \mid \mathbf{y}_{n})} \left[ f_{*} \right] &= k_{**} - \mathbf{k}_{\mathbf{x}*}^{T} \bm\alpha_{n} \left( \mathbf{K}_{\mathbf{x} \mathbf{x}} + \diag(\bm\beta_{n})^{-1} \right)^{-1} \mathbf{k}_{\mathbf{x}*}
% % \end{align}
% where the dual parameters $(\bm\alpha_n, \bm\beta_n)$ are vectors of,
% \begin{align} \label{eq-gp-dual-params}
%   \bm\alpha_{n} \coloneqq \left\{ \mathbb{E}_{p(f(\mathbf{x}_{i}) \mid \mathbf{y}_{n})} \left[ \nabla_{f_{i}} \log p(y_{i} \mid f(\mathbf{x}_{i})) \right] \right\}_{i=1}^{N} \qquad
%   \bm\beta_{n} \coloneqq \left\{ \mathbb{E}_{p(f(\mathbf{x}_{i}) \mid \mathbf{y}_{n})} \left[ \nabla^{2}_{f_{i}} \log p(y_{i} \mid f(\mathbf{x}_{i})) \right] \right\}_{i=1}^{N}
% \end{align}
% Importantly, this parameterisation

% \textbf{dual-SVGP}
% \begin{align} \label{eq-svgp-post}
%   \mathbb{E}_{q_{\mathbf{u}}(f(\mathbf{x}_{i}))} \left[ f(\mathbf{x}_{*}) \right] &= \mu(\mathbf{x}_{*}) + \mathbf{k}_{\mathbf{z}*}^{T} \bm\alpha_{\mathbf{u}} \\
%   \mathbb{V}_{q_{\mathbf{u}}(f(\mathbf{x}_{i}))} \left[ f(\mathbf{x}_{*}) \right] &= k_{**} - \mathbf{k}_{\mathbf{z}*}^{T} \left[ \mathbf{K}_{\mathbf{z}\mathbf{z}}^{-1} - \left( \mathbf{K}_{\mathbf{z} \mathbf{z}} + \bm\beta_{\mathbf{u}} \right)^{-1} \right] \mathbf{k}_{\mathbf{z}*}
% \end{align}
% % \begin{align} \label{eq-laplace-approx-function-space}
% %   \mathbb{E}_{p(f_{*} \mid \mathbf{y}_{n})} \left[ f_{*} \right] &= \mathbf{k}_{\mathbf{x}*}^{T} \bm\alpha_{n} \\
% %   \mathbb{V}_{p(f_{*} \mid \mathbf{y}_{n})} \left[ f_{*} \right] &= k_{**} - \mathbf{k}_{\mathbf{x}*}^{T} \bm\alpha_{n} \left( \mathbf{K}_{\mathbf{x} \mathbf{x}} + \diag(\bm\beta_{n})^{-1} \right)^{-1} \mathbf{k}_{\mathbf{x}*}
% % \end{align}
% where the dual parameters $(\bm\alpha_{\mathbf{u}}, \bm\beta_{\mathbf{u}})$ are vectors of,
% \begin{align} \label{eq-svgp-dual-params}
%   \bm\alpha_{\mathbf{u}} \coloneqq \mathbf{k}_{\mathbf{z}i} \mathbb{E}_{q_{\mathbf{u}}(f(\mathbf{x}_{i}))} \left[ \nabla_{f_{i}} \log p(y_{i} \mid f(\mathbf{x}_{i})) \right] \qquad
%   \bm\beta_{\mathbf{u}} \coloneqq \mathbf{k}_{\mathbf{z}i} \mathbb{E}_{q_{\mathbf{u}}(f(\mathbf{x}_{i}))} \left[ \nabla^{2}_{f_{i}} \log p(y_{i} \mid f(\mathbf{x}_{i})) \right] \mathbf{k}_{\mathbf{z}i}^{T}
% \end{align}

% weight space posterior in \cref{eq-laplace-approx-weight-space} in
% function space by linearising the BNN and interpreting it as a GP.
% \begin{align} \label{eq-laplace-approx-function-space}
%   \transitionFn_{\transitionParams_{i}}(\cdot) &\sim \mathcal{N} \left( \mu_{\transitionParams_{i}^{*}}(\cdot), K_{\transitionParams_{i}^{*}}(\cdot, \cdot') \right) \quad
%   \mu_{\transitionParams_{i}^{*}}(\cdot) &= \transitionFn_{\transitionParams^{*}_{i}}(\cdot) \quad
%   K_{\transitionParams_{i}^{*}}(\cdot, \cdot') &= \mathbf{J}_{\transitionParams^{*}_{i}}(\cdot) \bm\Sigma_{\transitionParams^{*}_{i}} \mathbf{J}_{\transitionParams^{*}_{i}}(\cdot')^{T}
% \end{align}
% where the Jacobian is given by $\mathbf{J}_{\transitionParams^{*}_{i}}(\cdot) = \odv{\transitionFnWithParams(\cdot)}{\transitionParams}_{\transitionParams=\transitionParams_{i}^{*}}$.

% \begin{align} \label{eq-laplace-approx-jacobian}
% \mathbf{J}_{\transitionParams^{*}_{i}}(\cdot) = \odv{\transitionFnWithParams(\cdot)}{\transitionParams}_{\transitionParams=\transitionParams_{i}^{*}}
% \end{align}


% % Note that the posterior in \cref{eq-laplace-approx-weight-space} is in weight space.
% We can represent the weight space posterior in \cref{eq-laplace-approx-weight-space} in
% function space by linearising the BNN and interpreting it as a GP.
% \begin{align} \label{eq-laplace-approx-function-space}
%   \transitionFn_{\transitionParams_{i}^{*}}(\cdot) &\sim \mathcal{N} \left( \mu_{\transitionParams_{i}^{*}}(\cdot), K_{\transitionParams_{i}^{*}}(\cdot, \cdot') \right) \quad
%   \mu_{\transitionParams_{i}^{*}}(\cdot) &= \transitionFn_{\transitionParams^{*}_{i}}(\cdot) \quad
%   K_{\transitionParams_{i}^{*}}(\cdot, \cdot') &= \mathbf{J}_{\transitionParams^{*}_{i}}(\cdot) \bm\Sigma_{\transitionParams^{*}_{i}} \mathbf{J}_{\transitionParams^{*}_{i}}(\cdot')^{T}
% \end{align}
% where the Jacobian is given by $\mathbf{J}_{\transitionParams^{*}_{i}}(\cdot) = \odv{\transitionFnWithParams(\cdot)}{\transitionParams}_{\transitionParams=\transitionParams_{i}^{*}}$.

% \begin{align} \label{eq-laplace-approx-jacobian}
% \mathbf{J}_{\transitionParams^{*}_{i}}(\cdot) = \odv{\transitionFnWithParams(\cdot)}{\transitionParams}_{\transitionParams=\transitionParams_{i}^{*}}
% \end{align}


\subsection{Fast Updates}
Importantly, \cite{changFantasizingDualGPs2022} show that in the dual space,
conditioning on new observations reduces to suming the dual variables from the previous time
step with an update, given by,
\todo{pick best way to show new data (using superscript new or just time indexing?)}
\begin{align} \label{eq-dual-update-svgp}
\dualParam{1}^{t+1} &\leftarrow \dualParam{1}^{t} +  \nabla_{\meanParam{1}} \mathbb{E}_{q_{\inducingVariable}(\latentFn(\state_{t}, \action_{t}))} \left[ \log p(\state_{t+1} \mid \latentFn(\state_{t}, \action_{t}) ) \right] \\
\dualParam{2}^{t+1} &\leftarrow \dualParam{2}^{t} +  \nabla_{\meanParam{2}} \mathbb{E}_{q_{\inducingVariable}(\latentFn(\state_{t}, \action_{t}))}  \left[ \log p(\state_{t+1} \mid \latentFn(\state_{t}, \action_{t}) ) \right]
\end{align}
Importantly, for a single new observation $((\state_{t}, \action_{t}), \state_{t+1})$ this update has
complexity $\mathcal{O}(\numInducing^{2})$.
This is a significant improvement to naive GP conditioning, which has complexity $\mathcal{O}((\numDataNew + \numDataOld)^{3})$
and sparse GP conditioning, which has complexity $\mathcal{O}((\numDataNew + \numDataOld)\numInducing^{2})$.
\todo{double check these complexities are right and cite them/show equaitons}
It is worth highlighting that the complexity of \cite{changFantasizingDualGPs2022} update does not increase during an episode.

\subsubsection{Model-based RL}

\subsubsection{Efficiently Sampling Functions for Posterior Sampling}
\cite{wilsonEfficiently2020}
\cite{wilsonPathwise2021}

% \begin{align} \label{eq-dual-update-svgp}
%  \dualParam{1}^{\text{new}} &\leftarrow \dualParam{1}^{\text{old}} +
%   \nabla_{\meanParam{1}} \mathbb{E}_{q_{\inducingVariable}(\latentFn(\state_t^{\text{new}}, \action_t^{\text{new}}))}
%  \left[ \log p(\state_{t+1}^{\text{new}} \mid \latentFn(\state_t^{\text{new}}, \action_t^{\text{new}}) ) \right] \\
%  \dualParam{2}^{\text{new}} &\leftarrow \dualParam{2}^{\text{old}} +
%   \nabla_{\meanParam{2}} \mathbb{E}_{q_{\inducingVariable}(\latentFn(\state_t^{\text{new}}, \action_t^{\text{new}}))}
%  \left[ \log p(\state_t^{\horizon+1} \mid \latentFn(\state_t^{\text{new}}, \action_t^{\text{new}}) ) \right] \\
%  \dualParam{1}^{\horizon+1} &\leftarrow \dualParam{1}^{\horizon} +
%   \nabla_{\meanParam{1}} \mathbb{E}_{q_{\inducingVariable}(\latentFn(\state_{\horizon}, \action_{\horizon}))}
%  \left[ \log p(\state_{\horizon+1} \mid \latentFn(\state_{\horizon}, \action_{\horizon}) ) \right] \\
%  \dualParam{2}^{\horizon+1} &\leftarrow \dualParam{2}^{\horizon} +
%   \nabla_{\meanParam{2}} \mathbb{E}_{q_{\inducingVariable}(\latentFn(\state_{\horizon}, \action_{\horizon}))}
%  \left[ \log p(\state_{\horizon+1} \mid \latentFn(\state_{\horizon}, \action_{\horizon}) ) \right]
% \end{align}


\begin{assumption} \label{assumption-ntk-linearisation}
  Something about NTK being a linearisation around $\theta^{*}_{i}$ but each update moves away from $\theta^{*}_{i}$
\end{assumption}

\cite{rossellApproximateLaplaceApproximations2021}



\section{Experiments}

% \begin{figure}
%   \centering
%   \includegraphics[width=0.5\textwidth, angle=270]{figs/weight-space-to-functio-space.pdf}
%   \caption{}
% \end{figure}

% \begin{figure}
%   \centering
%   \includegraphics[width=0.5\textwidth, trim=0 100 0 10]{figs/cartpole-training-curves.pdf}
%   \caption{}
% \end{figure}

% \begin{table}
%   \caption{Negative test log likelihood (lower is better) on UCI classification tasks (2 hidden layers, 50 tanh). Our SVGP predictive outperforms the GLM predictive. }
% \end{table}

\begin{itemize}
  \item Show prediction performance in classification: UCI classification results, compare to BNN predictive paper
  \item Fast adapatation in model-based RL: results for cartpole
  \item Is linearising NN mean and using fast updates during episode better than keeping non-linearised mean and not updating?
  \item Possible ablations
  \begin{itemize}
    \item How does number of inducing points affect performance?
  \end{itemize}
  \item How good is the assumption that we can linearise BNN to get NTK?
\end{itemize}

Environments
\begin{itemize}
  \item Compare updated posterior with non updated posterior
  \item Need to show update is fast enough for practical use?
  \begin{itemize}
    \item How frequently can we do the posterior update? Every time step or every $K$ steps?
    \item Update depends on number of inducing points?
    \item What experiments can we show for this? Wall clock time vs planning horizon with different numbers of inducing points?
  \end{itemize}
  \item Is linearising NN mean and using fast updates during episode better than keeping non-linearised mean and not updating?
  \item How does number of inducing points affect performance?
  \begin{itemize}
    \item More inducing points means better approximation but larger computational complexity.
  \end{itemize}
  \item How good is the assumption that we can linearise BNN to get NTK?
  \begin{itemize}
    \item Does model's performance get worse during an episode because linearisation becomes more wrong?
    \item SVGP does't have this issue so compare to SVGP in cartpole?
  \end{itemize}
  \item What length planning horizon to use?
  \begin{itemize}
    \item Too small (e.g. $\Horizon=0$) would mean no exploration?
    \item Can $\Horizon$ be too large?
  \end{itemize}
  \begin{itemize}
  \item Compare for different strategies of making decisions under dynamic model's uncertainty:
    \begin{itemize}
      \item \textbf{Greedy exploitation} \(\pi_{\text{greedy}}\)
      \item \textbf{Hallucinated-UCRL} \(\pi_{\text{HUCRL}}\)
      \item \textbf{Thompson sampling} \(\pi_{\text{TS}}\)
    \end{itemize}
  \end{itemize}
  \item Compare impact of stationary vs non-stationary priors
  \begin{itemize}
    \item First do this with GP
    \item Then try to do with Laplace BNN
  \end{itemize}
  \item Compare impact of function vs weight space
\end{itemize}

\section{Conclusion} \label{sec:conclusion}


\section*{Broader Impact}

\begin{ack}
ASK A ABOUT THIS SECTION:
Aidan Scannell is funded by the Finnish Center for Artificial Intelligence.
\end{ack}


% \section*{References}
%\small
%\printbibliography
%\normalsize
% TODO make bibliography small a better way

%References follow the acknowledgments. Use unnumbered first-level heading for
%the references. Any choice of citation style is acceptable as long as you are
%consistent. It is permissible to reduce the font size to \verb+small+ (9 point)
%when listing the references.
%Note that the Reference section does not count towards the page limit.
%\medskip


\phantomsection%
\addcontentsline{toc}{section}{References}
\begingroup
\small
\bibliographystyle{abbrvnat}
\bibliography{zotero-library,bibliography}
\endgroup

\clearpage




\appendix

\section{Appendix}

Optionally include extra information (complete proofs, additional experiments and plots) in the appendix.
This section will often be part of the supplemental material.

\section{Model-based Reinforcement Learning with Fast Updates}


\subsection{Problem Statement and Background} \label{sec:problem-statement}
We consider environments with states \(\state \in \stateDomain \), actions \(\action \in \actionDomain\) and transition dynamics \(\transitionFn: \stateDomain \times \actionDomain \rightarrow \stateDomain \) subject to
iid noise \(\noise_{t}\), given by,
\begin{align}
\state_{t+1} = \transitionFn(\state_{t}, \action_{t}) + \noise_{t}.
\end{align}
\begin{assumption}
  (system properties) The environment's dynamics are $L_{\transitionFn}\text{-Lipschitz}$ continuous and the transition noise $\noise_{t}$ is $\sigma\text{-sub-Gaussian}$ for all $t \geq 0$.
\end{assumption}

\subsection{Model-based Reinforcement Learning (RL)}
The goal of reinforcement learning is to find a policy \(\pi \in \Pi\) that maximises the sum of discounted
rewards in expecation under the transition noise (aleatoric uncertainty),
\begin{align} \label{eq-model-free-objective}
\policy^{*} = \arg \max_{\policy \in \policyDomain} J(\transitionFn, \policy) = \arg \max_{\policy \in \policyDomain} \mathbb{E}_{\noise_{0:\infty}} \left[ \sum_{t=0}^{\infty} \discount^{t} \rewardFn(\state_{t},\action_{t}) \right],
\end{align}
where $\gamma \in [0, 1]$

\textbf{Model-based}
In Bayesian model-based RL, we obtain the posterior over the dynamics \(p(f\mid\mathcal{D})\) after performing (approximate) Bayesian
inference given a state transition data set \(\mathcal{D} = \{\{(s_{t},a_{t}), s_{t+1}\}^{T_{i}}_{t=1}\}_{i=0}^{N}\).
\cref{alg-mbrl} shows the typical model-based RL loop.
Importantly, the dynamics are usually only updated after an episode $i$.

\begin{algorithm}[!t]
\caption{Model-based RL}\label{alg-mbrl}
\begin{algorithmic}[1]
  \Require Start state $\state_{0}$, initial data set $\dataset_{0}$, dynamics posterior $p(\transitionFn \mid \dataset_{0})$, policy $\policy_{0}$
\For{$i  \in \{1, 2, \ldots, \text{num episodes} \}$}
    \State Reset the system to $\state_{0}$ and reset trajectory buffers $\bm\tau_{t} = \emptyset \ \forall t$
    \For{$t  \in \{1, 2, \ldots, \text{num steps} \}$}
      % \State Collect  $\tau_{0:t} = \tau_{0:t-1} \cup (\state_{j}, \action_{j}, \state_{j+1}, r_{j+1})$
      \State Use \cref{eq-greedy}/\cref{eq-posterior-sampling}/\cref{eq-ucrl} to collect data $\bm\tau_{t} = \bm\tau_{t-1} \cup (\state_{t}, \policy_{i}(\state_{t}), \transitionFn(\state_{t}, \policy_{i}(\state_{t})), r_{t+1})$
      % \State Execute policy $\policy_{i}(\state_{t})$ in environment and update trajectory $\tau_{i+1} = \{\state_{j}, \action_{j}, \state_{j+1}, r_{j+1}) \}_{j=0}^{t}$
    \EndFor
    \State Update data set $\dataset_{0:i} = \dataset_{0:i-1} \cup \tau$
    \State Train dynamics $p(\transitionFn \mid \dataset_{0:i}) \leftarrow \text{update\_dynamics}(\dataset_{0:i}, p(\transitionFn \mid \dataset_{0:i-1}))$
    % \State Train dynamics $p(\transitionFn \mid \dataset_{0:i+1})$ using $\dataset_{0:i+1}$
    % \State Improve policy $\pi_{i+1}$ using $p(\transitionFn \mid \dataset_{0:i+1})$ and/or $\dataset_{0:i+1}$
    \State Improve policy $\pi_{i+1} \leftarrow \text{update\_policy}(p\left(\transitionFn \mid \dataset_{0:i}), \dataset_{0:i} \right)$
    %\State Improve policy $\pi_{i+1}$ using $p(\transitionFn \mid \dataset_{0:i+1})$ and/or $\dataset_{0:i+1}$
\EndFor
\end{algorithmic}
\end{algorithm}

% \begin{minipage}{0.499\textwidth}
% \begin{algorithm}[H]
% \caption{Model-based RL}\label{alg-mbrl}
% \begin{algorithmic}[1]
%   \Require Initial data set $\dataset_{0}$, dynamics posterior $p(\transitionFn \mid \dataset_{0})$, policy $\policy_{0}$
% \For{$i  \in \{0, 1, \ldots, \text{num episodes} \}$}
%     \For{$t  \in \{0, 1, \ldots, \text{num steps} \}$}
%       \State Execute policy $\policy_{i}(\state_{t})$ in environment
%       \State $\tau_{i+1} = \{\state_{j}, \action_{j}, \state_{j+1}, r_{j+1}) \}_{j=0}^{t}$
%     \EndFor
%     \State Update data set $\mathcal{D}_{0:i+1} = \mathcal{D}_{0:i} \cup \tau_{i+1}$
%     \State Train dynamics $p(\transitionFn \mid \dataset_{0:i+1})$
%     \State Improve policy $\pi_{i+1}$
%     %\State Improve policy $\pi_{i+1}$ using $p(\transitionFn \mid \dataset_{0:i+1})$ and/or $\dataset_{0:i+1}$
% \EndFor
% \end{algorithmic}
% \end{algorithm}
% \end{minipage}
% \hfill
% \begin{minipage}{0.499\textwidth}
% \begin{algorithm}[H]
% \caption{Model-based RL with fast updates}\label{alg-mbrl-fast-updates}
% \begin{algorithmic}[1]
%   \Require Initial data set $\dataset_{0}$, dynamics posterior $p(\transitionFn \mid \dataset_{0})$, policy $\policy_{0}$
%     % ${p(\state_{\timeInd+1} \mid \singleInput, \dataset_{0})}$}
% \For{$i  \in \{0, 1, \ldots, \text{num episodes} \}$}
%     \For{$t  \in \{0, 1, \ldots, \text{num steps} \}$}
%       \State Execute policy $\policy_{i}(\state_{t})$ in environment
%       % \State Append transition $\state_{t}, \action_{t}, \state_{t+1}, r_{t+1})$ to trajectory $\tau_{i}$
%       \State $\tau_{i+1} = \{\state_{j}, \action_{j}, \state_{j+1}, r_{j+1}) \}_{j=0}^{t}$
%       \State {\color{blue}Update dynamics $p(\transitionFn \mid \dataset_{0:i} \cup \tau_{i+1})$}
%     \EndFor
%     \State Update data set $\mathcal{D}_{0:i+1} = \mathcal{D}_{0:i} \cup \tau_{i+1}$
%     \State Train dynamics $p(\transitionFn \mid \dataset_{0:i+1})$
%     \State Improve policy $\pi_{i+1}$
%     %\State Improve policy $\pi_{i+1}$ using $p(\transitionFn \mid \dataset_{0:i+1})$ and/or $\dataset_{0:i+1}$
% \EndFor
% \end{algorithmic}
% \end{algorithm}
% \end{minipage}



\subsection{Exploration Strategies}
\textbf{Greedy exploitation}
Given the posterior dynamics \(p(\transitionFn \mid \mathcal{D})\),
a common approach is to simply take the expecation over both the aleatoric and epistemic uncertainty,
\begin{align} \label{eq-greedy}
\policy_{i+1}^{\text{greedy}} = \arg \max_{\policy \in \policyDomain} \mathbb{E}_{\transitionFn \sim p(\transitionFn \mid \dataset_{0:i})} \left[ J(\transitionFn, \policy) \right],
\end{align}
This approach has been widely adopted, for example, in PILCO, PETS, GP-MPC
\cite{deisenrothPILCO2011,chuaDeepReinforcementLearning2018,kamtheDataEfficient2018}.
This approach helps to alleviate model bias as the posterior ``knows what the model does not know''.
This is because the predictive posterior \(p(f(s_{t},a_{t}) \mid (s_{t},a_{t}),  \mathcal{D} )\) will be (or should be) uncertain when making
predictions far away from the training data.
The expectation considers all possible dynamics models which prevents the policy optimisation from
exploiting innacuracies in the model.
This approach has no guarantees for exploration in the general case.
However, under specific dynamics and reward structures (e.g. PILCO) this objective can achieve sublinear regret.
\todo{need to double check sublinear regret statement. And give a reference}


\textbf{Posterior sampling}
\cite{osbandWhyPosteriorSampling2017,osbandMoreEfficientReinforcement2013}
\begin{align} \label{eq-posterior-sampling}
\policy_{i+1}^{\text{PS}} = \arg \max_{\policy \in \policyDomain} \left[ J(\transitionFn, \policy) \right] \quad \text{s.t. } \transitionFn \sim p(\transitionFn \mid \dataset_{0:i})
\end{align}

\textbf{Hallucinated upper confidence RL}
A more theoretically grounded exploration strategy is UCRL \citep{jakschNearoptimal2010}, which optimises joinly over
policies and models inside the set
\(\mathcal{M} = \{ f \mid | f(s,a) - \mu_{i}(s, a) | \leq \beta_{i} \Sigma_{i}(s, a) \quad \forall s, a \in \mathcal{S} \times \mathcal{A} \}\), representing all statistically plausible
models under the posterior \(p(f(s,a) \mid \mathcal{D}_{0:i} \cup (s,a)) = \mathcal{N}(f(s,a) \mid \mu_{i}(s,a), \Sigma_{i}(s,a))\) at episode \(i\).
This strategy is given by,
\begin{align} \label{eq-ucrl}
\policy_{i+1}^{\text{UCRL}} = \arg \max_{\policy \in \policyDomain} \max_{\transitionFn \in \mathcal{M}} J(\transitionFn, \policy).
\end{align}
This strategy optimises an optimistic policy over the set of plausible dynamics models.
Although this joint optimisation is intractable in general,
\cite{curiEfficient2020} proposed a practical alternative which is detailed in \cref{sec-hucrl}.

\textbf{MPC vs policy learning}
It is worth noting that the strategies in \cref{eq-greedy,eq-posterior-sampling,eq-ucrl} can be used with both model predictive control (MPC)
techniques, such as the cross entoropy method (CEM), and model-free RL techniques, such as soft actor-critic (SAC).


In this work we are interested in how we can use \(p(f \mid \mathcal{D})\) to alleviate some of the issues in model-based RL,
for example, model bias and the exploration-exploitation trade-off.



\todo{show how to get new $\dualParam{1}$ and $\dualParam{2}$ in Train dynamics line of \cref{alg-mbrl-fast-updates}}
\begin{algorithm}[!t]
\caption{Model-based RL with fast updates}\label{alg-mbrl-fast-updates}
\begin{algorithmic}[1]
  \Require Start state $\state_{0}$, initial data set $\dataset_{0}$, dynamics posterior $p(\transitionFn \mid \dataset_{0})$ (inc. dual parameters $\dualParam{1}, \dualParam{2}$), policy $\policy_{0}$
    % ${p(\state_{\timeInd+1} \mid \singleInput, \dataset_{0})}$}
\For{$i  \in \{1, 2, \ldots, \text{num episodes} \}$}
    \State Reset the system to $\state_{0}$ and reset trajectory buffers $\bm\tau_{t} = \emptyset \ \forall t$
    \For{$t  \in \{1, 2, \ldots, \text{num steps} \}$}
      % \State Execute policy $\policy_{i}(\state_{t})$ (\cref{eq-fast-update-mpc}) in environment
      \State Use \cref{eq-fast-update-mpc} to collect data $\bm\tau_{t} = \bm\tau_{t-1} \cup (\state_{t}, \policy_{i}(\state_{t}), \transitionFn(\state_{t}, \policy^{\text{fast}}_{i}(\state_{t})), r_{t+1})$
      % \State Append transition $\state_{t}, \action_{t}, \state_{t+1}, r_{t+1})$ to trajectory $\tau_{i}$
      % \State $\tau_{i+1} = \{\state_{j}, \action_{j}, \state_{j+1}, r_{j+1}) \}_{j=0}^{t}$
      %\State {\color{blue}Update dynamics $p(\transitionFn \mid \dataset_{0:i} \cup \tau_{i+1})$}
      \State {\color{blue}Update dynamics posterior using \cref{eq-dual-update-svgp}, i.e. fast update}
      % \begin{align}
      % \dualParam{1}^{t+1} &\leftarrow \dualParam{1}^{t} +  \nabla_{\meanParam{1}} \mathbb{E}_{q_{\inducingVariable}(\latentFn(\state_{t}, \action_{t}))} \left[ \log p(\state_{t+1} \mid \latentFn(\state_{t}, \action_{t}) ) \right] \\
      % \dualParam{2}^{t+1} &\leftarrow \dualParam{2}^{t} +  \nabla_{\meanParam{2}} \mathbb{E}_{q_{\inducingVariable}(\latentFn(\state_{t}, \action_{t}))}  \left[ \log p(\state_{t+1} \mid \latentFn(\state_{t}, \action_{t}) ) \right]
      % \end{align}}
    \EndFor
    \State Update data set $\dataset_{0:i} = \dataset_{0:i-1} \cup \tau$
    \State Train dynamics $p(\transitionFn \mid \dataset_{0:i}) \leftarrow \text{update\_dynamics}(\dataset_{0:i}, p(\transitionFn \mid \dataset_{0:i-1}))$
    \State Improve policy $\pi^{\text{fast}}_{i+1} \leftarrow \text{update\_policy}(p\left(\transitionFn \mid \dataset_{0:i}), \dataset_{0:i} \right)$
\EndFor
\end{algorithmic}
\end{algorithm}


\subsubsection{Fast updates}


In this section we extend these fast updates to environment's with high dimensional state spaces and large data sets.
% Our method draws on the connection between BNNs and GPs and formulates a function space SVGP posterior given
Our method uses a BNN dynamic model and draws on the connection between BNNs and GPs
\citep{khanApproximate2019} to formulate a function space SVGP posterior,
where we can apply the fast updates from \cref{eq-dual-update-svgp}.
At a high-level, we first use Laplace's approximation to obtain a weight space posterior for our BNN.
We then linearise our BNN around the optimal parameters and interpret it as a GP.
Finally, we formulate a lower rank approximation of this GP posterior (i.e. a SVGP posterior) using inducing variables.
% formulates a function space SVGP posterior by drawing on the connection between BNNs and GPs.


The strategies in \cref{eq-greedy,eq-posterior-sampling,eq-hucrl} do not update the dynamic model during an episode.
A better approach would be to update the posterior at every time step during an episode, for example,
\begin{subequations}
\begin{align} \label{eq-fast-update-mpc}
  \policy_{i+1}^{\text{greedy}}(\state) &= \arg \max_{\action_{0}} \max_{\action_{1:\Horizon}}
\E_{p(\transitionFn \mid \dataset_{0:i})} \left[J^{\Horizon}(\action_{0:\Horizon}, \transitionFn) \right] + \stateValueFn(\state_{\Horizon+1}) \\
  \policy_{i+1}^{\text{PS}}(\state) &= \arg \max_{\action_{0}} \max_{\action_{1:\Horizon}}
J^{\Horizon}(\action_{0:\Horizon}, \transitionFn) + \stateValueFn(\state_{\Horizon+1}) \quad \text{s.t. } \transitionFn \sim p(\transitionFn \mid \dataset_{0:i} \cup \bm\tau_{t}) \\
  \policy_{i+1}^{\text{UCRL}}(\state) &= \arg \max_{\action_{0}} \max_{\action_{1:\Horizon}} \max_{\transitionFn \in \mathcal{M}}
J^{\Horizon}(\action_{0:\Horizon}, \transitionFn) + \stateValueFn(\state_{\Horizon+1}) \quad \text{s.t. } \mathcal{M} = \{\transitionFn(\state_{t},\action_{t}) - \mu_{i,t}(\state_{t},\action_{t}) \leq \beta_{i,t} \Sigma_{i,t}(\state_{t}, \action_{t})\} \\
  \stateValueFn(\state) &= \mathbb{E} \left[ \sum_{t=0}^{\infty}     \discount^{t} \rewardFn(\state_{t},\action_{t}) \mid \state_{0}=\state \right] \label{eq-value-fn}
\end{align}
\end{subequations}
\begin{subequations}
\begin{align} \label{eq-fast-update-mpc-old}
  \policy^{\text{fast}}(\state) = \arg &\max_{\action_{0}} \max_{\action_{1}, \ldots, \action_{\Horizon}}
  \mathbb{E}_{\state_{\horizon} \sim p(\state_{\horizon+1} \mid \transitionFn(\state_{\horizon}, \action_{\horizon}))} \left[ \sum_{\horizon=0}^{\Horizon}     \discount^{\horizon} \rewardFn(\state_{\horizon},\action_{\horizon}) \mid \state_{0}=\state \right] + \discount^{\Horizon+1} \stateValueFn(\state_{\Horizon+1}) \\
  \stateValueFn(\state) &= \mathbb{E} \left[ \sum_{t=0}^{\infty}     \discount^{t} \rewardFn(\state_{t},\action_{t}) \mid \state_{0}=\state \right] \label{eq-fast-update-mpc}
\end{align}
\end{subequations}

\begin{align} \label{}
  \policy(\state) = \arg &\max_{\action_{0}} \max_{\action_{1}, \ldots, \action_{\Horizon}} \max_{\optimisticTransition \in \optimisticTransitionSet}
  \sum_{\horizon=0}^{\Horizon}  \mathbb{E}_{\noise_{\horizon}} \left[  \discount^{\horizon} \rewardFn(\state_{\horizon},\action_{\horizon}) \right] + \discount^{\Horizon+1} \stateValueFn(\state_{\Horizon+1}) \\
  \text{s.t. } \state_{\horizon+1} &= \optimisticTransition(\state_{\horizon}, \action_{\horizon}) + \noise_{\horizon} \\
  \optimisticTransition(\state_{\horizon}, \action_{\horizon}) &=
\optimisticTransitionMean(\state_{\horizon}, \action_{\horizon}) \pm \beta_{i}
\optimisticTransitionCov(\state_{\horizon}, \action_{\horizon})
\end{align}


\section{Assumptions}\label{sec:assumptions}
This section is a draft, possibly contains mistakes
Assumptions on the model required for deriving regret bounds, in the spirit of \cite{curiCombining2021}. \cref{ass:lipschitz-f} and \cref{ass:lipschitz-gp} are copied from \cite{curiCombining2021}, whereas \cref{ass:calibrated} has been modified to accommodate the mean and variance changing during the episode.
\begin{assumption}\label{ass:lipschitz-f}
The true underlying function $f$ is $L_f$-Lipschitz and the elements of the noise vector $w_n$ are $\sigma$-sub-Gaussian.
\end{assumption}
\begin{assumption}[Well-calibrated]\label{ass:calibrated}
Assume that for any $\delta >0$ $\exists$ a set of sequences $\{\beta_{t, i}\}_{t>0, i=0}^{episode_length}$ such that the model mean and variance functions $\mu_{t, i}$, $\sigma_{t, i}$ satisfy for all $(s, a) \in \mathcal{S} \times \mathcal{A}$ the upper bound
\begin{equation}\label{eq:calibrated}
\text{Pr}(\| f(s, a) - \mu_{t, i}(s, a) \| \leq \beta_{t, i}\sigma_{t, i}) \geq 1 - \delta
\end{equation}
\end{assumption}

\begin{assumption}\label{ass:lipschitz-gp}
The mean and variance functions are lipschitz
\end{assumption}
\section{Proof of regret bounds}
This section is a draft, possibly contains mistakes
Plan for the proof:
\begin{itemize}
\item Following Appendix D in \cite{curiCombining2021}, derive a regret bound for the case where the model $\mu_t$ and $\sigma_t$ can be updated within the episode, if assumptions in \cref{sec:assumptions} are satisfied.
\item Show that sparse GPs satisfy \cref{ass:calibrated}
\end{itemize}



\section{Hallucinated Upper Confidence Reinforcement Learning (H-UCRL)} \label{sec-hucrl}
\cite{curiEfficient2020} introduced a tractable approximation which retains some of the theoretical guarantees whilst
being applicable with deep model-based RL.
They introduce a function \(\eta: \mathcal{S} \times \mathcal{A} \rightarrow [-1, 1]^{p}\) which acts as a hallucinated control input.
The strategy is given by,
\begin{align} \label{eq-hucrl}
\pi_i^{\text{UCRL}} = \arg \max_{\policy \in \policyDomain} \max_{\eta(\cdot) \in [-1,1]} J(\transitionFn, \policy) \quad \text{s.t.} \quad \transitionFn = \mu_{i}(\state_{t}, \action_{t}) + \beta_{i} \Sigma_{i}(\state_{t}, \action_{t}) \eta(\state_{t},\action_{t}).
\end{align}
Intuitively, \(\eta(\state,\action) \in [-1,1]\) enables the optimisation to select any dynamics model
\(\transitionFn\) within \(\pm \beta \Sigma_{i}(\state_{t}, \action_{t})\) of the posterior mean \(\mu_{i}(\state_{t}, \action_{t})\).

\section{Laplace Approximation} \label{sec-laplace-approximation}

\textbf{Laplace approximation}
The Laplace approximation \todo{add citation of original paper} constructs a Gaussian approximation of the weight-space posterior $p(\transitionParams \mid \dataset)$
by using a second-order Taylor expansion of $\mathcal{L}$ around $\transitionParams^{*}$.
The posterior approximation is given by,
\begin{align} \label{eq-laplace-approx-weight-space}
  p(\transitionParams \mid \dataset) \approx \mathcal{N} \left( \transitionParams \mid \transitionParams^{*} , \bm\Sigma_{\transitionParams^{*}} \right)
  \quad \text{with} \quad \bm\Sigma_{\transitionParams} =
 - \nabla_{\transitionParams \transitionParams}^{2} \mathcal{L} ( \dataset ; \transitionParams)|_{\transitionParams=\transitionParams^{*}}
\end{align}
That is, it sets the posterior precision to the Hessian of the loss at the optimal parameters $\transitionParams^{*}$.
It is worth noting that computing the Hessian of the loss can be computationally intractable for large networks.
The prior terms are usually trivial, so we focus on the likelihood here.
The Jacobian and Hessian of the log likelihood can be expressed per data point,
\begin{align} \label{eq-jac}
  \nabla_{\transitionParams} \log p(\mathbf{y} \mid \mathbf{f}(\mathbf{x}; \transitionParams)) &= \mathbf{J}(\mathbf{x})^{T} \mathbf{r}(\mathbf{y} ; \mathbf{f}) \\
  \nabla^{2}_{\transitionParams\transitionParams} \log p(\mathbf{y} \mid \mathbf{f}(\mathbf{x}; \transitionParams)) &= \mathbf{H}(\mathbf{x})^{T}
  \mathbf{r}(\mathbf{y};\mathbf{f}) - \mathbf{J}(\mathbf{x})^{T} \bm\Lambda(\mathbf{y};\mathbf{f}) \mathbf{J}(\mathbf{x}),
\label{eq-hess}
\end{align}
through the Jacobian $\mathbf{J} \in \R^{C \times P}$ and Hessian $\mathbf{H} \in \R^{C \times P \times P}$ of the feature extractor $\mathbf{f}(\mathbf{x}; \bm\transitionParams)$,
\begin{align} \label{eq-jac}
[\mathbf{J}(\cdot)]_{ci} = \odv{f_{c}(\cdot ; \transitionParams)}{\transitionParams_{i}}_{\transitionParams=\transitionParams^{*}} \qquad
[\mathbf{H}(\cdot)]_{cij} = \odv{f_{c}(\cdot ; \transitionParams)}{\transitionParams_{i}\transitionParams_{j}}_{\transitionParams=\transitionParams^{*}}
\end{align}
where
$\mathbf{r}(\mathbf{y}; \mathbf{f}) = \nabla_{\mathbf{f}} \log p(\mathbf{y} \mid \mathbf{f})$ can be interpreted as a residual and
$\bm\Lambda(\mathbf{y} ; \mathbf{f}) = \nabla^{2}_{\mathbf{f} \mathbf{f}} \log p(\mathbf{y} \mid \mathbf{f})$
as per-input noise.
Many approximations exist and can be used alongsied our method.
We refer the reader to \cite{daxbergerLaplace2021} for more details.
See \cref{sec-laplace-approximation} for further details on the Laplace approximation.


\section{Template stuff}
\subsection{Generate TikZ Figures from Python}
We can generate figures in \texttt{.tex} format directly from Python:
\begin{verbatim}
tikzplotlib.save("fig.tex", axis_width="\\figurewidth", axis_height="\\figureheight")
\end{verbatim}
\cref{fig:example} shows that we get nicely formatted lables/titles/etc when we include them in our paper.
\begin{figure}[h]
    \centering\footnotesize

    % Set your figure size here
    \setlength{\figurewidth}{.33\textwidth}
    \setlength{\figureheight}{.75\figurewidth}

    % Customize your plot here
    % (scale only axis applies the size to the axis box and not entire figure)
    \pgfplotsset{grid style={dotted},title={Foo},scale only axis}

    % Use the subcaption package (= subfigure) for sub-plots, that is
    % plot the separate plots separately in Python
    \begin{subfigure}{.4\textwidth}
        \centering
        % This file was created with tikzplotlib v0.10.1.
\begin{tikzpicture}

\definecolor{darkgray176}{RGB}{176,176,176}
\definecolor{steelblue31119180}{RGB}{31,119,180}

\begin{axis}[
height=\figureheight,
tick align=outside,
tick pos=left,
width=\figurewidth,
x grid style={darkgray176},
xmin=-0.05, xmax=1.05,
xtick style={color=black},
y grid style={darkgray176},
ymin=-0.0420735492403948, ymax=0.883544534048291,
ytick style={color=black}
]
\addplot [semithick, steelblue31119180]
table {%
0 0
0.111111111111111 0.110882628509953
0.222222222222222 0.220397743456122
0.333333333333333 0.327194696796152
0.444444444444444 0.429956363528356
0.555555555555556 0.527415385771866
0.666666666666667 0.618369803069737
0.777777777777778 0.701697876146735
0.888888888888889 0.77637192130066
1 0.841470984807897
};
\end{axis}

\end{tikzpicture}

    \end{subfigure}
    \hfill
    \begin{subfigure}{.4\textwidth}
        \centering
        % This file was created with tikzplotlib v0.10.1.
\begin{tikzpicture}

\definecolor{darkgray176}{RGB}{176,176,176}
\definecolor{steelblue31119180}{RGB}{31,119,180}

\begin{axis}[
height=\figureheight,
tick align=outside,
tick pos=left,
width=\figurewidth,
x grid style={darkgray176},
xmin=-0.05, xmax=1.05,
xtick style={color=black},
y grid style={darkgray176},
ymin=-0.0420735492403948, ymax=0.883544534048291,
ytick style={color=black}
]
\addplot [semithick, steelblue31119180]
table {%
0 0
0.111111111111111 0.110882628509953
0.222222222222222 0.220397743456122
0.333333333333333 0.327194696796152
0.444444444444444 0.429956363528356
0.555555555555556 0.527415385771866
0.666666666666667 0.618369803069737
0.777777777777778 0.701697876146735
0.888888888888889 0.77637192130066
1 0.841470984807897
};
\end{axis}

\end{tikzpicture}

    \end{subfigure}
    \caption{Foo}
    \label{fig:example}
\end{figure}

\subsection{Generate Tables from Python}
We can also generate tables straight from python using \href{https://github.com/astanin/python-tabulate}{tabulate}:
\begin{verbatim}
table = [["Sun",696000,1989100000],["Earth",6371,5973.6],
        ["Moon",1737,73.5],["Mars",3390,641.85]]
headers = ["Planet","R (km)", "mass (x 10^29 kg)"]
table = tabulate(table, headers=headers, tablefmt="latex")
with open("table.tex", 'w') as file:
    file.write(table)
\end{verbatim}

\begin{table}[h]
    \centering
    \begin{tabular}{lrr}
\hline
 Planet   &   R (km) &   mass (x 10\^{}29 kg) \\
\hline
 Sun      &   696000 &          1.9891e+09 \\
 Earth    &     6371 &       5973.6        \\
 Moon     &     1737 &         73.5        \\
 Mars     &     3390 &        641.85       \\
\hline
\end{tabular}
\end{table}

\subsection{Biblatex}
Rember when using biblatex to use 'parencite' for \citep{kamtheDataEfficient2018} and when using natbib to use 'citep'.

%\bibliography{biblio.bib}
\end{document}
